\documentclass[master,twoside,extern,palatino]{rgseThesis}

% für "lorem ipsum" Blindtext - sollte bei echten Arbeiten natürlich raus ;-)
\usepackage{float}
\usepackage{graphicx}
\usepackage{lipsum}

% Angaben für die Titelseite

\author{Stefan Hermann Strüder}
% \studiengang{Computervisualistik} % Default ist Informatik, bei Proposals ignoriert

\title{Featurebasierte Fehlererkennung mittels Methoden des Machine Learnings}

% \supervisor{w}{Dr. Sabine Mustermann} % Default ist Jan Jürjens
% \supervisorInfo{Muster-Institut} % Default ist Institut für Softwaretechnik

\secondSupervisor{w}{Dr. Daniel Strüber}
\secondSupervisorInfo{Chalmers University of Technology, Göteborg (Schweden)}

\externLogo{6cm}{logos/chalmers} % optional Logo des externen Partners
\externName{Department of Computer Science and Engineering} % optional Untertitel des Logos


% Literatur-Datenbank
\addbibresource{literature.bib}

% Tiefe des Inhaltsverzeichnisses; 1: bis section (empfohlen), 2: bis subsection
\setcounter{tocdepth}{2} 

\begin{document}

    % Umschalten der Sprache für englische Rubrikbezeichnungen (möglich: english, ngerman)
    \selectlanguage{ngerman}
    \pagenumbering{roman}

    % Titelseite und Erklärung
    \maketitle

    % Kurzfassung
    % !TEX root = ../thesis.tex

% Kurzfassung in Deutsch und Englisch
\begin{otherlanguage}{ngerman}
    \section*{Kurzfassung}

    Dies ist die Kurzfassung in Deutsch.
    
    \lipsum[1-1]
\end{otherlanguage}

\begin{otherlanguage}{english}
    \section*{Abstract}

    This is the abstract in English.
    
    \lipsum[2-2]
\end{otherlanguage}
\cleardoublepage

    
    % Kurzfassung
    % !TEX root = ../thesis.tex

% Kurzfassung in Deutsch und Englisch
\begin{otherlanguage}{ngerman}
    \section*{Anmerkung}

Diese Masterarbeit entstand in Teilen in Zusammenarbeit mit der Forschungsgruppe der Division of Software Engineering unter der Leitung von Thorsten Berger am Department of Computer Science and Engeneering der Chalmers Universität of Technology in Göteborg, Schweden.

\begin{figure}[H]
    \centering
    \includegraphics[width=6cm]{logos/chalmers}
\label{fig:chalmers}
\end{figure}

Mein besonderer Dank gilt Thorsten Berger für die Ermöglichung und Finanzierung dieser Zusammenarbeit. Ebenfalls gilt mein Dank dem gesamten Team der Forschungsgruppe für die Unterstützung bei Problemen und Fragen zu meiner Arbeit. Ein weiterer Dank gilt Daniel Strüber für seine Initiative zur Ermöglichung der Zusammenarbeit.

\end{otherlanguage}

\begin{otherlanguage}{english}
    \section*{Comment}

This master thesis was partly written in cooperation with the research group of the Division of Software Engineering headed by Thorsten Berger at the Department of Computer Science and Engineering of Chalmers University of Technology in Gothenburg, Sweden.

\begin{figure}[H]
    \centering
    \includegraphics[width=6cm]{logos/chalmers}
\label{fig:chalmers}
\end{figure}

My special thanks goes to Thorsten Berger for facilitating and financing this cooperation. I would also like to thank the entire team of the research group for their support in case of problems and questions concerning my work. A further thank you goes to Daniel Strüber for his initiative to make this cooperation possible.

\end{otherlanguage}
\cleardoublepage

    \tableofcontents
    \cleardoublepage

    \listoffigures   % fuer ein eventuelles Abbildungsverzeichnis
    % \cleardoublepage

    \pagenumbering{arabic}

    % Hier kommt jetzt der eigentliche Text der Arbeit
    
    % Du solltest mit \include für die einzelnen Kapitel arbeiten,
    % damit die Dokumente nicht zu lang werden

    % !TEX root = ../thesis.tex

\chapter{Introduction}

In `The Bible' \cite{Juerjens2005SSD}, Jan Jürjens defines an extension to the UML modeling language that enables modeling of security features in UML diagrams.

\paragraph{Suggestion:}
To make version control more easy, try to put each sentence on a single line.
Use your text editors wrapping option to wrap long sentences (i.e. lines) at the window borders.

\paragraph{Suggestion:}
You may optionally end each chapter by a \verb+\cleardoublepage+ command.
This flushes all floating objects (figures, tables, etc.) and makes chapters start on right pages in twoside mode.
You can safely use \verb+\cleardoublepage+ also in single sided printing, it behaves the same as \verb+\clearpage+.

\section{Some text}

\subsection{First part}

\lipsum[1-1]

\subsection{Second part}

\lipsum[2-3]

\section{Some more text}

\lipsum[4-7]

\cleardoublepage


    % !TEX root = ../thesis.tex

\chapter{Hintergrund}
\label{background}

Dieses Kapitel dient zur Einführung in die dieser Arbeit zugrundeliegenden Themen und hat das Ziel, Basiswissen für den weiteren Verlauf der Ausarbeitung aufzubauen. Dazu wird zunächst die featurebasierte Softwareentwicklung erläutert, ehe dann der Themenbereich des Machine Learnings vorgestellt wird. Dazu werden die Klassifikation und die Fehlervorhersage mittels Machine Learning erläutert.
\\
\hrule

\section{Featurebasierte Softwareentwicklung}
\label{feat-develop}

Das zentrale Konzept hinter der featurebasierten Softwareentwicklung stellen sogenannte Soft-ware-Produktlinien dar. Wie bereits in der Einleitung erwähnt wurde, beschreiben Software-Produktlinien eine Menge von ähnlichen Softwareprodukten, welche eine gemeinsame Menge von Features sowie eine gemeinsame Codebasis besitzen und sich durch die Auswahl der verwendeten Features unterscheiden, sodass eine breite Variabilität innerhalb einer Produktlinie entstehen kann \cite{Apel2013,Thuem2014}.

Der zentrale Prozess der Generierung einer Software-Produktlinie ist in \autoref{fig:spl} dargestellt. Aufgeteilt wird dieser Prozess in das \glqq Domain Engineering\grqq{} und das \glqq Application Engineering\grqq{}. Im Rahmen des \glqq Domain Engineerings\grqq{} wird ein sogenanntes Variabilitätsmodell (Variability Model) erstellt, welches die wählbaren Features und Constraints für mögliche Selektionen beschreibt \cite{Apel2013}. Gängige Implementationstechniken für Features reichen von einfachen Lösungen durch Annotationen, basierend auf Laufzeitparametern oder Präprozessor-Anweisungen, bis hin zu verfeinerten Lösungen, basierend auf erweiterten Programmiermethoden, wie zum Beispiel Aspektorientierung. In einigen dieser Implementierungstechniken wird jedes Feature als wiederverwendbares \glqq Domain Artifact\grqq{} modelliert und gekapselt. Diese können im Prozess des \glqq Application Engineerings\grqq{} in Form einer Konfiguration zusammen mit weiteren Features, im Hinblick auf die gewünschte Funktionalität der Software, ausgewählt werden. Ein Software Generator erzeugt dann die gewünschten Softwareprodukte basierend auf den bereits zuvor genannten Implementationstechniken für Features.

\begin{figure}[ht]
    \centering
    \includegraphics[width=\textwidth]{images/SPL}
    \caption{Generierung von Software-Produktlinien nach \cite{Thuem2014}\label{fig:spl}}
\end{figure}

Die in dieser Arbeit betrachtete Implementierungstechnik von Features basiert auf Anweisungen beziehungsweise Bedingungsdirektiven des C-Präprozessors. Die für diese Arbeit relevanten Direktiven lauten \texttt{\#IFDEF} und \texttt{\#IFNDEF}. Einfache Beispieleinsätze für beide Direktiven sind in \autoref{example1} und \autoref{example2} zu sehen. Sie wurden jeweils aus der wissenschaftlichen Literatur entnommen \cite{Medeiros2018,Preschern2019}. Die Direktive \texttt{\#IFDEF} leitet in \autoref{example1} den Code des Features \texttt{\_\_unix\_\_} ein, welcher mit der Anweisung \texttt{\#ENDIF} endet. Der in den Zeilen 2 bis 6 angegebene Codeteil wird genau dann ausgeführt, wenn das Feature \texttt{\_\_unix\_\_} im Rahmen der Konfiguration des Softwareproduktes definiert beziehungsweise aktiviert ist \cite{Stallmann2016}. In diesem Fall wird die Bedingung der Direktive erfolgreich erfüllt \cite{Stallmann2016}. Sie schlägt fehl, wenn das Feature nicht definiert beziehungsweise nicht aktiviert ist \cite{Stallmann2016}. Die Direktive \texttt{\#IFNDEF} wird für Code verwendet, der ausgeführt werden soll, wenn ein Feature nicht definiert ist. Im Falle des Beispiels in \autoref{example2} wird der in Zeile 3 angedeutete Code nur ausgeführt, wenn \texttt{NO\_XMALLOC} nicht aktiviert wurde.
Es besteht zudem die Möglichkeit, Features bzw. ihren Code zu verschachteln. Ein Beispiel dafür ist in \autoref{example3} angegeben. Es ist zu erkennen, dass sich der Code von \texttt{FEAT\_MZSCHEME} innerhalb des bedingten Codes von \texttt{USE\_XSMP} befindet. Der in Zeile 5 angedeutete Code kann somit nur ausgeführt werden, wenn \texttt{USE\_XSMP} aktiviert ist. Im Fall von Verschachtelung beendet ein \texttt{\#ENDIF} immer das nächstgelegene \texttt{\#IFDEF} oder \texttt{\#IFNDEF} \cite{Stallmann2016}. Es besteht zudem die Möglichkeit, Direktiven mittels \glqq und\grqq{} (\texttt{\&\&}, \texttt{and}) oder \glqq oder\grqq{} (\texttt{||}, \texttt{or}) zu erweiterten Bedingungen zu verknüpfen, die zudem Negation in Form des \texttt{!}-Operators (anstelle von \texttt{\#IFNDEF}) enthalten können \cite{Stallmann2016,Queiroz2015}. Dargestellt ist dies in \autoref{example4}.

\noindent\begin{minipage}{.45\textwidth}
\begin{lstlisting}[caption=Beispieleinsatz von \texttt{\#IFDEF} nach \cite{Preschern2019},frame=tlrb,language=C, label=example1]{example1}
#IFDEF __unix__
	#include "directorySelection.h"
	#include "directoryNames.h"
	void getDirectoryName(char* dirname) {
		getHomeDirectory(dirname);
	}
#ENDIF
\end{lstlisting}
\end{minipage}\hfill
\begin{minipage}{.45\textwidth}
\begin{lstlisting}[caption=Beispieleinsatz von \texttt{\#IFNDEF} nach \cite{Medeiros2018},frame=tlrb,language=C, label=example2]{example2}
int test = 1;
#IFNDEF NO_XMALLOC
	test = memory != NULL;
#ENDIF
if (test){
 // Lines of code here..
 } 
\end{lstlisting}
\end{minipage}

\noindent\begin{minipage}{.45\textwidth}
\begin{lstlisting}[caption=Beispiel eines verschachtelten Einsatzes von \texttt{\#IFDEF} nach \cite{Medeiros2018} ,frame=tlrb,language=C, label=example3, firstnumber=1]{example3}
bool time = msec > 0;
#IFDEF USE_XSMP
	time = time && xsmp_icefd != -1;
	#IFDEF FEAT_MZSCHEME
 		time = time || p_mzq > 0;
	#ENDIF
#ENDIF
if (time)
 gettime(&start_tv);
\end{lstlisting}
\end{minipage}\hfill
\begin{minipage}{.45\textwidth}
\begin{lstlisting}[caption=Beispiele von erweiterten Bedingungen nach \cite{Queiroz2015},frame=tlrb,language=C, label=example4, firstnumber=1]{example4}
#IFDEF FEATURE_A && FEATURE_B
	(...)
#ENDIF
(...)
#IFDEF !FEATURE_A && FEATURE_C
	(...)
#ENDIF
\end{lstlisting}
\end{minipage}

Die in den Listings gezeigten Beispiele zeigen jeweils nur den Featurecode in einer Methode beziehungsweise in einer Datei. Fragmente des Featurecodes erstrecken sich jedoch nicht nur möglicherweise mehrfach über eine Datei, sondern über mehrere Dateien - der Featurecode ist somit verstreut (englisch: code scattering), um eine Funktionalität des Features in der Gesamtheit der Software zu ermöglichen. Ein Defekt innerhalb eines Fragmentes des Featurecodes kann allerdings dazu führen, dass die gesamte Funktionalität des Features beeinträchtigt oder unterbunden wird, da der Fehler übergreifend wirkt (englisch: cross-cutting). Ebenfalls kann ein solcher Fehler dazu führen, dass die Funktionalität des gesamten Sourcecodes beeinträchtigt wird.

\section{Machine-Learning-Klassifikation}
\label{classification}

Das Themengebiet des Machine Learnings (ML) ist in zwei Teilgebiete unterteilt - das unüberwachte ML (englisch: unsupervised ML) und das überwachte ML (englisch: supervised ML). Die Methoden in diesen Teilgebieten verfolgen unterschiedliche Ziele. Im Rahmen des unüberwachten ML werden Prozesse durchgeführt, welche dazu dienen, die Struktur einer unbekannten Eingabemenge an Daten zu erlernen und anschließend zu repräsentieren \cite{Sammut2017}. Eine gängige Anwendung des unüberwachten ML ist das Clustering. Das überwachte ML beschreibt wiederum einen Prozess, welcher beabsichtigt, Vorhersagen über unbekannte Eingabedaten auf Basis des Trainings einer Abbildungsfunktion zu treffen \cite{Sammut2017}. Die Attribute \glqq unüberwacht\grqq{} und \glqq überwacht\grqq{} erhalten die Methoden aufgrund ihrer Art des Lernens beziehungsweise des Trainings. In der Anwendung des unüberwachten ML werden die Eingabedaten erfasst, gegebenenfalls vorverarbeitet, um dann auf deren Basis ein Modell zu erlernen, welches die Darstellung beziehungsweise Repräsentation der Eingabedaten bestimmt \cite{Alpaydin2010}. Auf der anderen Seite wird unter Anwendung des überwachten ML ein Modell auf Basis eines sogenannten \glqq gelabelten\grqq{} (beschrifteten) Datensatzes durch Merkmalsextraktion in Form von Attributen und dem Training auf der Grundlage der extrahierten Merkmale erstellt \cite{Alpaydin2010}. Der Datensatz, welcher zum Training verwendet wird, wird im gängigen Sprachgebrauch des Machine Learning Datenset (englisch: dataset) genannt. Das aus dem Training resultierende Modell wird Klassifikator (englisch: classifier) genannt. Gängige Anwendungen des überwachten ML sind Regression und Klassifikation. In dieser Arbeit kommt die Klassifikation als Anwendung des überwachten ML zum Einsatz. Der grundlegende Prozess der Machine-Learning-Klassifikation ist in \autoref{fig:ml} anhand eines Beispiels dargestellt.

\begin{figure}[ht]
    \centering
    \captionsetup{justification=centering,margin=2cm}
    \includegraphics[width=\textwidth]{images/ML}
    \caption{Allgemeiner Prozess des überwachten Machine Learnings dargestellt anhand eines Beispiels (vereinfacht)}\label{fig:ml}
\end{figure}

Die Abbildung zeigt den Prozess des überwachten Machine Learnings anhand des Beispiels des Trainings eines Klassifikators zur Erkennung beziehungsweise Vorhersage von geometrischen Formen. Der Prozess beginnt mit den \glqq gelabelten\grqq{} Eingabedaten (A). Die Werte der Label (kategorial oder numerisch) stellen dabei die zu vorhersagende Zielklasse dar. In diesem Falle bilden die Namen der geometrischen Formen die Label als kategorischen Wert. Die Rohdaten der Eingabemenge bestehen aus den geometrischen Formen selbst. Beide Datenmengen bilden das Datenset. Um nun einen Klassifikator trainieren zu können, müssen Merkmale der Eingangsdaten ausgewählt werden, anhand derer diese identifiziert werden können (B). Diese zu identifizierenden Charakteristika der Daten werden Attribute genannt. Diese Attribute können bereits vor dem Training festgelegt werden oder automatisiert extrahiert werden. Im vorliegenden Fall wurde die Metrik \glqq Anzahl der Ecken der geometrischen Formen\grqq{} als Attribut zum Training ausgewählt. Das Ergebnis ist der fertig trainierte Klassifikator, welcher das antrainierte Wissen auf neue Daten abbilden kann (C). Ein Teil des Datensets wird in der Regel verwendet, um den Klassifikator nach dessen Erstellung zu testen (D). Die in der Regel verwendeten Verhältnisse (englisch: Split-Ratio) zwischen Trainings- und Testdaten betragen $80:20$ (basierend auf dem Paretoprinzip) oder $75:25$ (zum Beispiel \cite{Queiroz2016}). Diese Testdaten werden dem Klassifikator als Eingabemenge zur Klassifikation ohne Label zur Verfügung gestellt. Die Label sollten jedoch nicht verworfen werden, da sie als Vergleichsgrundlage für die Vorhersageperformanz des Klassifikators dienen. Sie bilden die sogenannte \glqq Ground Truth\grqq{} (deutsch: Grundwahrheit). Dazu werden die vom Klassifikator vorhergesagten Label mit denen der Ground Truth verglichen. Sollte dieser Vergleich ergeben, dass die Label große Abweichungen zeigen, so kann der Klassifikator erneut mit anderen Attributen oder einer veränderten Split-Ratio trainiert werden. Erfüllt der Klassifikator die Anforderungen an die Performanz der Vorhersagen, so ist dieser bereit, Vorhersagen auf Basis neuer Eingabedaten zu treffen (E). Dazu müssen von den neuen Daten die Attribute ermittelt werden. Auf Basis dieser trifft der Klassifikator die Vorhersage und liefert als Ausgabe das Label des Wertes der Zielklasse. Im Voraus des Testens mit den Testdaten (D) werden in manchen Fällen zudem sogenannte Validierungsdaten verwendet. Dabei handelt es sich um eine eigenständige Teilmenge der Trainingsdaten, welche verwendet wird, um die Klassifikatoren nach jedem Training zu evaluieren, um die Auswahl der Attribute hinsichtlich der Performanz auf Eignung zu prüfen \cite{Sammut2017}. Die Anwendung der Testdaten erfolgt dann im Anschluss.

Der in \autoref{fig:ml} dargestellte Klassifikator stellt einen multinomiellen oder multi-class Klassifikator dar, da er zu drei oder mehr Werten der Zielklasse zuordnen kann \cite{Sammut2017}. Für viele praktische Anwendungen genügt jedoch ein binärer Klassifikator, welcher Vorhersagen zu zwei Werten der Zielklasse trifft. Dies trifft auch auf die Klassifikatoren dieser Arbeit zu.

Nachfolgend wird eine Auswahl an Klassifikationsalgorithmen vorgestellt. Diese zählen zu den meist verwendeten Algorithmen und finden auch in dieser Arbeit ihre Anwendung.

\label{algorithms}
\textbf{Decision Trees\medskip}\\
Decision Trees (deutsch: Entscheidungsbäume) zählen zu den meistverwendeten Klassifikatoren im Bereich des supervised Machine Learnings. Studien belegen, dass sie hinsichtlich der Verwendung im Kontext der Fehlererkennung die häufigste Anwendung finden \cite{Son2019}. Decision Trees sind gerichtete und verwurzelte Bäume, die als rekursive Partition der Eingabemenge des Datensets aufgebaut werden \cite{Rokach2005}. Den Ursprung des Baumes bildet die Wurzel, welche keine eingehenden Kanten besitzt - alle weiteren Knoten besitzen jedoch eine eingehende Kante \cite{Rokach2005}. Diese Knoten teilen wiederum die Eingabemenge anhand einer vorgegebenen Funktion in zwei oder mehr Unterräume der Menge auf \cite{Rokach2005}. Meist geschieht dies anhand eines Attributs, sodass die Eingabemenge anhand der Werte des einzelnen Attributs geteilt wird \cite{Rokach2005}. Die Blätter des Baumes bilden die Zielklassen ab. Eine Klassifizierung kann folglich durchgeführt werden, indem man von der Wurzel bis zu einem Blatt den Kanten anhand der entsprechenden Werte der Eingangsmenge folgt. Es existieren verschiedene Algorithmen zur Erstellung von Decision Trees. Bekannte Stellvertreter dieser sind ID3, C4.5 (J48) und CART \cite{Rokach2005}. Der grundlegende Aufbau eines Decision Trees ist in \autoref{fig:dt} dargestellt.

\begin{figure}[ht]
    \centering
    \includegraphics[width=0.5\textwidth]{images/DT}
    \caption{Grundsätzlicher Aufbau eines Decision Trees\label{fig:dt}}
\end{figure}

Eine Besonderheit von Decision Trees stellen sogenannte Random Forests dar. Diese beschreiben eine Menge von Klassifikatoren, bei der mehrere einzelne Decision Trees gleichzeitig erzeugt werden und deren Ergebnisse anschließend aggregiert werden \cite{Alam2013}. Dazu erhält jeder Decision Tree eine Teilmenge der Eingabemenge des Datensets \cite{Alam2013}. Random Forests eigenen sich besonders zur Anwendung, wenn viele Attribute im Datenset vorhanden sind \cite{Alam2013}.

\textbf{k-Nearest-Neighbors\medskip}\\
Ein k-Nearest-Neighbor-Klassifikator (deutsch: k-nächste-Nachbarn) basiert auf zwei Konzepten \cite{Zhang2016}. Das erste Konzept basiert auf der Abstandsmessung zwischen den Werten der zu klassifizierenden Datenmenge und den Werten der Attribute des Datensets \cite{Zhang2016}. Die Abstandmessung erfolgt in der Regel durch die Berechnung der Euklidischen Distanz $D(p,q)$:

\[D(p,q) = \sqrt{\sum_1^n(p_{n}-q_{n})^{2}}\] 

Die Anzahl der Attribute wird durch den Parameter $n$ wiedergegeben, $p$ und $q$ repräsentieren jeweils die Werte der zu klassifizierenden Datenmenge und die Werte der Attribute des Datensets. Das zweite Konzept bildet der Parameter $k$, der angibt, wie viele nächste Nachbarn zum Vergleich der zuvor berechneten Abstände in Betracht gezogen werden \cite{Zhang2016}. Bei einem $k > 1$ wird diejenige Zielklasse gewählt, deren Auftreten innerhalb der nächsten Nachbarn überwiegt.

\textbf{Künstliche neuronale Netze\medskip}\\
Künstliche neuronale Netze (KNN, englisch: Artificial Neural Networks) verwenden nicht-lineare Funktionen zur schrittweisen Erzeugung von Beziehungen zwischen der Eingabemenge und den Zielklassen durch einen Lernprozess \cite{Linder2004}. Sie sind angelehnt an die Funktionsweise von biologischen Nervensystemen und bestehen aus einer Vielzahl von verbundenen Berechnungsknoten, den Neuronen \cite{OShea2015}. Der grundsätzliche Aufbau eines künstlichen neuronalen Netzes kann in \autoref{fig:ann} eingesehen werden. Der Lernprozess besteht aus zwei Phasen - einer Trainingphase und einer Recall-Phase \cite{Linder2004}. In der Trainingsphase werden die Eingabedaten, meist als multidimensionaler Vektor, in den Input-Layer geladen und anschließend an die Hidden-Layer verteilt \cite{OShea2015}. In den Hidden-Layers werden dann Entscheidungen anhand der Beziehungen zwischen den Eingabedaten und Zielklassen sowie die den Verbindungen zuvor zugewiesenen Gewichtsfaktoren getroffen \cite{Linder2004,OShea2015}. Im Rahmen der Recall-Phase wird die Vorhersage basierend auf der zu klassifizierenden Datenmenge anhand der zuvor getroffenen Entscheidungen der Hidden-Layers getroffen und an die jeweiligen Output-Layer, welche die Werte der Zielklasse repräsentieren, weitergeleitet \cite{Linder2004}. 

\begin{figure}[ht]
    \centering
    \includegraphics[width=0.5\textwidth]{images/ANN}
    \caption{Grundsätzlicher Aufbau eines KNN mit drei Input-Layer-Neuronen, fünf Hidden-Layer-Neuronen und zwei Output-Layer-Neuronen\label{fig:ann}}
\end{figure}

\textbf{Logistische Regression\medskip}\\
Logistische-Regressions-Klassifikatoren (englisch: Logistic Regression) basieren auf dem mathematischen Konzept des Logits, welcher den natürlichen Logarithmus eines Chancenverhältnisses beschreibt \cite{Peng2002}. Seine Formel lautet:

\[logit(Y) = ln(\frac{\pi}{1-\pi})\]

$Y$ beschreibt dabei die zu klassifizierende Datenmenge, wohingegen $\pi$ die Verhältnisse der Wahrscheinlichkeiten der Werte der Attribute der Eingabemenge bezeichnet. Am besten geeignet ist dieser Klassifikator für eine Kombination aus kategorialen oder numerischen Eingabedaten und kategorischen Zielklassen \cite{Peng2002}.

\textbf{Na\"{\i}ve Bayes\medskip}\\
Na\"{\i}ve-Bayes-Klassifikatoren zählen zu den linearen Klassifikatoren und basieren auf dem Satz von Bayes. Die Bezeichnung \glqq naiv\grqq{} erhält der Klassifikator durch die Annahme, dass die Attribute der Eingabemenge unabhängig voneinander sind \cite{Raschka2014}. Diese Annahme wird zwar in der realen Verwendung des Klassifikators häufig verletzt, dennoch erzielt er in der Regel eine hohe Performanz \cite{Raschka2014}. Der Klassifikator gilt als effizient, robust, schnell und einfach implementierbar \cite{Raschka2014}. Die zur Durchführung einer Klassifikation mittels Na\"{\i}ve Bayes benötigte Formel nach Thomas Bayes ist in \autoref{fig:nb} samt Erläuterung der einzelnen Faktoren aufgeführt.

\begin{figure}[ht]
    \centering
    \includegraphics[width=0.6\textwidth]{images/NB}
    \caption{Satz von Bayes als Grundlage des Na\"{\i}ve-Bayes-Klassifikators\label{fig:nb}}
\end{figure}

Es existiert zudem eine Mehrzahl an Varianten des Na\"{\i}ve-Bayes-Klassifikators, die verschiedene Annahmen über die Verteilung der Attribute der Eingabemenge machen. Beispiele dafür sind der Gaußsche-Na\"{\i}ve-Bayes (normalverteilte Attribute), der multinomiale Na\"{\i}ve-Bayes (multinomiale Verteilung der Attribute) sowie der Bernoulli-Na\"{\i}ve-Bayes (unabhängige binäre Attribute).

\textbf{Stochastic Gradient Descent\medskip}\\
Ein Stochastic-Gradient-Descent-Klassifikator basiert auf dem Gradientenverfahren (englisch: Gradient Descent), welches das Ziel hat, mittels einer Kostenfunktion $f$ zu einem gegebenen $x$ das minimale $y$ zu finden \cite{Srinivasan2019}. Im Falle der Klassifikation mittels Machine Learning bedeutet dies, dass die Funktion $f$ auf Basis der Trainingsdaten erzeugt wird, die wiederum die Attribute der Daten auf die Werte der Zielklasse überträgt \cite{Diab2019}. Eine festgelegte Kostenfunktion, versucht dann auf Basis der Trainingsdaten ($x$) die minimale Fehlerquote für die Vorhersagen ($y$) anhand von verschiedenen Koeffizientenwerten zu ermitteln \cite{Diab2019}.

\textbf{Support Vector Machines\medskip}\\
Support Vector Machines verfolgen das Ziel, eine sogenannte \glqq Hyperplane\grqq{} in einem $n$-dimen-sionalen Raum ($n$ = Anzahl der Attribute der Eingabemenge) zu finden, welche die Datenpunkte der Eingabemenge eindeutig klassifizieren kann \cite{Gandhi2018}. Die Hyperplane beschreibt eine Trennlinie beziehungsweise Trennfläche, mit deren Hilfe die Daten der zu klassifizierenden Menge den Zielklassen zugeordnet werden können \cite{Luber2019}. Dabei gilt es, dass die Trennflächen, welche die Eingangsmenge anhand der Attribute in verschiedene Trennungsebenen unterteilen, einen möglichst großen Abstand ohne Datenpunkte voneinander haben \cite{Luber2019}. Dies funktioniert sowohl für linear als auch nicht-lineare trennbare Mengen.

\section{Fehlervorhersage mittels Machine Learning}

Der Hintergrund der Fehlervorhersage mittels Machine Learning basiert auf dem zuvor vorgestellten Konzept des überwachten Machine Learnings. Als Grundlage dienen dabei meist Daten, die aus Software-Repositories entnommen beziehungsweise extrahiert werden. Viele Studien und wissenschaftliche Arbeiten setzen jedoch auch auf vorgefertigte Datensets, wie zum Beispiel von der NASA oder von Eclipse \cite{Son2019}. Die zugehörigen Label der Datensets lauten in der Regel \glqq fehlerfrei\grqq{} und \glqq defekt\grqq{} und können auf verschiedene Weisen ermittelt werden. Eine gängige Methode ist die Identifizierung von korrektiven und fehlereinführenden Commits als Entscheidungsgrundlage für das Label. Die üblichen Vorgehensweisen setzen auf die Einbindung von Bugtracking-Systemen oder die Analyse von Commit-Nachrichten zur Identifizierung der korrektiven Commits \cite{Queiroz2016,Zimmermann2007}. Fehlereinführende Commits können anschließend unter Verwendung von Git-Kommandos oder durch die Anwendung des sogenannten SZZ-Algorithmus ermittelt werden. Dieser Algorithmus wird in dieser Arbeit verwendet und in \hyperref[szz-def]{Abschnitt 3.2} erläutert. Auf Basis dieser Daten werden die Attribute bestimmt. Dabei handelt es sich in der Regel um Metriken, die entweder die Charakteristika des Sourcecodes (Codemetriken) oder Aktivitäten und Prozesse im Bezug auf Software-Repositories (Prozessmetriken) beschreiben \cite{Son2019,Rahman2013}. Mithilfe dieser Attribute werden die Klassifikatoren trainiert. Eine Studie, welche 156 wissenschaftliche Arbeiten zum Thema der Fehlervorhersage mittels Machine Learning analysierte, ergab, dass besonders Entscheidungsbaum-basierte, Bayessche Verfahren, Regression und künstliche neuronale Netze als Klassifikationsalgorithmen zur Anwendung kommen \cite{Son2019}. Diese Algorithmen wurden unter anderem auch in dieser Arbeit verwendet. Erläuterungen können in \hyperref[algorithms]{Abschnitt 4.1} gefunden werden. Die fertig trainierten Klassifikatoren können dann auf Basis neuer Daten Vorhersagen zum Zustand einer Software treffen. Die Vorhersagen beruhen in der Regel auf defekten Dateien im Kontext von Commits oder Releases.

Als Beispiel für zwei konkrete Anwendungen von Machine Learning gestützter Fehlervorhersage werden im Folgenden zwei wissenschaftliche Arbeiten vorgestellt. Die erste Arbeit \glqq Comparative Analysis of the Efficiency of Change Metrics and Static Code Attributes for Defect Prediction\grqq{} von Moser et al. \cite{Moser2008} stellt eine Methodik zur dateibasierten Fehlervorhersage vor. Die zweite Arbeit \glqq Towards Predicting Feature Defects in Software Product Lines\grqq{} von Queiroz et al. \cite{Queiroz2016} knüpft an den zuvor vorgestellten Ansatz der Software-Features an. Beide Literaturquellen werden im weiteren Verlauf dieser Arbeit eine Rolle spielen, die im jeweiligen Abschnitt erläutert werden.

\subsection*{Dateibasierte Fehlervorhersage}
\label{moser}

Das Beispiel zur dateibasierten Fehlervorhersage stammt aus einer wissenschaftlichen Arbeit von Moser et al. \cite{Moser2008}. Sie widmet sich einer vergleichenden Analyse von zwei verschiedenen Mengen von Metriken zur dateibasierten Fehlervorhersage mittels Methoden des Machine Learnings. Die Zuordnung erfolgt in \glqq defekt\grqq{} und \glqq defektfrei\grqq. Als Datenbasis dient ein vorgefertigtes Datenset von Eclipse. Auf Basis dieses Datensets wurden \glqq produktbasierte\grqq{} Metriken (Codemetriken) und Prozessmetriken berechnet. Zur Anwendung kamen die Klassifikationsalgorithmen logistische Regression, Na\"{\i}ve Bayes und Entscheidungsbäume. Die Prozessmetriken stellen eine Besonderheit dieser Arbeit dar, da sie in dieser Arbeit zum ersten Mal näher betrachtet und auf ihre Eignung als Attribute zum Training der Klassifikatoren erörtert wurden. Die Metriken berechneten unter anderem die Anzahl der Änderungen an einer Datei, die Anzahl der Autoren einer Datei, die Anzahl der hinzugefügten oder entfernten Zeilen einer Datei oder das Alter einer Datei. Das Resultat der Arbeit lautet, dass Prozessmetriken effektiver zur Fehlervorhersage genutzt werden können als Codemetriken. Im Folgenden werden die 17 Prozessmetriken samt ihrer Beschreibungen und Abkürzungen (für diese Arbeit) vorgestellt.

\label{moser-metrics}
\begin{multicols}{2}
\begin{itemize}
\setlength{\itemsep}{-2pt}
\item REVISIONS (REVI)\\Anzahl der Revisionen (Bearbeitungen) der Datei.
\item REFACTORINGS (REFA)\\Anzahl der Fälle, in denen die Datei in einem Refactoring involviert war. Basierend auf Analyse der Commit-Nachricht auf das Vorhandensein des Begriffs "refactor".
\item BUGFIXES (BUGF)\\Anzahl der Fälle, in denen die Datei in einer Fehlerbehebung involviert war.
\item AUTHORS (AUTH)\\Anzahl der verschiedenen Autoren, die die Datei in das Repository eingecheckt haben.
\item LOC\_ADDED (ADDL)\\Summe der zur Datei hinzugefügten Codezeilen über alle Revisionen.
\item MAX\_LOC\_ADDED (ADDM)\\maximale Anzahl von Codezeilen, die für alle Revisionen hinzugefügt wurden.
\item AVE\_LOC\_ADDED (ADDA)\\durchschnittlich hinzugefügte Codezeilen pro Revision.
\item LOC\_DELETED (REML)\\Summe der von der Datei entfernten Codezeilen über alle Revisionen.
\item MAX\_LOC\_DELETED (REMM)\\maximale Anzahl von Codezeilen, die für alle Revisionen entfernt wurden.
\item AVE\_LOC\_DELETED (REMA)\\durchschnittlich entfernte Codezeilen pro Revision.
\item CODECHURN (CCHN)\\Summe von (hinzugefügte Codezeilen - entfernte Codezeilen) über alle Revisionen.
\item MAX\_CODECHURN (CCHM)\\maximaler CODECHURN für alle Revisionen.
\item AVE\_CODECHURN (CCHA)\\durchschnittlicher CODECHURN pro Revision.
\item MAX\_CHANGESET (MAXC)\\maximale Anzahl von Dateien, die gemeinsam committed wurden.
\item AVE\_CHANGESET (AVGC)\\durchschnittliche Anzahl von Dateien, die gemeinsam committed wurden.
\item AGE (AAGE)\\Alter der Datei in Wochen (rückwärts zählend bis zu einem bestimmten Release).
\item WEIGHTED\_AGE (WAGE)\\$\text{Weighted Age} = \frac{\sum_{i=1}^N Age(i)*LOC\_ADDED(i)}{\sum_{i=1}^N LOC\_ADDED(i)}$
\end{itemize}
\end{multicols}

Im Rahmen der Evaluation (\hyperref[classic-eval]{Abschnitt 5.3}) dienen diese dateibasierten Metriken als Grundlage für den Vergleich, ob die Metriken des für diese Arbeit erstellten featurebasierten Datensets einen Einfluss auf die Ergebnisse der Fehlervorhersage auf Dateiebene hervorrufen. 

\subsection*{Featurebasierte Fehlervorhersage}

Das Beispiel zur featurebasierten Fehlervorhersage stammt aus einer wissenschaftlichen Arbeit von Queiroz et al. \cite{Queiroz2016}. Bei dieser Fallstudie handelt es sich um die erste und bisher einzige Arbeit über Fehlervorhersage mit Bezug zu Software-Features. Sie stellt somit für diese Masterarbeit eine bedeutende literarische Grundlage dar. Der Ablauf des von Queiroz et al. angewandten Prozesses zur Erstellung eines featurebasierten Datensets und dessen Anwendung zum Training von Klassifikatoren orientiert sich am zuvor vorgestellten allgemeinen Prozess des überwachten Machine Learnings.

Die Erläuterung des Beispiels erfolgt anhand von zwei Abbildungen, welche den in der Arbeit von Queiroz et al. vorgestellten Prozess in zwei Teilen visualisieren. Der erste Teil ist in \autoref{fig:ml1} dargestellt.

\begin{figure}[ht]
    \centering
    \captionsetup{justification=centering}
    \includegraphics[width=\textwidth]{images/ML1}
    \caption{Teil 1: Featurebasierter Prozess des überwachten Machine Learnings nach \cite{Queiroz2016}}\label{fig:ml1}
\end{figure}

\subsubsection*{Datenset}

Die Datenbasis des Datensets bilden historische Commits des UNIX-Toolkits BusyBox\footnote{\url{https://busybox.net/}}, dessen Quellcode frei verfügbar in einem Git-Repository\footnote{\url{https://git.busybox.net/busybox/}} eingesehen und von dort geklont werden kann. Diese Commits wurden wiederum ihren entsprechenden Releases zugeordnet, welche auf der vergebenen Tag-Struktur des Repositories beruhen. Ferner wurden aus den Diffs der Commits die dort bearbeiteten Features extrahiert und anschließend zusammen mit den Release-Informationen in einer MySQL-Datenbank gespeichert. Zusätzlich enthält jeder Datenbankeintrag aggregierte Werte von fünf auf das Feature und den Release bezogenen Prozessmetriken (Erläuterung folgt) sowie das binäre Label, ob ein Feature in einem Release fehlerhaft oder fehlerfrei war. Ein Feature gilt in einem Release als fehlerhaft, sofern in einem Commit des darauffolgenden Releases ein fehlerbehebender Commit bezüglich des Features festgestellt werden konnte. Dies geschieht über die Analyse der Commit-Nachrichten. Sofern eine Commit-Nachricht die Begriffe \glqq bug\grqq{} (Fehler), \glqq error\grqq{} (schwerwiegender Fehler), \glqq fail\grqq{} (fehlschlagen) oder \glqq fix\grqq{} (beheben) enthält, werten die Autoren des Papers den Commit als fehlerbehebend. Alternative Methoden zur Durchführung dieser Analyse bestehen aus der Einbindung von Daten aus Bug-Tracking-Systemen, die häufig an Software-Repositories angebunden sind, sowie aus der Anwendung des sogenannten SZZ-Algorithmus, welcher in dieser Arbeit verwendet wurde und in \hyperref[szz-def]{Abschnitt 3.2} erläutert wird \cite{Sliwerski2005,Zimmermann2007}. Wie im Rahmen des überwachten Machine Learning üblich, wird das Datenset in Trainings- und Testdaten in einem Verhältnis von $75:25$ geteilt. 

\subsubsection*{Metriken und Klassifikation}

Die Trainingsdaten werden dann den Klassifikatoren zum Training zur Verfügung gestellt. Als Attribute dienen fünf Prozessmetriken mit spezifischer Betrachtung von Software-Features. Einen Überblick über die Beschreibungen dieser gibt \autoref{tab:metrics-rodrigo}. Diese Metriken werden auch für diese Arbeit im Rahmen der Erstellung des featurebasierten Datensets übernommen. Als Klassifikationsalgorithmen wurden Na\"{\i}ve Bayes, Random Forest und J48-Entscheidungsbäume gewählt.

\begin{table}[ht]
\centering
\caption{Übersicht der von \cite{Queiroz2016} verwendeten Prozessmetriken}
\label{tab:metrics-rodrigo}
\begin{tabular}{|c|l|} 
\hline
\textbf{Metrik}  & \textbf{Beschreibung}  \\ 
\hline
COMM & \begin{tabular}[c]{@{}l@{}}Anzahl der Commits, die in einem Release dem betreffenden \\ Feature gewidmet sind. \end{tabular} \\ 
\hline
ADEV & \begin{tabular}[c]{@{}l@{}}Anzahl der Entwickler, die das betreffende Feature\\in einem Release bearbeitet haben. \end{tabular} \\ 
\hline
DDEV & \begin{tabular}[c]{@{}l@{}}kummulierte Anzahl der Entwickler, die das betreffende Feature\\in einem Release bearbeitet haben. \end{tabular} \\ 
\hline
EXP & \begin{tabular}[c]{@{}l@{}}geometrisches Mittel der \glqq Erfahrung*\grqq{} aller Entwickler, die am \\ betreffenden Feature in einem Release gearbeitet haben. \end{tabular} \\ 
\hline
OEXP & \begin{tabular}[c]{@{}l@{}}\glqq Erfahrung*\grqq{} des Entwicklers, der am meisten zum betreffenden \\ Feature in einem Release beigetragen hat. \end{tabular} \\ 
\hline
\multicolumn{2}{|c|}{\begin{tabular}[c]{@{}c@{}}*Erfahrung ist definiert als Summe der geänderten, gelöschten\\oder hinzugefügten Zeilen im zugehörigen Release. \end{tabular}} \\
\hline
\end{tabular}
\end{table}

\begin{figure}[ht]
    \centering
    \includegraphics[width=0.8\textwidth]{images/ML2}
    \caption{Teil 2: Featurebasierter Prozess des überwachten Machine Learnings nach \cite{Queiroz2016}}\label{fig:ml2}
\end{figure}

\subsubsection*{Test der Klassifikatoren}

Wie in \autoref{fig:ml2} dargestellt ist, wird für jeden Klassifikationsalgorithmus ein Klassifikator erstellt, welcher anschließend getestet und evaluiert wird. Dazu werden die jeweiligen Klassifikatoren auf das Testdatenset angewendet, ohne jedoch die Werte der Zielklassen mit anzugeben. Diese werden im Anschluss an den Klassifikationsvorgang mit den vorhergesagten Werten auf Übereinstimmung verglichen. Anhand dieses Vergleiches können die Genauigkeit sowie weitere Metriken zur Bewertung der Leistung der Klassifikatoren gemessen werden. Eine Übersicht von Evaluationsmetriken kann in \hyperref[eval-metrics]{Abschnitt 5.2.1} gefunden werden.

Die so erstellten Klassifikatoren können dann zur Vorhersage von neuen und unbekannten Daten genutzt werden, um defekte Features eines zukünftigen Releases zu identifizieren. Dazu müssen die fünf zuvor genannten Prozessmetriken dieser Daten berechnet werden. 

\cleardoublepage

    
    % !TEX root = ../thesis.tex

\chapter{Erstellung eines featurebasierten Datensets}

\paragraph{Ausblick:}
Dieses Kapitel widmet sich der schrittweisen Erläuterung des Prozesses zur Erstellung des featurebasierten Datensets, welches zur Anlernung der Machine-Learning-Klassifikatoren dient. Dazu wird zunächst die Datenauswahl näher beleuchtet. Darauf folgt eine Darlegung der Konstruktion des Datensets sowie der Auswahl und Berechnung der Metriken, welche als Attribute (Features) im Rahmen der Anlernung der Klassifikatoren dienen. Eine Gliederung der Kapitel kann Abbildung XX entnommen werden.

\begin{figure}[H]
    \centering
    \includegraphics[width=\textwidth]{images/Kap3}
    \caption{Übersicht zur Gliederung des dritten Kapitels\label{fig:kap3}}
\end{figure}

\hrule

\section{Datenauswahl}

Wie im vorangegangenen Kapitel bereits erwähnt wurde, bildet das Datenset die Grundlage für die Anlernung der Machine-Learning-Klassifikatoren und wird eigens für diese Arbeit auf Basis von Commits von 13 featurebasierten Software-Projekten erstellt. Die Auswahl der Software-Projekte erfolge anhand von vorheriger Verwendung in wissenschaftlicher Literatur \cite{Hunsen2015,Liebig2010,Queiroz2016}. Die für diese Arbeit verwendeten Software-Projekte sind samt ihres Einsatzzweckes und ihrer Datenquellen in Tabelle XX aufgeführt.

\begin{table}
\centering
\caption{Übersicht der verwendeten Software-Projekten\protect\footnotemark{}}
\label{tab:tools}
\resizebox{\linewidth}{!}{%
\begin{tabular}{|l|c|c|l|c|c|} 
\cline{2-3}\cline{5-6}
\multicolumn{1}{l|}{} & \textbf{Zweck}       & \textbf{Datenquelle}  &                      & \textbf{Zweck}       & \textbf{Datenquelle}     \\ 
\hline
\textbf{Blender}      & 3D-Modellierungstool & GitHub-Mirror         & \textbf{libxml2}     & XML-Parser           & GitLab-Repository        \\ 
\hline
\textbf{Busybox}      & UNIX-Toolkit         & Git-Repository        & \textbf{lighttpd}    & Webserver            & Git-Repository           \\ 
\hline
\textbf{Emacs}        & Texteditor           & GitHub-Mirror         & \textbf{MPSolve}     & Polynomlöser         & GitHub-Repository        \\ 
\hline
\textbf{GIMP}         & Bildbearbeitung      & GitLab-Repository     & \textbf{Parrot}      & virtuelle Maschine   & GitHub-Repository        \\ 
\hline
\textbf{Gnumeric}     & Tabellenkalkulation  & GitLab-Repository     & \textbf{Vim}         & Texteditor           & GitHub-Repository        \\ 
\hline
\textbf{gnuplot}      & Plotting-Tool        & GitHub-Mirror         & \textbf{xfig}        & Grafikeditor         & Sourceforge-Repository~  \\ 
\hline
\textbf{Irssi}        & IRC-Client           & GitHub-Repository     & \multicolumn{1}{l}{} & \multicolumn{1}{l}{} & \multicolumn{1}{l}{}     \\
\cline{1-3}
\end{tabular}
}
\end{table}
\footnotetext{Links zu den Websites der Softwareprojekte und deren Repositories können im \hyperref[appendix1]{Anhang} eingesehen werden.}

Zum Erhalt der Commit-Daten der Software-Projekte wurde die Python-Library PyDriller\footnote{\href{https://github.com/ishepard/pydriller}{https://github.com/ishepard/pydriller}} verwendet \cite{Spadini2018}. Diese ermöglicht eine einfache Datenextraktion von Git-Repositories zum Erhalt von Commits, Commit-Nachrichten, Entwicklern, Diffs und mehr. Ein beispielhafter Sourcecode-Ausschnitt zur Konsolenausgabe von Metadaten eines Commits (Autor, Name der veränderten Dateien, Typ der Veränderung und jeweilige zyklomatische Komplexität der Dateien) ist in Listing 3.1 aufgeführt. 

\begin{lstlisting}[language=Python, caption=Beispielhafter PyDriller-Code zur Ausgabe von Metadaten von Commits, frame=single]
for commit in RepositoryMining("link_to_repo").traverse_commits(): 
	for m in commit.modifications: 
		print( 
		     "Author {}".format(commit.author.name), 
		     " modified {}".format(m.filename), 
		     " with a change type of {}".format(m.change_type.name), 
		     " and the complexity is {}".format(m.complexity) 
		)
\end{lstlisting}

Als Input der Python-Scripte zum Erhalt der Commit-Daten dienten jeweils die URLs zu den Git-Repositories der Software-Projekte. Weiterhin wurden die Daten in Commits je Release aufgeteilt. Durchgeführt wurde dies durch die Angabe von Release-Tags, basierend auf der Tag-Struktur von Git-Repositories, im PyDriller-Code. Für jede veränderte Datei innerhalb eines Commits und eines Releases wurden die folgenden Metadaten mit Hilfe von PyDriller abgerufen:

\begin{multicols}{2}
\begin{itemize}
\item Commit-Hash (eindeutiger Bezeichner des zugehörigen Commits)
\item Autor des zugehörigen Commits
\item zugehörige Commit-Nachricht
\item Name der veränderten Datei
\item Lines-of-Code der veränderten Datei
\item zyklomatische Komplexität der veränderten Datei
\item Anzahl der hinzugefügten Zeilen zur Datei
\item Anzahl der entfernten Zeilen von der Datei
\item Art der Änderung (ADD, REM, MOD)\footnote{Diese Information fand in der weiteren Erstellung des Datensets keine Verwendung.}
\item Diff der Veränderung
\end{itemize}
\end{multicols}

Die auf diese Weise erhaltenen Daten wurden nach dem Abruf in einer MySQL-Datenbank gespeichert. Für jedes Software-Projekt wurde eine eigene Tabelle erstellt, in welcher neben der oben stehenden Metadaten zudem der Name des betreffenden Software-Projekts und die den Commits zugehörigen Release-Nummern gespeichert wurden. Jede veränderte Datei eines Commits erhält eine Zeile der Datenbank-Tabellen. In Tabelle XX kann eingesehen werden wie viele Releases je Software-Projekt zum Abruf einbezogen wurden und wie viele Commits daraus resultieren.

\begin{table}
\centering
\caption{Übersicht der Anzahl der Releases und Commits je Software-Projekt}
\label{tab:tools}
\resizebox{\linewidth}{!}{%
\begin{tabular}{|l|c|c|l|c|c|} 
\cline{2-3}\cline{5-6}
\multicolumn{1}{l|}{} & \textbf{\#Releases}  & \textbf{\#Commits}  &                      & \multicolumn{1}{l|}{\textbf{\#Releases}~} & \textbf{\#Commits}~   \\ 
\hline
\textbf{Blender}      & 11                   & 19119               & \textbf{libxml2}     & 10                                        & 732                   \\ 
\hline
\textbf{Busybox}      & 14                   & 4984                & \textbf{lighttpd}    & 6                                         & 2597                  \\ 
\hline
\textbf{Emacs}        & 7                    & 12805               & \textbf{MPSolve}     & 8                                         & 668                   \\ 
\hline
\textbf{GIMP}         & 14                   & 7240                & \textbf{Parrot}      & 7                                         & 16245                 \\ 
\hline
\textbf{Gnumeric}     & 8                    & 6025                & \textbf{Vim}         & 7                                         & 9849                  \\ 
\hline
\textbf{gnuplot}      & 5                    & 6619                & \textbf{xfig}        & 7                                         & 18                    \\ 
\hline
\textbf{Irssi}        & 7                    & 253                 & \multicolumn{1}{l}{} & \multicolumn{1}{l}{}                      & \multicolumn{1}{l}{}  \\
\cline{1-3}
\end{tabular}
}
\end{table}

Diese \glqq Rohdaten\grqq{} dienen zur weiteren Verarbeitung hinsichtlich der Erstellung des Datensets und der anschließenden Berechnung der Metriken. Eine Erläuterung der weiteren Verarbeitung der Daten folgt im kommenden Abschnitt.

\begin{table}[]
\caption[Caption for LOF]{Übersicht der zur Erstellung des Datensets verwendeten Software-Projekten mit zugehörigen Werten}
\label{tab:tools}
\resizebox{\textwidth}{!}{%
\begin{tabular}{lccccccc}
                                       & \textbf{Zweck}       & \textbf{Datenquelle}   & \textbf{\#Releases} & \textbf{\#Commits} & \textbf{\#Korrektiv} & \textbf{\#Fehlereinführend} & \textbf{\#Features} \\ \hline
\multicolumn{1}{l|}{\textbf{Blender}}  & 3D-Modellierungstool & GitHub-Mirror          & 11                  & 19119              & 8333                 & 1418                        & 1400         \\
\multicolumn{1}{l|}{\textbf{Busybox}}  & UNIX-Toolkit         & Git-Repository         & 14                  & 4984               & 1408                 & 142                         & 628                 \\
\multicolumn{1}{l|}{\textbf{Emacs}}    & Texteditor           & GitHub-Mirror          & 7                   & 12805              & 6959                 & 685                         & 718                 \\
\multicolumn{1}{l|}{\textbf{GIMP}}     & Bildbearbeitung      & GitLab-Repository      & 14                  & 7240               & 1703                 & 272                         & 204                \\
\multicolumn{1}{l|}{\textbf{Gnumeric}} & Tabellenkalkulation  & GitLab-Repository      & 8                   & 6025               & 1591                 & 136                         & 637                 \\
\multicolumn{1}{l|}{\textbf{gnuplot}}  & Plotting-Tool        & GitHub-Mirror          & 5                   & 6619               & 880                  & 1323                        & 558                 \\
\multicolumn{1}{l|}{\textbf{Irssi}}    & IRC-Client           & GitHub-Repository      & 7                   & 253                & 77                   & 1                           & 9                  \\
\multicolumn{1}{l|}{\textbf{libxml2}}  & XML-Parser           & GitLab-Repository      & 10                  & 732                & 409                  & 37                          & 200                 \\
\multicolumn{1}{l|}{\textbf{lighttpd}} & Webserver            & Git-Repository         & 6                   & 2597               & 1202                 & 555                         & 230                 \\
\multicolumn{1}{l|}{\textbf{MPSolve}}  & Polynomlöser         & GitHub-Repository      & 8                   & 668                & 158                  & 69                          & 54                 \\
\multicolumn{1}{l|}{\textbf{Parrot}}   & Virtuelle Maschine   & GitHub-Repository      & 7                   & 16245              & 3437                 & 824                         & 397                 \\
\multicolumn{1}{l|}{\textbf{Vim}}      & Texteditor           & GitHub-Repository      & 7                   & 9849               & 1033                 & 2571                        & 1158                \\
\multicolumn{1}{l|}{\textbf{xfig}}     & Grafikeditor         & Sourceforge-Repository & 7                   & 18                 & 0                    & 0                           & 137                
\end{tabular}%
}
\end{table}

\section{Konstruktion des Datensets}

Die Konstruktion des Datensets gliedert sich in mehrere Phasen der Datenverarbeitung und -optimierung. Die erste Phase besteht aus der Extraktion der involvierten Features einer veränderten Datei. Dazu wurden mithilfe eines Python-Scripts die sogenannten Präprozessor-Direktiven \texttt{\#IFDEF} und \texttt{\#IFNDEF} in den Diffs der veränderten Dateien identifiziert und anschließend die den Direktiven folgende Zeichenfolge bis zum Ende der Codezeile als Feature gespeichert. Die Identifizierung erfolgte mittels regulären Ausdrücken. Gespeichert werden die pro Datei identifizierten Features in einer zusätzlichen Spalte in den jeweiligen MySQL-Tabellen der Software-Projekte. Konnte kein Feature identifiziert werden, wird entsprechend \texttt{none} gepseichert.

Dieser Weg der Identifizierung birgt einige Hindernisse. Diese können, neben dem Normalfall, in Abbildung XX gesehen werden. In einigen C-Programmierparadigmen ist es üblich, Header-Dateien mittels Präprozessor-Direktiven in Sourcecode einzubinden, sodass sie wie Features scheinen (siehe erster unerwünschter Fall in Abbildung XX). Diese "Header-Features", wie sie im weiteren Verlauf genannt werden, sollten jedoch ignoriert werden, da sie im Sourcecode keine Variabilität erzeugen. In der Regel sind diese Header-Features identifizierbar durch ihre Namensgebung in Form eines angehängten \texttt{\_h\_} an den Featurenamen, wie beispielsweise \texttt{featurename\_h\_}. Dieser angehängte Teil erlaubt es, die Header-Features mittels regulärer Ausdrücke zu erkennen und auszufiltern. 

Ebenfalls besteht die Möglichkeit, dass \glqq falsche\grqq{} Features identifiziert werden können. Beispiele dafür können von \texttt{\#IFDEFs} stammen, welche in Kommentaren verwendet wurden (siehe zweiter unerwünschter Fall in Abbildung XX). Solche falschen Features wurden in einer manuellen Sichtung der identifizierten Features entfernt und durch \texttt{none} ersetzt.

\begin{figure}[]
    \centering
    \includegraphics[width=\textwidth]{images/Features}
    \caption{Normalfall und unerwünschte Fälle bei der Identifizierung von Features\label{fig:feat}}
\end{figure}

Die nächste Phase der Verarbeitung besteht aus der Identifizierung von korrektiven Commits. Eine dafür gängige Methode, die auch in dieser Arbeit Anwendung fand, besteht aus der Analyse der Commit-Nachrichten auf das Vorhandensein von bestimmten Schlagworten (\textbf{HIER ANGABE ZU LITERATUR VON RODRIGO}). Bei den Schlagworten handelt es sich um \glqq bug\grqq, \glqq error\grqq, \glqq fail\grqq{} und \glqq fix\grqq. Durchgeführt wurde die Analyse mittels Python-Skripte unter Zuhilfenahme von einfachen Formen des Text Minings. Die Ergebnisse wurden in einer weiteren boole'schen Spalte der MySQL-Tabellen (true = korrektiv, false = nicht korrektiv) gespeichert.

Der Suche nach korrektiven Commits folgt eine Analyse auf fehlereinführende Commits. Dazu wurde eine PyDriller-Implementierung des SZZ-Algorithmus nach Sliwerski, Zimmermann und Zeller verwendet \cite{Sliwerski2005}. Dieser ursprünglich für CVS-Versionskontrollsysteme entwickelte Algorithmus erlaubt es, in zwei Phasen fehlereinführende Commits in lokal gespeicherten Software-Repositories zu finden \cite{Borg2019}. Die erste Phase besteht dabei aus der Identifizierung der korrektiven Commits. Dies kann entweder anhand der zuvor beschriebenen Analyse der Commit-Nachrichten geschehen oder durch die Analyse von Bug-Tracking-Systemen \cite{Borg2019}. Die zweite Phase umfasst die Identifikation der fehlereinführenden Commits auf Basis der zuvor erkannten korrektiven Commits. Diese Phase ist in mehrere Schritte unterteilt und wird in Abbildung XX dargestellt. Die Erläuterungen der mit Buchstaben versehenen Schritte erfolgt im Anschluss. Die PyDriller-Implementierung des Algorithmus folgt dem gezeigten Ablauf.

\begin{figure}[]
    \centering
    \includegraphics[width=\textwidth]{images/SZZ}
    \caption{Ablauf der zweiten Phase des SZZ-Algorithmus (übersetzt, \cite{Borg2019})\label{fig:szz}}
\end{figure}

Die zweite Phase des SZZ-Algorithmus, die als Input eine Liste der Commit-Hashes der zuvor erkannten korrektiven Commits (a) erhält, beginnt mit der Ausführung eines \texttt{git blame} Befehls (b) zur Identifizierung sämtlicher Commits, in denen Veränderungen an den selben Dateien und Codezeilen vorgenommen wurden wie in den korrektiven Commits \cite{Borg2019}. Daraus resultieren mögliche fehlereinführende Commit-Kandidaten (c). Für jeden dieser Commit-Kandidaten wird dann erörtert, ob er fehlereinführend ist (d). Dazu wird zunächst das Datum des Commit-Kandidaten mit dem zugehörigen korrektiven Commits verglichen. Liegt dieses vor dem Datum des korrektiven Commits, so gilt der Kandidat als tatsächlich fehlereinführend \cite{Borg2019}. Liegt das Datum danach, so kann der Kandidat nur fehlereinführend sein, sofern er teilweise den vorhandenen Fehler löst (teilweiser Fix) oder für einen anderen Fehler verantwortlich ist, der nicht dem korrektiven Commit zugehörig ist (Kandidat ist Fehlerursache eines anderen korrektiven Commits) \cite{Borg2019}. Die Ausgabe ist eine Liste von Commit-Hashes von fehlereinführenden Commits für jeden korrektiven Commit. Diese neuen Informationen werden in einer zusätzlichen boole'schen Spalte in den MySQL-Tabellen gespeichert (true = fehlereinführend, false = nicht fehlereinführend). 

Eine Übersicht des Schemas der nun vollständigen initialen  MySQL-Tabellen (im folgenden Haupttabellen genannt) ist in Tabelle XX aufgeührt. Wie bereits zuvor erwähnt, umfasst diese Tabellle für jede veränderte Datei eines Commits eine Ziele. Sollten in einem Diff einer veränderten Datei mehrere Features identifiziert worden sein, so wird für jedes Feature die entsprechende Zeile dupliziert.

\begin{table}[]
\caption{Übersicht des Schemas der MySQL-Haupttabellen}
\label{tab:schema1}
\resizebox{\textwidth}{!}{%
\begin{tabular}{ll|ll}
\textbf{Spaltenname} & \textbf{Beschreibung}                                                                              & \textbf{Spaltenname} & \textbf{Beschreibung}                                                                           \\ \hline
name                 & Name des Softwareprojekts                                                                          & lines\_added         & \begin{tabular}[c]{@{}l@{}}Anzahl der hinzugefügten Zeilen\\ zur geänderten Datei\end{tabular}  \\
release\_number      & \begin{tabular}[c]{@{}l@{}}zugehörige Release-Version\\ basierend auf vergebenen Tags\end{tabular} & lines\_removed       & \begin{tabular}[c]{@{}l@{}}Anzahl der entfernten Zeilen\\ von der geänderten Datei\end{tabular} \\
commit\_hash         & \begin{tabular}[c]{@{}l@{}}eindeutiger Bezeichner eines \\ Commits\end{tabular}                    & change\_type         & Art der Änderung                                                                                \\
commit\_author       & Autor eines Commits                                                                                & diff                 & Diff der geänderten Datei                                                                       \\
commit\_msg          & Nachricht eines Commits                                                                            & corrective           & \begin{tabular}[c]{@{}l@{}}Indikator, ob Commit\\ fehlerbehebend war\end{tabular}               \\
filename             & Name der geänderten Datei                                                                          & bug\_introducing     & \begin{tabular}[c]{@{}l@{}}Indikator, ob Commit\\ fehlereinführend war\end{tabular}             \\
nloc                 & \begin{tabular}[c]{@{}l@{}} \glqq Lines of code\grqq{} der geänderten\\ Datei\end{tabular}                     & feature              & \begin{tabular}[c]{@{}l@{}}Namen der zugehörigen Features\\ der geänderten Datei\end{tabular}   \\
cycomplexity         & \begin{tabular}[c]{@{}l@{}}Zyklomatische Komplexität\\ der geänderten Datei\end{tabular}           &                      &                                                                                                
\end{tabular}%
}
\end{table}

Auf Basis der Daten der Haupttabellen können nun die für das Training der Klassifikatoren benötigten Metriken berechnet werden.

\section{Metriken}

Wie bereits in Kapitel XX erwähnt wurde, bilden sogenannte Metriken die Attribute zum Training der Machine-Learning-Klassifikatoren. Bei Metriken handelt sich sich um Zahlenwerte, die in Codemetriken und Prozessmetriken aufgeteilt sind und jeweils anhand der vorhandenen Daten des Datensets berechnet werden \cite{Rahman2013}. Codemetriken werden genutzt um Eigenschaften von Sourcecode, wie zum Beispiel \glqq Größe\grqq{} oder Komplexität, zu messen \cite{Rahman2013}. Prozessmetriken dienen hingegen zur Messung von Eigenschaften, die anhand von Metadaten aus Software-Repositories erörtert werden können \cite{Rahman2013}. Beispiele dafür sind Anzahl der Veränderungen einer bestimmten Datei oder Anzahl der aktiven Entwickler an einem Projekt. Für diese Arbeit wurden 11 Metriken errechnet, aufgeteilt in 7 Prozess- und 4 Codemetriken. Fünf der Prozessmetriken wurden aus wissenschaftlichen Arbeiten \cite{Rahman2013,Queiroz2016} entnommen. Die weiteren sechs Metriken wurden auf Basis der von PyDriller erhaltenen Metadaten der Commits berechnet.

Im Hinblick auf die spätere Evaluation der Arbeit wurden die Metriken nicht nur auf Basis von Features sondern auch auf Basis von Dateien berechnet. Der letztgenannte Ansatz stellt die in der Machine-Learning-gestützten Fehlererkennung üblicherweise verwendete Methodik dar. Diese beiden Ansätze können somit im Rahmen der Evaluation verglichen werden. Für die Berechnung der dateibasierten Metriken mussten die Daten der Haupttabellen nicht weiter verarbeitet werden, da die erforderlichen Metadaten bezüglich der Dateien bereits mit PyDriller abgerufen wurden, da sie die Grundlage der Identifikation der Features bildeten. In Abbildung XX wird die Abgrenzung zwischen feature- und dateibasierten Datensets ahhand des Ablaufs der Verarbeitung der Rohdaten von PyDriller visualisiert. Es ist zu erkennen, dass für die Berechnung der dateibasierten Metriken keine weiteren Verarbeitungsschritte (Schritte A + B) nötig sind. Lediglich zur Herstellung des Featurebezugs sind weitere Schritte nötig, welche bereits im vorherigen Abschnitt erläutert wurden (Schritte C - E). Eine Übersicht der berechneten Metriken samt Beschreibung befindet sich in Tabelle XX.

\begin{figure}[]
    \centering
    \includegraphics[width=\textwidth]{images/Dataset}
    \caption{Visualisierung zur Unterscheidung der Datensets\label{fig:dataset}}
\end{figure}

\textbf{Ergänzen!!!! Feature und File betrachtung}

\begin{table}
\centering
\caption{Übersicht der berechneten Metriken}
\label{tab:metrics}
\resizebox{\linewidth}{!}{%
\begin{tabular}{l|llc} 
\cline{2-4}
                             & \textbf{Metrik}                                                                               & \textbf{Beschreibung}                                                                                                                                                                                                                                                                                                                                        & \multicolumn{1}{c|}{\textbf{Quelle} }  \\ 
\hline
\multirow{7}{*}{\textbf{\parbox[t]{2mm}{\multirow{3}{*}{\rotatebox[origin=c]{90}{Prozessmetriken}}}}} & Anzahl der~Commits (COMM)                                                                     & Anzahl der Commits, die dem Feature / der Datei in einem Release zugeordnet sind.                                                                                                                                                                                                                                                                            & \cite{Rahman2013,Queiroz2016}                                    \\
                             & \begin{tabular}[c]{@{}l@{}}Anzahl der\\aktiven Entwickler (ADEV)\end{tabular}                 & \begin{tabular}[c]{@{}l@{}}Anzahl der Entwickler, die innerhalb eines Releases das Feature /\\die Datei bearbeitet (geändert, gelöscht oder hinzugefügt) haben.\end{tabular}                                                                                                                                                                                 & \cite{Rahman2013,Queiroz2016}                                    \\
                             & \begin{tabular}[c]{@{}l@{}}eindeutige\\Entwickleranzahl (DDEV)\end{tabular}                   & \begin{tabular}[c]{@{}l@{}}kumulierte Anzahl der Entwickler, die innerhalb eines Releases das Feature / \\die Datei bearbeitet (geändert, gelöscht oder hinzugefügt) haben.\end{tabular}                                                                                                                                                                     & \cite{Rahman2013,Queiroz2016}                                    \\
                             & \begin{tabular}[c]{@{}l@{}}Erfahrung~aller\\Entwickler (EXP)\end{tabular}                     & \begin{tabular}[c]{@{}l@{}}geometrisches Mittel der \glqq Erfahrung\grqq{} aller Entwickler, die innerhalb eines Releases\\das Feature /~die Datei bearbeitet (geändert, gelöscht oder hinzugefügt) haben. \\Erfahrung ist definiert als Summe der geänderten, gelöschten oder hinzugefügten \\Zeilen in den dem Feature / der Datei zugeordneten Commits. \end{tabular} & \cite{Rahman2013,Queiroz2016}                                    \\
                             & \begin{tabular}[c]{@{}l@{}}Erfahrung des meist \\beteiligten Entwicklers\\(OEXP)\end{tabular} & \begin{tabular}[c]{@{}l@{}}\glqq Erfahrung\grqq{} des Entwicklers, die innerhalb eines Releases das Feature / die Datei \\am häufigsten bearbeitet (geändert, gelöscht oder hinzugefügt) hat.\\Erfahrung ist definiert als Summe der geänderten, gelöschten oder hinzugefügten\\Zeilen in den dem Feature / der Datei zugeordneten Commits. \end{tabular}                & \cite{Rahman2013,Queiroz2016}                                    \\
                             & \begin{tabular}[c]{@{}l@{}}Grad der\\Änderungen (MODD)\end{tabular}                           & \begin{tabular}[c]{@{}l@{}}Anzahl der Bearbeitungen (Änderung, Entfernung, Erweiterung) des Features /\\der Datei innerhalb eines Releases.\end{tabular}                                                                                                                                                                                                     & neu                                    \\
                             & \begin{tabular}[c]{@{}l@{}}Umfang der\\Änderungen (MODS)\end{tabular}                         & \begin{tabular}[c]{@{}l@{}}Anzahl der bearbeiteten Features / Dateien innerhalb eines Releases \\(Feature- bzw. dateiübergreifender Wert). Idee: Je mehr Features / \\Dateien in einem Release bearbeitet worden, desto fehleranfälliger scheinen \\diese zu sein. \end{tabular}                                                                             & neu                                    \\ 
\hline
\multirow{4}{*}{\textbf{\parbox[t]{2mm}{\multirow{3}{*}{\rotatebox[origin=c]{90}{Codemetriken}}}}} & \begin{tabular}[c]{@{}l@{}}Anzahl der\\Codezeilen (NLOC)\end{tabular}                         & \begin{tabular}[c]{@{}l@{}}Durchschnittliche Anzahl der Codezeilen der dem Feature \\zugeordneten Dateien / der Datei innerhalb eines Releases.\end{tabular}                                                                                                                                                                                                 & neu                                    \\
                             & \begin{tabular}[c]{@{}l@{}}Zyklomatische\\Komplexität (CYCO)\end{tabular}                     & \begin{tabular}[c]{@{}l@{}}Durchschnittliche zyklomatische Komplexität der dem Feature \\zugeordneten Dateien / der Datei innerhalb eines Releases.\end{tabular}                                                                                                                                                                                             & neu                                    \\
                             & \begin{tabular}[c]{@{}l@{}}Anzahl der\\hinzugefügten Zeilen (ADDL)\end{tabular}               & \begin{tabular}[c]{@{}l@{}}Durchschnittliche Anzahl der hinzugefügten Codezeilen zu den dem Feature\\zugeordneten Dateien / zur Datei innerhalb eines Releases.\end{tabular}                                                                                                                                                                                 & neu                                    \\
                             & \begin{tabular}[c]{@{}l@{}}Anzahl der\\entfernten Zeilen (REML)\end{tabular}                  & \begin{tabular}[c]{@{}l@{}}Durchschnittliche Anzahl der gelöschten Codezeilen von den dem Feature\\zugeordneten Dateien /~ von der Datei innerhalb eines Releases.\end{tabular}                                                                                                                                                                              & neu                                   
\end{tabular}
}
\end{table}

\begin{table}
\centering
\caption{Overview of used metrics}
\label{tab:metrics}
\resizebox{\linewidth}{!}{%
\begin{tabular}{l|llc} 
\cline{2-4}
                             & \textbf{Metric}                                                                            & \textbf{Description}                                                                                                                                                                                                                                                                                        & \multicolumn{1}{c|}{\textbf{Source} }  \\ 
\hline
\multirow{7}{*}{\textbf{\rotatebox[origin=c]{90}{Process metrics}}} & Number of commits (COMM)                                                                   & Number of commits associated with the feature/file in a release.                                                                                                                                                                                                                                             & \cite{Rahman2013,Queiroz2016}                                    \\
                             & Number of active developers (ADEV)                                                         & \begin{tabular}[c]{@{}l@{}}Number of developers who have edited (changed, deleted or added) \\the feature / file within a release\end{tabular}                                                                                                                                                               & \cite{Rahman2013,Queiroz2016}                                    \\
                             & Number of distinct developers (DDEV)                                                       & \begin{tabular}[c]{@{}l@{}}Cumultative number of developers who have edited (changed, deleted or added) \\the feature / file within a release\end{tabular}                                                                                                                                                   & \cite{Rahman2013,Queiroz2016}                                    \\
                             & Experience of all develepoers (EXP)                                                        & \begin{tabular}[c]{@{}l@{}}geometric mean of the experience of all developers who have edited \\(changed, deleted or added) the feature / file within a release.~\\Experience is defined as the sum of the changed, deleted or added \\lines in the commits associated with the feature / file.\end{tabular} & \cite{Rahman2013,Queiroz2016}                                    \\
                             & \begin{tabular}[c]{@{}l@{}}Experience of the most involved developers\\(OEXP)\end{tabular} & \begin{tabular}[c]{@{}l@{}}Experience of the developer who has edited (changed, deleted or added) \\the feature / file most often within a release. Experience is defined as the \\sum of changed, deleted, or added lines in the commits associated with the \\feature/file.\end{tabular}                   & \cite{Rahman2013,Queiroz2016}                                    \\
                             & Degree of modifications (MODD)                                                             & Number of edits (change, removal, extension) of the feature / file within a release.                                                                                                                                                                                                                         & new                                    \\
                             & Scope of modifications (MODS)                                                              & \begin{tabular}[c]{@{}l@{}}Number of edited features / files within a release (feature or file overlapping value). \\Idea: The more features / files have been edited in a release, \\the more error-prone they seem to be.\end{tabular}                                                                     & new                                    \\ 
\hline
\multirow{4}{*}{\textbf{\rotatebox[origin=c]{90}{Code metrics}}} & Lines of code (NLOC)                                                                       & \begin{tabular}[c]{@{}l@{}}Average number of lines of code of the files associated with the feature /\\~file within a release.\end{tabular}                                                                                                                                                                  & new                                    \\
                             & Cyclomatic Complexity (CYCO)                                                               & \begin{tabular}[c]{@{}l@{}}Average cyclomatic complexity of the files associated with the feature / \\file within a release.\end{tabular}                                                                                                                                                                    & new                                    \\
                             & Number of added lines (ADDL)                                                               & \begin{tabular}[c]{@{}l@{}}Average number of lines of code added to the files associated with \\the feature / file within a release.\end{tabular}                                                                                                                                                            & new                                    \\
                             & Number of removed lines (REML)                                                             & \begin{tabular}[c]{@{}l@{}}Average number of lines of code deleted from the files associated \\with the feature / from the file within a release\end{tabular}                                                                                                                                                & new                                   
\end{tabular}
}
\end{table}

\begin{table}[]
\caption{Übersicht des Schemas der Metrics-Tabellen des Datensets}
\label{tab:schema2}
\resizebox{\textwidth}{!}{%
\begin{tabular}{ll|ll}
\textbf{Spaltenname} & \textbf{Beschreibung}                                                                                                                                                                          & \textbf{Spaltenname} & \textbf{Beschreibung}                                                                                                                                                           \\ \hline
name                 & Name des Softwareprojekts                                                                                                                                                                      & oexp                 & \begin{tabular}[c]{@{}l@{}}"Erfahrung" des Entwicklers, der am\\ meisten zum betreffenden Feature / \\ zur betreffenden Datei in  einem Release \\ beigetragen hat\end{tabular} \\
release\_number      & \begin{tabular}[c]{@{}l@{}}zugehörige Release-Version\\ basierend auf vergebene Tags\end{tabular}                                                                                              & scat                 & \begin{tabular}[c]{@{}l@{}}Scattering Degree des betreffenden Features / \\ der betreffenden Datei\end{tabular}                                                                 \\
feature / filename   & \begin{tabular}[c]{@{}l@{}}betreffendes Feature / \\ betreffende Datei\end{tabular}                                                                                                            & tang                 & \begin{tabular}[c]{@{}l@{}}Tangling Degree des betreffenden Features / \\ der betreffenden Datei\end{tabular}                                                                   \\
comm                 & \begin{tabular}[c]{@{}l@{}}Anzahl der Commits, die in einem\\ Release dem betreffenden Feature / \\ der betroffenen Datei gewidmet sind\end{tabular}                                           & nloc                 & \begin{tabular}[c]{@{}l@{}}Durchschnittliche Lines of Code der Bearbeitungen \\ des betreffenden Features / der betreffenden Datei\\ in einem Release\end{tabular}              \\
adev                 & \begin{tabular}[c]{@{}l@{}}Anzahl der Entwickler, die das\\ betreffende Feature / die betreffende \\ Datei in einem Release bearbeitet haben\end{tabular}                                      & cyco                 & \begin{tabular}[c]{@{}l@{}}Durchschnittliche zyklomatische Komplexität der \\ Bearbeitungen  des betreffenden Features /\\ der betreffenden Datei in einem Release\end{tabular} \\
ddev                 & \begin{tabular}[c]{@{}l@{}}kummulierte Anzahl der Entwickler, \\ die das betreffende Feature / \\ die betreffende Datei in einem\\ Release bearbeitet haben\end{tabular}                       & addl                 & \begin{tabular}[c]{@{}l@{}}Durchschnittliche Anzahl der hinzugefügten Zeilen \\ des betreffenden Features / der betreffenden Datei\\ in einem Release\end{tabular}              \\
exp                  & \begin{tabular}[c]{@{}l@{}}Geometrisches Mittel der "Erfahrung"\\ aller Entwickler, die am betreffenden\\ Feature / an der betreffenden Datei\\ in einem Release gearbeitet haben\end{tabular} & reml                 & \begin{tabular}[c]{@{}l@{}}Durchschnittliche Anzahl der entfernten Zeilen \\ des betreffenden Features / der betreffenden Datei\\ in einem Release\end{tabular}                
\end{tabular}%
}
\end{table}



\cleardoublepage
    
    % !TEX root = ../thesis.tex

\chapter{Training und Test der Machine-Learning-Klassifikatoren}

\paragraph{Ausblick:}
Dieses Kapitel gibt einen detaillierten Einblick in das Training der Machine-Learning-Klassifikatoren. Dazu werden zunächst die verwendeten Klassifikatoren und deren initiale Auswahl erläutert. Anschließend werden der Trainingsprozess sowie die zum Einsatz kommenden Softwarewerkzeuge beschrieben.
\\
\hrule

\section{Auswahl der Werkzeuge und Klassifikationsalgorithmen}

Durch die Wahl der Programmiersprache Python, war die Entscheidung zur Auswahl eines Machine-Learning-Werkzeugs bereits absehbar. Zur Anwendung kommt die Python-Library scikit-learn\footnote{\href{https://scikit-learn.org/}{https://scikit-learn.org/}}, die im Jahr 2007 von Pedregosa et. al entwickelt wurde \cite{scikit}. Das Werkzeug bietet eine große Auswahl an Machine-Learning-Algorithmen für überwachtes und unüberwachtes Lernen und ermöglicht darüber hinaus eine einfache Implementation sowie eine einfache Einbindung weiterer Python-Libraries, wie beispielsweise die Matplotlib zur Erstellung von mathematischen Darstellungen \cite{scikit}.

Ebenfalls wird der WEKA-Workbench\footnote{\href{https://www.cs.waikato.ac.nz/ml/weka/}{https://www.cs.waikato.ac.nz/ml/weka/}} als weiteres Machine-Learning-Wekzeug verwendet. Im Rahmen der strukturierten Literaturanalyse zu Beginn der Erarbeitung der Masterarbeit, erwies sich dieses Werkzeug durch zahlreiche Zitierungen in wissenschaftlichen Arbeiten (unter anderem in \cite{Queiroz2016}) ebenfalls als geeignet. Der WEKA-Workbench (WEKA als Akronym für Waikato Environment for Knowledge Analysis) wurde an der University of Waikato in Neuseeland entwickelt und bietet eine große Kollektion an Machine-Learning-Algorithmen und Preprocessing-Tools zur Verwendung innerhalb einer grafischen Benutzeroberfläche \cite{Weka2016}. 

Die Verwendung von zwei Machine-Learning-Werkzeugen ermöglicht einen Vergleich der jeweiligen Implementierungen der verwendeten Klassifikationsalgorithmen in der anschließenden Evaluation. Eine Übersicht über die ausgewählten Klassifikationsalgorithmen befindet sich in Tabelle XX. Kurze Erläuterungen der Algorithmen befinden sich im Anschluss.

\begin{table}
\centering
\caption{Zum Training verwendete Klassifikationsalgorithmen}
\label{tab:classifiers}
\resizebox{\linewidth}{!}{%
\begin{tabular}{|>{\hspace{0pt}}p{0.497\linewidth}|>{\hspace{0pt}}p{0.499\linewidth}|} 
\hline
\textbf{scikit-learn}  & \textbf{WEKA}  \\ 
\hline
Decision Trees & J48-Decision-Trees \\
k-Nearest-Neighbors & k-Nearest-Neighbors \\
Ridge Classifier & Logistic Regression \\
Na\"{\i}ve Bayes & Na\"{\i}ve Bayes \\
künstliche neuronale Netze & künstliche neuronale Netze \\
Random Forest & Random Forest \\
Stochastic Gradient Descent & Stochastic Gradient Descent \\
Support Vector Machines & Support Vector Machines \\
\hline
\end{tabular}
}
\end{table}

\label{algorithms}
\subsubsection*{Decision Trees}

\textbf{Überarbeiten?}

Decision Trees (deutsch: Entscheidungsbäume) zählen zu den meistverwendeten Klassifikatoren im Bereich des supervised Machine Learnings. Studien belegten, dass sie hinsichtlich der Verwendung im Kontext von Fehlererkennung am häufigsten Anwendung finden \cite{Son2019}. Decision Trees sind gerichtete und verwurzelte Bäume, die als rekursive Partition der Eingabemenge des Datensets aufgebaut wird \cite{Rokach2005}. Den Ursprung des Baumes bildet die Wurzel, welche keine eingehenden Kanten besitzt - alle weiteren Knoten besitzen jedoch eine eingehende Kante \cite{Rokach2005}. Diese Knoten teilen wiederum die Eingabemenge anhand einer vorgegebenen Funktion in zwei oder mehr Unterräume der Menge auf \cite{Rokach2005}. Meist geschieht dies anhand eines Attributs, sodass die Eingabemenge anhand der Werte des einzelnen Attributs geteilt wird \cite{Rokach2005}. Die Blätter des Baumes bilden die Zielklassen ab. Eine Klassifizierung kann folglich durchgeführt werden, indem man von der Wurzel bis zu einem Blatt den Kanten anhand der entsprechenden Werte der Eingangsmenge folgt. Es existieren verschiedene Algorithmen zur Erstellung von Decision Trees. Bekannte Stellvertreter dieser sind ID3, C4.5 (J48) und CART \cite{Rokach2005}. Der grundlegende Aufbau eines Decision Trees ist in Abbildung XX dargestellt.

\begin{figure}[]
    \centering
    \includegraphics[width=0.5\textwidth]{images/DT}
    \caption{Grundsätzlicher Aufbau eines Decision Trees\label{fig:dt}}
\end{figure}

Eine Besonderheit von Decision Trees stellen sogenannte Random Forests dar. Diese beschreiben eine Lernmethode von Klassifikatoren, bei der mehrere einzelne Decision Trees gleichzeitig erzeugt werden und deren Ergebnisse anschließen aggregiert werden \cite{Alam2013}. Dazu erhält jeder Decision Tree eine Teilmenge der Eingabemenge des Datensets \cite{Alam2013}. Random Forests eigenen sich besonders zur Anwendung, wenn viele Attribute im Datenset vorhanden sind \cite{Alam2013}.

\subsubsection*{k-Nearest-Neighbors}
Ein k-Nearest-Neighbor-Klassifikator (deutsch: k-nächste-Nachbarn) basiert auf zwei Konzepten \cite{Zhang2016}. Das erste basiert auf der Abstandsmessung zwischen den Werten der zu klassifizierenden Datenmenge und den Werten der Attribute des Datensets \cite{Zhang2016}. Die Abstandmessung erfolgt in der Regel durch die Berechnung der Euklidischen Distanz (siehe Abbildung XX). Das zweite Konzept bildet der Parameter k, der angibt, wie viele nächste Nachbarn zum Vergleich der zuvor berechneten Abstände in Betracht gezogen werden. Bei einem k > 1 wird diejenige Zielklasse gewählt, deren Auftreten innerhalb der nächsten Nachbarn überwiegt.

\begin{figure}[]
    \centering
    \includegraphics[width=0.4\textwidth]{images/EUKLID}
    \caption{Formel zur Berechnung der Euklidischen Distanz (n = Anzahl der Attribute)\label{fig:euklid}}
\end{figure}

\subsubsection*{Künstliche neuronale Netze}
Künstliche neuronale Netze (KNN) verwenden nicht-lineare Funktionen zur schrittweisen Erzeugung von Beziehungen zwischen der Eingabemenge und den Zielklassen durch einen Lernprozess \cite{Linder2004}. Sie sind angelehnt an die Funktionsweise von biologischen Nervensystemen und bestehen aus einer Vielzahl von einander verbundenen Berechnungsknoten, den Neuronen \cite{OShea2015}. Der grundsätzliche Aufbau eines künstlichen neuronalen Netzes kann in Abbildung XX eingesehen werden. Der Lernprozess besteht aus zwei Phasen - einer Trainingphase und einer Recall-Phase \cite{Linder2004}. In der Trainingsphase werden die Eingabedaten, meist als multidimensionaler Vektor, in den Input-Layer geladen und anschließend an die Hidden-Layer verteilt \cite{OShea2015}. In den Hidden-Layers werden dann Entscheidungen anhand der Beziehungen zwischen den Eingabedaten und Zielklassen sowie die den Verbindungen zuvor zugewiesenen Gewichtsfaktoren getroffen \cite{Linder2004}. 
\textbf{HIER!}
\cite{Linder2004}

\begin{figure}[]
    \centering
    \includegraphics[width=0.6\textwidth]{images/ANN}
    \caption{Grundsätzlicher Aufbau eines KNN mit 4 Input-Neuronen, 5 Hidden-Neuronen und 2 Output-Neuronen\label{fig:ann}}
\end{figure}

\textbf{HIER!}

\subsubsection*{Na\"{\i}ve Bayes}
Na\"{\i}ve-Bayes-Klassifikatoren zählen zu den linearen Klassifikatoren und basieren auf dem Satz von Bayes. Die Bezeichnung "naiv" erhält der Klassifikator durch die Annahme, dass Attribute der Eingabemenge unabhängig voneinander sind (diese Annahme wird häufig verletzt, dennoch erzielt der Klassifikator eine hohe Performanz) \cite{Raschka2014}. Der Klassifikator gilt als effizient, robust, schnell und einfach implementierbar \cite{Raschka2014}. Die zur Durchführung einer Klassifikation mittels Na\"{\i}ve Bayes benötigte Formel nach Thomas Bayes ist in Abbildung XX samt Erläuterung aufgeführt.

\begin{figure}[]
    \centering
    \includegraphics[width=0.7\textwidth]{images/NB}
    \caption{Satz von Bayes als Grundlage des Na\"{\i}ve-Bayes-Klassifikators\label{fig:nb}}
\end{figure}

Es existiert zudem eine Mehrzahl an Varianten des Na\"{\i}ve-Bayes-Klassifikators, die verschiedene Annahmen über die Verteilung der Attribute der Eingabemenge machen. Beispiele dafür sind der Gauß'sche-Na\"{\i}ve-Bayes (normalverteilte Attribute), der multinomiale Na\"{\i}ve-Bayes (multinomiale Verteilung der Attribute) sowie der Bernoulli-Na\"{\i}ve-Bayes (unabhängige binäre Attribute).

\subsubsection*{Logistic Regression}
Logistische Regressions-Klassifikatoren basieren auf dem mathematischen Konzept des Logits, welcher den natürlichen Logarithmus eines Chancenverhältnisses beschreibt \cite{Peng2002}. Am besten geeignet ist dieser Klassifikator für eine Kombination aus kategorischen oder kontinuierlichen Eingabedaten und kategorischen Zielklassen \cite{Peng2002}.

\textbf{HIER}

\subsubsection*{Stochastic Gradient Descent}
\cite{Bottou2010}

\subsubsection*{Support Vector Machines}
Support Vector Machines verfolgen das Ziel, linear separierbare Klassen 
\cite{Tzotsos2006}

\fbox{\parbox{\linewidth}{\textit{RQ2: WELCHE MACHINE-LEARNING-KLASSIFIKATOREN KOMMEN FÜR DIE GEGEBENE AUFGABE IN FRAGE?\medskip }\\
Es werden neun verschiedene Klassifikationsalgorithmen zur Anwendung kommen. Sieben Algorithmen werden sowohl mit scikit-learn als auch mit WEKA verwendet (DT / J48, KNN, NB, NN, RF, SGD, SVM). Jeweils ein Algorithmus ist werkzeugspezifisch (scikit-learn: RC, WEKA: LR), jedoch unterliegen beide Algorithmen dem Konzept der Regression. Das Hauptkriterium für die Auswahl sämtlicher Algorithmen war die vorherige Verwendung im Rahmen der wissenschaftlichen Literatur \cite{Son2019}.}}

\section{Analyse des Testprozesses}

\textbf{KONFIGURATION DER KLASSIFIKATOREN ERLÄUTERN + FINALE KONFIG ALS TABELLE}\\\textbf{SMOTE}

Im weiteren Verlauf dieses Abschnitts und im Rahmen der Evaluation im folgenden Kapitel, werden die Namen der Klassifikatoren auf Abbildungen abgekürzt. Die Abkürzungen können Tabelle XX entnommen werden.

\begin{table}
\centering
\caption{Zuordnung der verwendeten Abkürzungen}
\label{tab:abbs}
\resizebox{\linewidth}{!}{%
\begin{tabular}{|>{\centering\hspace{0pt}}p{0.15\linewidth}>{\hspace{0pt}}p{0.329\linewidth}|>{\centering\hspace{0pt}}p{0.15\linewidth}>{\hspace{0pt}}p{0.362\linewidth}|} 
\hline
\textbf{Abkürzung}  & \textbf{Klassifikator}  & \textbf{Abkürzung}  & \textbf{Klassifikator}  \\ 
\hline
DT / J48 & Decision Trees & RC & Ridge Classifier \\
KNN & k-Nearest-Neighbor & RF & Random Forest \\
LR & Logistic Regression & SGD & Stochastic Grandient Descent \\
NB & Na\"{\i}ve Bayes & SVM & Support Vector Machines \\
NN & künstliche neuronale Netze &  &  \\
\hline
\end{tabular}
}
\end{table}

Die Analyse des Testprozesses zeigte zudem, dass das dateibasierte Datenset stark unbalanciert hinsichtlich der Zielklasse ist. Mit einem Wert von etwa 98\% existieren weitaus mehr Einträge, die dem Label \glqq fehlerfrei\grqq{} zugeordnet sind. Balanciertheit, also ein ausgeglichenes Verhältnis (50:50 ist im binären Fall nicht zwingend notwendig) innerhalb der Zielklassen, ist jedoch eine Voraussetzung für das korrekte Erlernen der meisten Klassifikatoren. Eine Nichtbeachtung dieses Problem kann zu einer irreführenden Accuracy führen, da die meisten Datensätze korrekt der überrepräsentierten Klasse zugeordnet werden. Als Lösung dieses Problems wurde der sogenannte SMOTE-Algorithmus auf das dateibasierte Datenset angewendet \cite{Chawla2002}. Der Algorithmus, dessen Akronym für \textbf{S}ynthetic \textbf{M}inority \textbf{O}ver-sampling \textbf{Te}chnique steht, führt ein Oversampling der unterrepräsentierten Klasse durch \cite{Chawla2002}. Anhand von nächste-Nachbarn-Berechnungen auf Basis der Euklidischen Distanz zwischen den Attributwerten der einzelnen Datensätze des Datensets, werden neue synthetische Datensätze hinzugefügt (Oversampling), sodass sich die Anzahl der Datensätze der relevanten Klasse erhöht \cite{Chawla2002}. Im hier durchgeführten Fall wurde der Prozentsatz für die Generierung der synthetischen Datensätze auf 3000 festgelegt, sodass für jeden vorhandenen Datensatz der unterrepräsentierten Klasse 30 zusätzliche synthetische Datensätze erzeugt wurden. So konnte der Anteil der Datensätze mit dem Label \glqq fehlerhaft\grqq{} auf etwa 27\% erhöht werden. In Abbildung XX ist dargestellt, welchen Einfluss die Anwendung des SMOTE-Algorithmus auf die Accuracies der Klassifikatoren des datenbasierten Datensets im Rahmen des Testprozesses hatte. Das mit \glqq vorher\grqq{} deklarierte Diagramm zeigt, dass nahezu alle Klassifikatoren eine Accuracy von nahezu 100\% besitzen, was das zuvor beschriebene Problem widerspiegelt. Das Diagramm, welches die Testergebnisse nach Anwendung des SMOTE-Algorithmus darstellt, weist hingegen wesentlich realistischere Accuracies auf.


\begin{figure}[]
    \centering
    \includegraphics[width=0.8\textwidth]{images/smoted}
    \caption{Vergleich der Accuracies je Klassifikator vor und nach der Anwendung des SMOTE-Algorithmus\label{fig:smoted}}
\end{figure}

\begin{figure}[]
    \centering
    \includegraphics[width=\textwidth]{images/Vergleich1}
    \caption{Vergleich der Klassifikatoren und Werkzeuge im Hinblick auf ihre Accuracies\label{fig:vergl1}}
\end{figure}

\cleardoublepage
    
    % !TEX root = ../doc.tex

\section{Evaluation}

Lorem ipsum.
    
    % !TEX root = ../doc.tex

\section{Conclusion}

\st{Lorem ipsum.}


    % Literaturverzeichnis
    \printbibliography[heading=bibintoc]

\appendix

\chapter{Test 1}

Lorem ipsum

\chapter{Test 2}

Lorem ipsum

\end{document}