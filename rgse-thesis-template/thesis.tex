\documentclass[bachelor,twoside,intern,palatino]{rgseThesis}

% für "lorem ipsum" Blindtext - sollte bei echten Arbeiten natürlich raus ;-)
\usepackage{lipsum}

% Angaben für die Titelseite

\author{John Doe}
% \studiengang{Computervisualistik} % Default ist Informatik, bei Proposals ignoriert

\title{Frying without Fat}

% \supervisor{w}{Dr. Sabine Mustermann} % Default ist Jan Jürjens
% \supervisorInfo{Muster-Institut} % Default ist Institut für Softwaretechnik

\secondSupervisor{w}{Dr. Karin Nickel}
\secondSupervisorInfo{Rabbit Burrow Inc.}

\externLogo{6cm}{logos/ist-logo-en} % optional Logo des externen Partners
\externName{Institut für Softwaretechnik} % optional Untertitel des Logos


% Literatur-Datenbank
\addbibresource{literature.bib}

% Tiefe des Inhaltsverzeichnisses; 1: bis section (empfohlen), 2: bis subsection
\setcounter{tocdepth}{1} 

\begin{document}

    % Umschalten der Sprache für englische Rubrikbezeichnungen (möglich: english, ngerman)
    \selectlanguage{english}
    \pagenumbering{roman}

    % Titelseite und Erklärung
    \maketitle

    % Kurzfassung
    % !TEX root = ../thesis.tex

% Kurzfassung in Deutsch und Englisch
\begin{otherlanguage}{ngerman}
    \section*{Kurzfassung}

    Dies ist die Kurzfassung in Deutsch.
    
    \lipsum[1-1]
\end{otherlanguage}

\begin{otherlanguage}{english}
    \section*{Abstract}

    This is the abstract in English.
    
    \lipsum[2-2]
\end{otherlanguage}
\cleardoublepage


    \tableofcontents
    \cleardoublepage

    % \listoffigures   % fuer ein eventuelles Abbildungsverzeichnis
    % \cleardoublepage

    \pagenumbering{arabic}

    % Hier kommt jetzt der eigentliche Text der Arbeit
    
    % Du solltest mit \include für die einzelnen Kapitel arbeiten,
    % damit die Dokumente nicht zu lang werden

    % !TEX root = ../thesis.tex

\chapter{Introduction}

In `The Bible' \cite{Juerjens2005SSD}, Jan Jürjens defines an extension to the UML modeling language that enables modeling of security features in UML diagrams.

\paragraph{Suggestion:}
To make version control more easy, try to put each sentence on a single line.
Use your text editors wrapping option to wrap long sentences (i.e. lines) at the window borders.

\paragraph{Suggestion:}
You may optionally end each chapter by a \verb+\cleardoublepage+ command.
This flushes all floating objects (figures, tables, etc.) and makes chapters start on right pages in twoside mode.
You can safely use \verb+\cleardoublepage+ also in single sided printing, it behaves the same as \verb+\clearpage+.

\section{Some text}

\subsection{First part}

\lipsum[1-1]

\subsection{Second part}

\lipsum[2-3]

\section{Some more text}

\lipsum[4-7]

\cleardoublepage


    % !TEX root = ../thesis.tex

\chapter{Extending BibLaTeX and JabRef with Entry Type for Standards}

Standards, e.g.\ ISO or DIN standards, are a document type that is not very well supported by the standard BibLaTeX entry types. However, Katharina Großer provided a BibLaTeX driver\footnote{cf.\ document class file \texttt{rgseThesis.cls}} to deal with the special fields and format that often describe standards.

JabRef\footnote{\url{https://www.jabref.org}} can be used to manage your bibliography. It can easily be extended to handle the entry type \texttt{standard}. Use the menu \emph{BibTeX -- Customize Entry Type} to add the new type. See \prettyref{fig:jabref} for details.

\begin{figure}
    \centering
    \includegraphics[height=10cm]{images/JabRef}
    \caption{Settings for entry type \texttt{standard}\label{fig:jabref}}
\end{figure}

Here are some cites to test for the bibliography entry type \texttt{standard}: ISO 25964 \cite{ISO20111TI, ISO20132IO} defines an international standard for glossaries.

\cleardoublepage


    % Literaturverzeichnis
    \printbibliography[heading=bibintoc]

\appendix

\chapter{Test 1}

\chapter{Test 2}

\end{document}