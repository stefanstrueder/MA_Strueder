% !TEX root = ../thesis.tex

\chapter{Erstellung eines featurebasierten Datensets}

\paragraph{Ausblick:}
Dieses Kapitel widmet sich der schrittweisen Erläuterung des Prozesses zur Erstellung des featurebasierten Datensets, welches zur Anlernung der Machine-Learning-Klassifikatoren dient. Dazu wird zunächst die Datenauswahl näher beleuchtet. Darauf folgt eine Darlegung der Konstruktion des Datensets sowie der Auswahl und Berechnung der Metriken, welche als Attribute (Features) im Rahmen der Anlernung der Klassifikatoren dienen.
\\
\hrule

\section{Datenauswahl}

\begin{table}[H]
\caption{Übersicht der zur Erstellung des Datensets verwendeten Tools mit zugehörigen Werten}
\label{tab:tools}
\resizebox{\textwidth}{!}{%
\begin{tabular}{lccccccc}
                                       & \textbf{Zweck}       & \textbf{Datenquelle}   & \textbf{\#Releases} & \textbf{\#Commits} & \textbf{\#Korrektiv} & \textbf{\#Fehlereinführend} & \textbf{\#Features} \\ \hline
\multicolumn{1}{l|}{\textbf{Blender}}  & 3D-Modellierungstool & GitHub-Mirror          & 11                  & 19119              & 8258                 & 1418                        & $\sim$3000          \\
\multicolumn{1}{l|}{\textbf{Busybox}}  & UNIX-Toolkit         & Git-Repository         & 14                  & 4984               & 1408                 & 142                         & 702                 \\
\multicolumn{1}{l|}{\textbf{Emacs}}    & Texteditor           & GitHub-Mirror          & 7                   & 12805              & 6959                 & 685                         & 863                 \\
\multicolumn{1}{l|}{\textbf{GIMP}}     & Bildbearbeitung      & GitLab-Repository      & 14                  & 7240               & 1703                 & 272                         & 1620                \\
\multicolumn{1}{l|}{\textbf{Gnumeric}} & Tabellenkalkulation  & GitLab-Repository      & 8                   & 6025               & 1591                 & 136                         & 725                 \\
\multicolumn{1}{l|}{\textbf{gnuplot}}  & Plotting-Tool        & GitHub-Mirror          & 5                   & 6619               & 880                  & 1323                        & 625                 \\
\multicolumn{1}{l|}{\textbf{Irssi}}    & IRC-Client           & GitHub-Repository      & 7                   & 253                & 77                   & 1                           & 17                  \\
\multicolumn{1}{l|}{\textbf{libxml2}}  & XML-Parser           & GitLab-Repository      & 10                  & 732                & 409                  & 37                          & 225                 \\
\multicolumn{1}{l|}{\textbf{lighttpd}} & Webserver            & Git-Repository         & 6                   & 2597               & 1202                 & 555                         & 323                 \\
\multicolumn{1}{l|}{\textbf{MPSolve}}  & Polynomlöser         & GitHub-Repository      & 8                   & 668                & 158                  & 69                          & 130                 \\
\multicolumn{1}{l|}{\textbf{Parrot}}   & Virtuelle Maschine   & GitHub-Repository      & 7                   & 16245              & 3437                 & 824                         & 559                 \\
\multicolumn{1}{l|}{\textbf{Vim}}      & Texteditor           & GitHub-Repository      & 7                   & 9849               & 1033                 & 2571                        & 1227                \\
\multicolumn{1}{l|}{\textbf{xfig}}     & Grafikeditor         & Sourceforge-Repository & 7                   & 18                 & 0                    & 0                           & 205                
\end{tabular}%
}
\end{table}

\section{Konstruktion des Datensets}

\begin{table}[H]
\caption{Übersicht des Schemas der Haupttabellen des Datensets}
\label{tab:schema1}
\resizebox{\textwidth}{!}{%
\begin{tabular}{ll|ll}
\textbf{Spaltenname} & \textbf{Beschreibung}                                                                              & \textbf{Spaltenname} & \textbf{Beschreibung}                                                                           \\ \hline
name                 & Name des Softwareprojekts                                                                          & lines\_added         & \begin{tabular}[c]{@{}l@{}}Anzahl der hinzugefügten Zeilen\\ zur geänderten Datei\end{tabular}  \\
release\_number      & \begin{tabular}[c]{@{}l@{}}zugehörige Release-Version\\ basierend auf vergebenen Tags\end{tabular} & lines\_removed       & \begin{tabular}[c]{@{}l@{}}Anzahl der entfernten Zeilen\\ von der geänderten Datei\end{tabular} \\
commit\_hash         & \begin{tabular}[c]{@{}l@{}}eindeutiger Bezeichner eines \\ Commits\end{tabular}                    & change\_type         & Art der Änderung                                                                                \\
commit\_author       & Autor eines Commits                                                                                & diff                 & Diff der geänderten Datei                                                                       \\
commit\_msg          & Nachricht eines Commits                                                                            & corrective           & \begin{tabular}[c]{@{}l@{}}Indikator, ob Commit\\ fehlerbehebend war\end{tabular}               \\
filename             & Name der geänderten Datei                                                                          & bug\_introducing     & \begin{tabular}[c]{@{}l@{}}Indikator, ob Commit\\ fehlereinführend war\end{tabular}             \\
nloc                 & \begin{tabular}[c]{@{}l@{}} \glqq Lines of code\grqq{} der geänderten\\ Datei\end{tabular}                     & feature              & \begin{tabular}[c]{@{}l@{}}Namen der zugehörigen Features\\ der geänderten Datei\end{tabular}   \\
cycomplexity         & \begin{tabular}[c]{@{}l@{}}Zyklomatische Komplexität\\ der geänderten Datei\end{tabular}           &                      &                                                                                                
\end{tabular}%
}
\end{table}

\begin{table}[]
\caption{Übersicht des Schemas der SZZ-Tabellen des Datensets}
\label{tab:schema3}
\resizebox{\textwidth}{!}{%
\begin{tabular}{ll|ll}
\textbf{Spaltenname} & \textbf{Beschreibung}                                                            & \textbf{Spaltenname} & \textbf{Beschreibung}                                                                     \\ \hline
name                 & Name des Softwareprojekts                                                        & filename             & einem Commit zugehörige Dateien                                                           \\
commit\_hash         & \begin{tabular}[c]{@{}l@{}}Commit-Hashes fehlerbehebender\\ Commits\end{tabular} & bug\_introducing     & \begin{tabular}[c]{@{}l@{}}Auflistung der fehlereinführenden\\ Commit-Hashes\end{tabular} \\
filepath             & einem Commit zugehörige Dateipfade                                               &                      &                                                                                          
\end{tabular}%
}
\end{table}

\section{Metriken}

\textbf{Ergänzen!!!!}

\begin{table}[H]
\caption{Übersicht des Schemas der Metrics-Tabellen des Datensets}
\label{tab:schema2}
\resizebox{\textwidth}{!}{%
\begin{tabular}{ll|ll}
\textbf{Spaltenname}    & \textbf{Beschreibung}                                                                              & \textbf{Spaltenname}   & \textbf{Beschreibung}                                                                                                                         \\ \hline
name                    & Name des Softwareprojekts                                                                          & comm                   & \begin{tabular}[c]{@{}l@{}}Anzahl der Commits, die in einem\\ Release dem betreffenden Feature\\ gewidmet sind\end{tabular}                   \\
release\_number         & \begin{tabular}[c]{@{}l@{}}zugehörige Release-Version\\ basierend auf vergebenen Tags\end{tabular} & adev                   & \begin{tabular}[c]{@{}l@{}}Anzahl der Entwickler, die das\\ betreffende Feature in einem\\ Release bearbeitet haben\end{tabular}              \\
feature                 & betreffendes Feature                                                                               & ddev                   & \begin{tabular}[c]{@{}l@{}}kummulierte Anzahl der Entwickler, \\ die das betreffende Feature in einem\\ Release bearbeitet haben\end{tabular} \\
\multicolumn{1}{c}{...} & \multicolumn{1}{c|}{...}                                                                           & \multicolumn{1}{c}{..} & \multicolumn{1}{c}{...}                                                                                                                      
\end{tabular}%
}
\end{table}

\cleardoublepage