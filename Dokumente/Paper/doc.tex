\documentclass[sigconf]{acmart}
\usepackage{hyperref}
\usepackage{relsize,xspace}
\usepackage[framemethod=tikz,
innerleftmargin=1pt,
innerrightmargin=1pt,
innertopmargin=1pt,
innerbottommargin=1pt,
skipabove=6pt,
skipbelow=6pt,
splittopskip=2pt,
splitbottomskip=2pt,
roundcorner = 0pt
]{mdframed}
\usepackage{listings}
\usepackage{multirow}
\usepackage{microtype}

\definecolor{light-gray-fork}{gray}{0.925}
\definecolor{light-gray-mainline}{gray}{0.8}
\definecolor{javared}{rgb}{0.6,0,0} % for strings
\definecolor{javagreen}{rgb}{0.25,0.5,0.35} % comments
\definecolor{javapurple}{rgb}{0.5,0,0.35} % keywords
\definecolor{javadocblue}{rgb}{0.25,0.35,0.75} % javadoc
\lstdefinestyle{all}{
	basicstyle=\fontsize{11}{11}\selectfont\ttfamily\footnotesize,
	numbers=none,
	xleftmargin=0.1cm,
	xrightmargin=0.1cm,
	stepnumber=1,
	showstringspaces=false,
	tabsize=1,
	breaklines=true,
	breakatwhitespace=false,
	frame=single,
	commentstyle=\color{gray}
}

\lstdefinestyle{result} {
	frame=none,
	escapechar=`,
	language=Java,
	basicstyle=\fontsize{11}{12}\scriptsize,
	postbreak=\mbox{\textcolor{gray}{$\hookrightarrow$}\space},
	keywordstyle=\color{javapurple}\bfseries,
	stringstyle=\color{javared},
	commentstyle=\color{javagreen},
	morecomment=[s][\color{javadocblue}]{/**}{*/},
	moredelim=**[is][\color{red}]{@@}{@@},
}

\lstdefinestyle{dataset} {
	frame=single,
	escapechar=`,
	language=Java,
	basicstyle=\fontsize{11}{12}\scriptsize,
	postbreak=\mbox{\textcolor{gray}{$\hookrightarrow$}\space},
	keywordstyle=\color{javapurple}\bfseries,
	stringstyle=\color{javared},
	commentstyle=\color{javagreen},
	morecomment=[s][\color{javadocblue}]{/**}{*/},
	moredelim=**[is][\color{red}]{@@}{@@},
}
\usepackage{array}
\newcommand*\rotbf[1]{\rotatebox{90}{\textbf{#1}}}
\newcommand{\specialcell}[2][c]{\begin{tabular}[#1]{@{}l@{}}#2\end{tabular}}
\newcommand{\specialcellbold}[2][c]{%
	\bfseries
	\begin{tabular}[#1]{@{}l@{}}#2\end{tabular}%
}
\newcommand\parhead[1]{\vspace{.26mm}\noindent\textbf{{#1}}.}  
\newcommand{\secref}[1]{Section~\ref{#1}}
\newcommand{\figref}[1]{Fig.~\ref{#1}}
\newcommand{\Figref}[1]{Figure~\ref{#1}}
\newcommand{\tabref}[1]{Table~\ref{#1}}
\newcommand{\lstref}[1]{Listing~\ref{#1}}

\newcommand{\tb}[1]{{\textsf{TB}[\smaller\sffamily\color{blue} #1}]}
\newcommand{\rh}[1]{{\textsf{RH}[\smaller\sffamily\color{orange} #1}]}
\newcommand{\mm}[1]{{\textsf{MM}[\smaller\sffamily\color{green!10!orange!90!} #1}]}
\newcommand{\wm}[1]{{\textsf{WM}[\smaller\sffamily\color{red} #1}]}
\newcommand{\es}{ElasticSearch\xspace}

\usepackage{xcolor}

\newcommand{\quotebox}[3]{\vspace{.5em}\noindent\begin{tikzpicture}
\node[align=center,draw,thin,minimum width=\columnwidth,inner sep=2.2mm] (titlebox)%
{\parbox{0.95\columnwidth}{\noindent\textit{#2}}};\node[fill=white] (W) at ([xshift=#3]titlebox.south) {\small #1};%
\end{tikzpicture}}

\newcounter{recommendationno}
\newcommand{\recommendation}[1]{\refstepcounter{recommendationno}Recommendation \the\numexpr\value{recommendationno}~(\factor{#1})}
\newcounter{observationno}
\newcommand{\observation}[1]{\refstepcounter{observationno}Observation \the\numexpr\value{observationno}~(#1)}


\setcopyright{acmcopyright}
\copyrightyear{2020}
\acmYear{2020}
\acmDOI{}

\setcopyright{none}

\begin{document}
\title{Feature Defect Prediction}
\renewcommand{\shorttitle}{}

\author{Stefan Strüder}
\affiliation{University of Koblenz-Landau, Germany}

\author{Daniel Strüber}
\affiliation{Radboud University Nijmegen, Netherlands}

\author{Thorsten Berger}
\affiliation{Chalmers $|$ University of Gothenburg, Sweden}

\renewcommand{\shortauthors}{}


\begin{abstract}
Software errors are a major nuisance in software development and can lead not only to damage of reputation but also to considerable financial losses for companies. For this reason, numerous techniques for detecting and predicting errors have been developed over the past decade, which are largely based on machine learning methods. The usual approach of these techniques is to predict errors at file level. For some years now, however, the popularity of feature-based software development has been increasing - a paradigm that relies on function increments of a software system (features) and thus ensures a wide variability of the software product. A common implementation technique for features is based on annotations with preprocessor instructions, such as \texttt{\#IFDEF} and \texttt{\#IFNDEF}, whose code is spread over several files of the software's source code files ("code scattering"). A bug in such a feature code can have far-reaching consequences for the functionality of the entire software. If a part of the feature code contains errors, the entire function of the feature becomes faulty and may lead to the failure of the entire functionality of the Software (features are "cross-cutting"). This problem is the subject of this thesis. A prediction technique for faulty features is developed, which is based on methods of machine learning. The evaluation of eight classifiers, each based on an individual classification algorithm, shows that the feature-based data set created for this thesis allows an accuracy of up to 92\% for the prediction of faulty or error-free features. It is also shown how the feature orientation aspect was incorporated into the creation of the dataset and what results were achieved compared to the traditional file-based methodology.

\end{abstract}

\keywords{}

\maketitle

% !TEX root = ../thesis.tex

\chapter{Introduction}

In `The Bible' \cite{Juerjens2005SSD}, Jan Jürjens defines an extension to the UML modeling language that enables modeling of security features in UML diagrams.

\paragraph{Suggestion:}
To make version control more easy, try to put each sentence on a single line.
Use your text editors wrapping option to wrap long sentences (i.e. lines) at the window borders.

\paragraph{Suggestion:}
You may optionally end each chapter by a \verb+\cleardoublepage+ command.
This flushes all floating objects (figures, tables, etc.) and makes chapters start on right pages in twoside mode.
You can safely use \verb+\cleardoublepage+ also in single sided printing, it behaves the same as \verb+\clearpage+.

\section{Some text}

\subsection{First part}

\lipsum[1-1]

\subsection{Second part}

\lipsum[2-3]

\section{Some more text}

\lipsum[4-7]

\cleardoublepage


\input{relatedwork}

% !TEX root = ../thesis.tex

\chapter{Hintergrund}
\label{background}

Dieses Kapitel dient zur Einführung in die dieser Arbeit zugrundeliegenden Themen und hat das Ziel, Basiswissen für den weiteren Verlauf der Ausarbeitung aufzubauen. Dazu wird zunächst die featurebasierte Softwareentwicklung erläutert, ehe dann der Themenbereich des Machine Learnings vorgestellt wird. Dazu werden die Klassifikation und die Fehlervorhersage mittels Machine Learning erläutert.
\\
\hrule

\section{Featurebasierte Softwareentwicklung}
\label{feat-develop}

Das zentrale Konzept hinter der featurebasierten Softwareentwicklung stellen sogenannte Soft-ware-Produktlinien dar. Wie bereits in der Einleitung erwähnt wurde, beschreiben Software-Produktlinien eine Menge von ähnlichen Softwareprodukten, welche eine gemeinsame Menge von Features sowie eine gemeinsame Codebasis besitzen und sich durch die Auswahl der verwendeten Features unterscheiden, sodass eine breite Variabilität innerhalb einer Produktlinie entstehen kann \cite{Apel2013,Thuem2014}.

Der zentrale Prozess der Generierung einer Software-Produktlinie ist in \autoref{fig:spl} dargestellt. Aufgeteilt wird dieser Prozess in das \glqq Domain Engineering\grqq{} und das \glqq Application Engineering\grqq{}. Im Rahmen des \glqq Domain Engineerings\grqq{} wird ein sogenanntes Variabilitätsmodell (Variability Model) erstellt, welches die wählbaren Features und Constraints für mögliche Selektionen beschreibt \cite{Apel2013}. Gängige Implementationstechniken für Features reichen von einfachen Lösungen durch Annotationen, basierend auf Laufzeitparametern oder Präprozessor-Anweisungen, bis hin zu verfeinerten Lösungen, basierend auf erweiterten Programmiermethoden, wie zum Beispiel Aspektorientierung. In einigen dieser Implementierungstechniken wird jedes Feature als wiederverwendbares \glqq Domain Artifact\grqq{} modelliert und gekapselt. Diese können im Prozess des \glqq Application Engineerings\grqq{} in Form einer Konfiguration zusammen mit weiteren Features, im Hinblick auf die gewünschte Funktionalität der Software, ausgewählt werden. Ein Software Generator erzeugt dann die gewünschten Softwareprodukte basierend auf den bereits zuvor genannten Implementationstechniken für Features.

\begin{figure}[ht]
    \centering
    \includegraphics[width=\textwidth]{images/SPL}
    \caption{Generierung von Software-Produktlinien nach \cite{Thuem2014}\label{fig:spl}}
\end{figure}

Die in dieser Arbeit betrachtete Implementierungstechnik von Features basiert auf Anweisungen beziehungsweise Bedingungsdirektiven des C-Präprozessors. Die für diese Arbeit relevanten Direktiven lauten \texttt{\#IFDEF} und \texttt{\#IFNDEF}. Einfache Beispieleinsätze für beide Direktiven sind in \autoref{example1} und \autoref{example2} zu sehen. Sie wurden jeweils aus der wissenschaftlichen Literatur entnommen \cite{Medeiros2018,Preschern2019}. Die Direktive \texttt{\#IFDEF} leitet in \autoref{example1} den Code des Features \texttt{\_\_unix\_\_} ein, welcher mit der Anweisung \texttt{\#ENDIF} endet. Der in den Zeilen 2 bis 6 angegebene Codeteil wird genau dann ausgeführt, wenn das Feature \texttt{\_\_unix\_\_} im Rahmen der Konfiguration des Softwareproduktes definiert beziehungsweise aktiviert ist \cite{Stallmann2016}. In diesem Fall wird die Bedingung der Direktive erfolgreich erfüllt \cite{Stallmann2016}. Sie schlägt fehl, wenn das Feature nicht definiert beziehungsweise nicht aktiviert ist \cite{Stallmann2016}. Die Direktive \texttt{\#IFNDEF} wird für Code verwendet, der ausgeführt werden soll, wenn ein Feature nicht definiert ist. Im Falle des Beispiels in \autoref{example2} wird der in Zeile 3 angedeutete Code nur ausgeführt, wenn \texttt{NO\_XMALLOC} nicht aktiviert wurde.
Es besteht zudem die Möglichkeit, Features bzw. ihren Code zu verschachteln. Ein Beispiel dafür ist in \autoref{example3} angegeben. Es ist zu erkennen, dass sich der Code von \texttt{FEAT\_MZSCHEME} innerhalb des bedingten Codes von \texttt{USE\_XSMP} befindet. Der in Zeile 5 angedeutete Code kann somit nur ausgeführt werden, wenn \texttt{USE\_XSMP} aktiviert ist. Im Fall von Verschachtelung beendet ein \texttt{\#ENDIF} immer das nächstgelegene \texttt{\#IFDEF} oder \texttt{\#IFNDEF} \cite{Stallmann2016}. Es besteht zudem die Möglichkeit, Direktiven mittels \glqq und\grqq{} (\texttt{\&\&}, \texttt{and}) oder \glqq oder\grqq{} (\texttt{||}, \texttt{or}) zu erweiterten Bedingungen zu verknüpfen, die zudem Negation in Form des \texttt{!}-Operators (anstelle von \texttt{\#IFNDEF}) enthalten können \cite{Stallmann2016,Queiroz2015}. Dargestellt ist dies in \autoref{example4}.

\noindent\begin{minipage}{.45\textwidth}
\begin{lstlisting}[caption=Beispieleinsatz von \texttt{\#IFDEF} nach \cite{Preschern2019},frame=tlrb,language=C, label=example1]{example1}
#IFDEF __unix__
	#include "directorySelection.h"
	#include "directoryNames.h"
	void getDirectoryName(char* dirname) {
		getHomeDirectory(dirname);
	}
#ENDIF
\end{lstlisting}
\end{minipage}\hfill
\begin{minipage}{.45\textwidth}
\begin{lstlisting}[caption=Beispieleinsatz von \texttt{\#IFNDEF} nach \cite{Medeiros2018},frame=tlrb,language=C, label=example2]{example2}
int test = 1;
#IFNDEF NO_XMALLOC
	test = memory != NULL;
#ENDIF
if (test){
 // Lines of code here..
 } 
\end{lstlisting}
\end{minipage}

\noindent\begin{minipage}{.45\textwidth}
\begin{lstlisting}[caption=Beispiel eines verschachtelten Einsatzes von \texttt{\#IFDEF} nach \cite{Medeiros2018} ,frame=tlrb,language=C, label=example3, firstnumber=1]{example3}
bool time = msec > 0;
#IFDEF USE_XSMP
	time = time && xsmp_icefd != -1;
	#IFDEF FEAT_MZSCHEME
 		time = time || p_mzq > 0;
	#ENDIF
#ENDIF
if (time)
 gettime(&start_tv);
\end{lstlisting}
\end{minipage}\hfill
\begin{minipage}{.45\textwidth}
\begin{lstlisting}[caption=Beispiele von erweiterten Bedingungen nach \cite{Queiroz2015},frame=tlrb,language=C, label=example4, firstnumber=1]{example4}
#IFDEF FEATURE_A && FEATURE_B
	(...)
#ENDIF
(...)
#IFDEF !FEATURE_A && FEATURE_C
	(...)
#ENDIF
\end{lstlisting}
\end{minipage}

Die in den Listings gezeigten Beispiele zeigen jeweils nur den Featurecode in einer Methode beziehungsweise in einer Datei. Fragmente des Featurecodes erstrecken sich jedoch nicht nur möglicherweise mehrfach über eine Datei, sondern über mehrere Dateien - der Featurecode ist somit verstreut (englisch: code scattering), um eine Funktionalität des Features in der Gesamtheit der Software zu ermöglichen. Ein Defekt innerhalb eines Fragmentes des Featurecodes kann allerdings dazu führen, dass die gesamte Funktionalität des Features beeinträchtigt oder unterbunden wird, da der Fehler übergreifend wirkt (englisch: cross-cutting). Ebenfalls kann ein solcher Fehler dazu führen, dass die Funktionalität des gesamten Sourcecodes beeinträchtigt wird.

\section{Machine-Learning-Klassifikation}
\label{classification}

Das Themengebiet des Machine Learnings (ML) ist in zwei Teilgebiete unterteilt - das unüberwachte ML (englisch: unsupervised ML) und das überwachte ML (englisch: supervised ML). Die Methoden in diesen Teilgebieten verfolgen unterschiedliche Ziele. Im Rahmen des unüberwachten ML werden Prozesse durchgeführt, welche dazu dienen, die Struktur einer unbekannten Eingabemenge an Daten zu erlernen und anschließend zu repräsentieren \cite{Sammut2017}. Eine gängige Anwendung des unüberwachten ML ist das Clustering. Das überwachte ML beschreibt wiederum einen Prozess, welcher beabsichtigt, Vorhersagen über unbekannte Eingabedaten auf Basis des Trainings einer Abbildungsfunktion zu treffen \cite{Sammut2017}. Die Attribute \glqq unüberwacht\grqq{} und \glqq überwacht\grqq{} erhalten die Methoden aufgrund ihrer Art des Lernens beziehungsweise des Trainings. In der Anwendung des unüberwachten ML werden die Eingabedaten erfasst, gegebenenfalls vorverarbeitet, um dann auf deren Basis ein Modell zu erlernen, welches die Darstellung beziehungsweise Repräsentation der Eingabedaten bestimmt \cite{Alpaydin2010}. Auf der anderen Seite wird unter Anwendung des überwachten ML ein Modell auf Basis eines sogenannten \glqq gelabelten\grqq{} (beschrifteten) Datensatzes durch Merkmalsextraktion in Form von Attributen und dem Training auf der Grundlage der extrahierten Merkmale erstellt \cite{Alpaydin2010}. Der Datensatz, welcher zum Training verwendet wird, wird im gängigen Sprachgebrauch des Machine Learning Datenset (englisch: dataset) genannt. Das aus dem Training resultierende Modell wird Klassifikator (englisch: classifier) genannt. Gängige Anwendungen des überwachten ML sind Regression und Klassifikation. In dieser Arbeit kommt die Klassifikation als Anwendung des überwachten ML zum Einsatz. Der grundlegende Prozess der Machine-Learning-Klassifikation ist in \autoref{fig:ml} anhand eines Beispiels dargestellt.

\begin{figure}[ht]
    \centering
    \captionsetup{justification=centering,margin=2cm}
    \includegraphics[width=\textwidth]{images/ML}
    \caption{Allgemeiner Prozess des überwachten Machine Learnings dargestellt anhand eines Beispiels (vereinfacht)}\label{fig:ml}
\end{figure}

Die Abbildung zeigt den Prozess des überwachten Machine Learnings anhand des Beispiels des Trainings eines Klassifikators zur Erkennung beziehungsweise Vorhersage von geometrischen Formen. Der Prozess beginnt mit den \glqq gelabelten\grqq{} Eingabedaten (A). Die Werte der Label (kategorial oder numerisch) stellen dabei die zu vorhersagende Zielklasse dar. In diesem Falle bilden die Namen der geometrischen Formen die Label als kategorischen Wert. Die Rohdaten der Eingabemenge bestehen aus den geometrischen Formen selbst. Beide Datenmengen bilden das Datenset. Um nun einen Klassifikator trainieren zu können, müssen Merkmale der Eingangsdaten ausgewählt werden, anhand derer diese identifiziert werden können (B). Diese zu identifizierenden Charakteristika der Daten werden Attribute genannt. Diese Attribute können bereits vor dem Training festgelegt werden oder automatisiert extrahiert werden. Im vorliegenden Fall wurde die Metrik \glqq Anzahl der Ecken der geometrischen Formen\grqq{} als Attribut zum Training ausgewählt. Das Ergebnis ist der fertig trainierte Klassifikator, welcher das antrainierte Wissen auf neue Daten abbilden kann (C). Ein Teil des Datensets wird in der Regel verwendet, um den Klassifikator nach dessen Erstellung zu testen (D). Die in der Regel verwendeten Verhältnisse (englisch: Split-Ratio) zwischen Trainings- und Testdaten betragen $80:20$ (basierend auf dem Paretoprinzip) oder $75:25$ (zum Beispiel \cite{Queiroz2016}). Diese Testdaten werden dem Klassifikator als Eingabemenge zur Klassifikation ohne Label zur Verfügung gestellt. Die Label sollten jedoch nicht verworfen werden, da sie als Vergleichsgrundlage für die Vorhersageperformanz des Klassifikators dienen. Sie bilden die sogenannte \glqq Ground Truth\grqq{} (deutsch: Grundwahrheit). Dazu werden die vom Klassifikator vorhergesagten Label mit denen der Ground Truth verglichen. Sollte dieser Vergleich ergeben, dass die Label große Abweichungen zeigen, so kann der Klassifikator erneut mit anderen Attributen oder einer veränderten Split-Ratio trainiert werden. Erfüllt der Klassifikator die Anforderungen an die Performanz der Vorhersagen, so ist dieser bereit, Vorhersagen auf Basis neuer Eingabedaten zu treffen (E). Dazu müssen von den neuen Daten die Attribute ermittelt werden. Auf Basis dieser trifft der Klassifikator die Vorhersage und liefert als Ausgabe das Label des Wertes der Zielklasse. Im Voraus des Testens mit den Testdaten (D) werden in manchen Fällen zudem sogenannte Validierungsdaten verwendet. Dabei handelt es sich um eine eigenständige Teilmenge der Trainingsdaten, welche verwendet wird, um die Klassifikatoren nach jedem Training zu evaluieren, um die Auswahl der Attribute hinsichtlich der Performanz auf Eignung zu prüfen \cite{Sammut2017}. Die Anwendung der Testdaten erfolgt dann im Anschluss.

Der in \autoref{fig:ml} dargestellte Klassifikator stellt einen multinomiellen oder multi-class Klassifikator dar, da er zu drei oder mehr Werten der Zielklasse zuordnen kann \cite{Sammut2017}. Für viele praktische Anwendungen genügt jedoch ein binärer Klassifikator, welcher Vorhersagen zu zwei Werten der Zielklasse trifft. Dies trifft auch auf die Klassifikatoren dieser Arbeit zu.

Nachfolgend wird eine Auswahl an Klassifikationsalgorithmen vorgestellt. Diese zählen zu den meist verwendeten Algorithmen und finden auch in dieser Arbeit ihre Anwendung.

\label{algorithms}
\textbf{Decision Trees\medskip}\\
Decision Trees (deutsch: Entscheidungsbäume) zählen zu den meistverwendeten Klassifikatoren im Bereich des supervised Machine Learnings. Studien belegen, dass sie hinsichtlich der Verwendung im Kontext der Fehlererkennung die häufigste Anwendung finden \cite{Son2019}. Decision Trees sind gerichtete und verwurzelte Bäume, die als rekursive Partition der Eingabemenge des Datensets aufgebaut werden \cite{Rokach2005}. Den Ursprung des Baumes bildet die Wurzel, welche keine eingehenden Kanten besitzt - alle weiteren Knoten besitzen jedoch eine eingehende Kante \cite{Rokach2005}. Diese Knoten teilen wiederum die Eingabemenge anhand einer vorgegebenen Funktion in zwei oder mehr Unterräume der Menge auf \cite{Rokach2005}. Meist geschieht dies anhand eines Attributs, sodass die Eingabemenge anhand der Werte des einzelnen Attributs geteilt wird \cite{Rokach2005}. Die Blätter des Baumes bilden die Zielklassen ab. Eine Klassifizierung kann folglich durchgeführt werden, indem man von der Wurzel bis zu einem Blatt den Kanten anhand der entsprechenden Werte der Eingangsmenge folgt. Es existieren verschiedene Algorithmen zur Erstellung von Decision Trees. Bekannte Stellvertreter dieser sind ID3, C4.5 (J48) und CART \cite{Rokach2005}. Der grundlegende Aufbau eines Decision Trees ist in \autoref{fig:dt} dargestellt.

\begin{figure}[ht]
    \centering
    \includegraphics[width=0.5\textwidth]{images/DT}
    \caption{Grundsätzlicher Aufbau eines Decision Trees\label{fig:dt}}
\end{figure}

Eine Besonderheit von Decision Trees stellen sogenannte Random Forests dar. Diese beschreiben eine Menge von Klassifikatoren, bei der mehrere einzelne Decision Trees gleichzeitig erzeugt werden und deren Ergebnisse anschließend aggregiert werden \cite{Alam2013}. Dazu erhält jeder Decision Tree eine Teilmenge der Eingabemenge des Datensets \cite{Alam2013}. Random Forests eigenen sich besonders zur Anwendung, wenn viele Attribute im Datenset vorhanden sind \cite{Alam2013}.

\textbf{k-Nearest-Neighbors\medskip}\\
Ein k-Nearest-Neighbor-Klassifikator (deutsch: k-nächste-Nachbarn) basiert auf zwei Konzepten \cite{Zhang2016}. Das erste Konzept basiert auf der Abstandsmessung zwischen den Werten der zu klassifizierenden Datenmenge und den Werten der Attribute des Datensets \cite{Zhang2016}. Die Abstandmessung erfolgt in der Regel durch die Berechnung der Euklidischen Distanz $D(p,q)$:

\[D(p,q) = \sqrt{\sum_1^n(p_{n}-q_{n})^{2}}\] 

Die Anzahl der Attribute wird durch den Parameter $n$ wiedergegeben, $p$ und $q$ repräsentieren jeweils die Werte der zu klassifizierenden Datenmenge und die Werte der Attribute des Datensets. Das zweite Konzept bildet der Parameter $k$, der angibt, wie viele nächste Nachbarn zum Vergleich der zuvor berechneten Abstände in Betracht gezogen werden \cite{Zhang2016}. Bei einem $k > 1$ wird diejenige Zielklasse gewählt, deren Auftreten innerhalb der nächsten Nachbarn überwiegt.

\textbf{Künstliche neuronale Netze\medskip}\\
Künstliche neuronale Netze (KNN, englisch: Artificial Neural Networks) verwenden nicht-lineare Funktionen zur schrittweisen Erzeugung von Beziehungen zwischen der Eingabemenge und den Zielklassen durch einen Lernprozess \cite{Linder2004}. Sie sind angelehnt an die Funktionsweise von biologischen Nervensystemen und bestehen aus einer Vielzahl von verbundenen Berechnungsknoten, den Neuronen \cite{OShea2015}. Der grundsätzliche Aufbau eines künstlichen neuronalen Netzes kann in \autoref{fig:ann} eingesehen werden. Der Lernprozess besteht aus zwei Phasen - einer Trainingphase und einer Recall-Phase \cite{Linder2004}. In der Trainingsphase werden die Eingabedaten, meist als multidimensionaler Vektor, in den Input-Layer geladen und anschließend an die Hidden-Layer verteilt \cite{OShea2015}. In den Hidden-Layers werden dann Entscheidungen anhand der Beziehungen zwischen den Eingabedaten und Zielklassen sowie die den Verbindungen zuvor zugewiesenen Gewichtsfaktoren getroffen \cite{Linder2004,OShea2015}. Im Rahmen der Recall-Phase wird die Vorhersage basierend auf der zu klassifizierenden Datenmenge anhand der zuvor getroffenen Entscheidungen der Hidden-Layers getroffen und an die jeweiligen Output-Layer, welche die Werte der Zielklasse repräsentieren, weitergeleitet \cite{Linder2004}. 

\begin{figure}[ht]
    \centering
    \includegraphics[width=0.5\textwidth]{images/ANN}
    \caption{Grundsätzlicher Aufbau eines KNN mit drei Input-Layer-Neuronen, fünf Hidden-Layer-Neuronen und zwei Output-Layer-Neuronen\label{fig:ann}}
\end{figure}

\textbf{Logistische Regression\medskip}\\
Logistische-Regressions-Klassifikatoren (englisch: Logistic Regression) basieren auf dem mathematischen Konzept des Logits, welcher den natürlichen Logarithmus eines Chancenverhältnisses beschreibt \cite{Peng2002}. Seine Formel lautet:

\[logit(Y) = ln(\frac{\pi}{1-\pi})\]

$Y$ beschreibt dabei die zu klassifizierende Datenmenge, wohingegen $\pi$ die Verhältnisse der Wahrscheinlichkeiten der Werte der Attribute der Eingabemenge bezeichnet. Am besten geeignet ist dieser Klassifikator für eine Kombination aus kategorialen oder numerischen Eingabedaten und kategorischen Zielklassen \cite{Peng2002}.

\textbf{Na\"{\i}ve Bayes\medskip}\\
Na\"{\i}ve-Bayes-Klassifikatoren zählen zu den linearen Klassifikatoren und basieren auf dem Satz von Bayes. Die Bezeichnung \glqq naiv\grqq{} erhält der Klassifikator durch die Annahme, dass die Attribute der Eingabemenge unabhängig voneinander sind \cite{Raschka2014}. Diese Annahme wird zwar in der realen Verwendung des Klassifikators häufig verletzt, dennoch erzielt er in der Regel eine hohe Performanz \cite{Raschka2014}. Der Klassifikator gilt als effizient, robust, schnell und einfach implementierbar \cite{Raschka2014}. Die zur Durchführung einer Klassifikation mittels Na\"{\i}ve Bayes benötigte Formel nach Thomas Bayes ist in \autoref{fig:nb} samt Erläuterung der einzelnen Faktoren aufgeführt.

\begin{figure}[ht]
    \centering
    \includegraphics[width=0.6\textwidth]{images/NB}
    \caption{Satz von Bayes als Grundlage des Na\"{\i}ve-Bayes-Klassifikators\label{fig:nb}}
\end{figure}

Es existiert zudem eine Mehrzahl an Varianten des Na\"{\i}ve-Bayes-Klassifikators, die verschiedene Annahmen über die Verteilung der Attribute der Eingabemenge machen. Beispiele dafür sind der Gaußsche-Na\"{\i}ve-Bayes (normalverteilte Attribute), der multinomiale Na\"{\i}ve-Bayes (multinomiale Verteilung der Attribute) sowie der Bernoulli-Na\"{\i}ve-Bayes (unabhängige binäre Attribute).

\textbf{Stochastic Gradient Descent\medskip}\\
Ein Stochastic-Gradient-Descent-Klassifikator basiert auf dem Gradientenverfahren (englisch: Gradient Descent), welches das Ziel hat, mittels einer Kostenfunktion $f$ zu einem gegebenen $x$ das minimale $y$ zu finden \cite{Srinivasan2019}. Im Falle der Klassifikation mittels Machine Learning bedeutet dies, dass die Funktion $f$ auf Basis der Trainingsdaten erzeugt wird, die wiederum die Attribute der Daten auf die Werte der Zielklasse überträgt \cite{Diab2019}. Eine festgelegte Kostenfunktion, versucht dann auf Basis der Trainingsdaten ($x$) die minimale Fehlerquote für die Vorhersagen ($y$) anhand von verschiedenen Koeffizientenwerten zu ermitteln \cite{Diab2019}.

\textbf{Support Vector Machines\medskip}\\
Support Vector Machines verfolgen das Ziel, eine sogenannte \glqq Hyperplane\grqq{} in einem $n$-dimen-sionalen Raum ($n$ = Anzahl der Attribute der Eingabemenge) zu finden, welche die Datenpunkte der Eingabemenge eindeutig klassifizieren kann \cite{Gandhi2018}. Die Hyperplane beschreibt eine Trennlinie beziehungsweise Trennfläche, mit deren Hilfe die Daten der zu klassifizierenden Menge den Zielklassen zugeordnet werden können \cite{Luber2019}. Dabei gilt es, dass die Trennflächen, welche die Eingangsmenge anhand der Attribute in verschiedene Trennungsebenen unterteilen, einen möglichst großen Abstand ohne Datenpunkte voneinander haben \cite{Luber2019}. Dies funktioniert sowohl für linear als auch nicht-lineare trennbare Mengen.

\section{Fehlervorhersage mittels Machine Learning}

Der Hintergrund der Fehlervorhersage mittels Machine Learning basiert auf dem zuvor vorgestellten Konzept des überwachten Machine Learnings. Als Grundlage dienen dabei meist Daten, die aus Software-Repositories entnommen beziehungsweise extrahiert werden. Viele Studien und wissenschaftliche Arbeiten setzen jedoch auch auf vorgefertigte Datensets, wie zum Beispiel von der NASA oder von Eclipse \cite{Son2019}. Die zugehörigen Label der Datensets lauten in der Regel \glqq fehlerfrei\grqq{} und \glqq defekt\grqq{} und können auf verschiedene Weisen ermittelt werden. Eine gängige Methode ist die Identifizierung von korrektiven und fehlereinführenden Commits als Entscheidungsgrundlage für das Label. Die üblichen Vorgehensweisen setzen auf die Einbindung von Bugtracking-Systemen oder die Analyse von Commit-Nachrichten zur Identifizierung der korrektiven Commits \cite{Queiroz2016,Zimmermann2007}. Fehlereinführende Commits können anschließend unter Verwendung von Git-Kommandos oder durch die Anwendung des sogenannten SZZ-Algorithmus ermittelt werden. Dieser Algorithmus wird in dieser Arbeit verwendet und in \hyperref[szz-def]{Abschnitt 3.2} erläutert. Auf Basis dieser Daten werden die Attribute bestimmt. Dabei handelt es sich in der Regel um Metriken, die entweder die Charakteristika des Sourcecodes (Codemetriken) oder Aktivitäten und Prozesse im Bezug auf Software-Repositories (Prozessmetriken) beschreiben \cite{Son2019,Rahman2013}. Mithilfe dieser Attribute werden die Klassifikatoren trainiert. Eine Studie, welche 156 wissenschaftliche Arbeiten zum Thema der Fehlervorhersage mittels Machine Learning analysierte, ergab, dass besonders Entscheidungsbaum-basierte, Bayessche Verfahren, Regression und künstliche neuronale Netze als Klassifikationsalgorithmen zur Anwendung kommen \cite{Son2019}. Diese Algorithmen wurden unter anderem auch in dieser Arbeit verwendet. Erläuterungen können in \hyperref[algorithms]{Abschnitt 4.1} gefunden werden. Die fertig trainierten Klassifikatoren können dann auf Basis neuer Daten Vorhersagen zum Zustand einer Software treffen. Die Vorhersagen beruhen in der Regel auf defekten Dateien im Kontext von Commits oder Releases.

Als Beispiel für zwei konkrete Anwendungen von Machine Learning gestützter Fehlervorhersage werden im Folgenden zwei wissenschaftliche Arbeiten vorgestellt. Die erste Arbeit \glqq Comparative Analysis of the Efficiency of Change Metrics and Static Code Attributes for Defect Prediction\grqq{} von Moser et al. \cite{Moser2008} stellt eine Methodik zur dateibasierten Fehlervorhersage vor. Die zweite Arbeit \glqq Towards Predicting Feature Defects in Software Product Lines\grqq{} von Queiroz et al. \cite{Queiroz2016} knüpft an den zuvor vorgestellten Ansatz der Software-Features an. Beide Literaturquellen werden im weiteren Verlauf dieser Arbeit eine Rolle spielen, die im jeweiligen Abschnitt erläutert werden.

\subsection*{Dateibasierte Fehlervorhersage}
\label{moser}

Das Beispiel zur dateibasierten Fehlervorhersage stammt aus einer wissenschaftlichen Arbeit von Moser et al. \cite{Moser2008}. Sie widmet sich einer vergleichenden Analyse von zwei verschiedenen Mengen von Metriken zur dateibasierten Fehlervorhersage mittels Methoden des Machine Learnings. Die Zuordnung erfolgt in \glqq defekt\grqq{} und \glqq defektfrei\grqq. Als Datenbasis dient ein vorgefertigtes Datenset von Eclipse. Auf Basis dieses Datensets wurden \glqq produktbasierte\grqq{} Metriken (Codemetriken) und Prozessmetriken berechnet. Zur Anwendung kamen die Klassifikationsalgorithmen logistische Regression, Na\"{\i}ve Bayes und Entscheidungsbäume. Die Prozessmetriken stellen eine Besonderheit dieser Arbeit dar, da sie in dieser Arbeit zum ersten Mal näher betrachtet und auf ihre Eignung als Attribute zum Training der Klassifikatoren erörtert wurden. Die Metriken berechneten unter anderem die Anzahl der Änderungen an einer Datei, die Anzahl der Autoren einer Datei, die Anzahl der hinzugefügten oder entfernten Zeilen einer Datei oder das Alter einer Datei. Das Resultat der Arbeit lautet, dass Prozessmetriken effektiver zur Fehlervorhersage genutzt werden können als Codemetriken. Im Folgenden werden die 17 Prozessmetriken samt ihrer Beschreibungen und Abkürzungen (für diese Arbeit) vorgestellt.

\label{moser-metrics}
\begin{multicols}{2}
\begin{itemize}
\setlength{\itemsep}{-2pt}
\item REVISIONS (REVI)\\Anzahl der Revisionen (Bearbeitungen) der Datei.
\item REFACTORINGS (REFA)\\Anzahl der Fälle, in denen die Datei in einem Refactoring involviert war. Basierend auf Analyse der Commit-Nachricht auf das Vorhandensein des Begriffs "refactor".
\item BUGFIXES (BUGF)\\Anzahl der Fälle, in denen die Datei in einer Fehlerbehebung involviert war.
\item AUTHORS (AUTH)\\Anzahl der verschiedenen Autoren, die die Datei in das Repository eingecheckt haben.
\item LOC\_ADDED (ADDL)\\Summe der zur Datei hinzugefügten Codezeilen über alle Revisionen.
\item MAX\_LOC\_ADDED (ADDM)\\maximale Anzahl von Codezeilen, die für alle Revisionen hinzugefügt wurden.
\item AVE\_LOC\_ADDED (ADDA)\\durchschnittlich hinzugefügte Codezeilen pro Revision.
\item LOC\_DELETED (REML)\\Summe der von der Datei entfernten Codezeilen über alle Revisionen.
\item MAX\_LOC\_DELETED (REMM)\\maximale Anzahl von Codezeilen, die für alle Revisionen entfernt wurden.
\item AVE\_LOC\_DELETED (REMA)\\durchschnittlich entfernte Codezeilen pro Revision.
\item CODECHURN (CCHN)\\Summe von (hinzugefügte Codezeilen - entfernte Codezeilen) über alle Revisionen.
\item MAX\_CODECHURN (CCHM)\\maximaler CODECHURN für alle Revisionen.
\item AVE\_CODECHURN (CCHA)\\durchschnittlicher CODECHURN pro Revision.
\item MAX\_CHANGESET (MAXC)\\maximale Anzahl von Dateien, die gemeinsam committed wurden.
\item AVE\_CHANGESET (AVGC)\\durchschnittliche Anzahl von Dateien, die gemeinsam committed wurden.
\item AGE (AAGE)\\Alter der Datei in Wochen (rückwärts zählend bis zu einem bestimmten Release).
\item WEIGHTED\_AGE (WAGE)\\$\text{Weighted Age} = \frac{\sum_{i=1}^N Age(i)*LOC\_ADDED(i)}{\sum_{i=1}^N LOC\_ADDED(i)}$
\end{itemize}
\end{multicols}

Im Rahmen der Evaluation (\hyperref[classic-eval]{Abschnitt 5.3}) dienen diese dateibasierten Metriken als Grundlage für den Vergleich, ob die Metriken des für diese Arbeit erstellten featurebasierten Datensets einen Einfluss auf die Ergebnisse der Fehlervorhersage auf Dateiebene hervorrufen. 

\subsection*{Featurebasierte Fehlervorhersage}

Das Beispiel zur featurebasierten Fehlervorhersage stammt aus einer wissenschaftlichen Arbeit von Queiroz et al. \cite{Queiroz2016}. Bei dieser Fallstudie handelt es sich um die erste und bisher einzige Arbeit über Fehlervorhersage mit Bezug zu Software-Features. Sie stellt somit für diese Masterarbeit eine bedeutende literarische Grundlage dar. Der Ablauf des von Queiroz et al. angewandten Prozesses zur Erstellung eines featurebasierten Datensets und dessen Anwendung zum Training von Klassifikatoren orientiert sich am zuvor vorgestellten allgemeinen Prozess des überwachten Machine Learnings.

Die Erläuterung des Beispiels erfolgt anhand von zwei Abbildungen, welche den in der Arbeit von Queiroz et al. vorgestellten Prozess in zwei Teilen visualisieren. Der erste Teil ist in \autoref{fig:ml1} dargestellt.

\begin{figure}[ht]
    \centering
    \captionsetup{justification=centering}
    \includegraphics[width=\textwidth]{images/ML1}
    \caption{Teil 1: Featurebasierter Prozess des überwachten Machine Learnings nach \cite{Queiroz2016}}\label{fig:ml1}
\end{figure}

\subsubsection*{Datenset}

Die Datenbasis des Datensets bilden historische Commits des UNIX-Toolkits BusyBox\footnote{\url{https://busybox.net/}}, dessen Quellcode frei verfügbar in einem Git-Repository\footnote{\url{https://git.busybox.net/busybox/}} eingesehen und von dort geklont werden kann. Diese Commits wurden wiederum ihren entsprechenden Releases zugeordnet, welche auf der vergebenen Tag-Struktur des Repositories beruhen. Ferner wurden aus den Diffs der Commits die dort bearbeiteten Features extrahiert und anschließend zusammen mit den Release-Informationen in einer MySQL-Datenbank gespeichert. Zusätzlich enthält jeder Datenbankeintrag aggregierte Werte von fünf auf das Feature und den Release bezogenen Prozessmetriken (Erläuterung folgt) sowie das binäre Label, ob ein Feature in einem Release fehlerhaft oder fehlerfrei war. Ein Feature gilt in einem Release als fehlerhaft, sofern in einem Commit des darauffolgenden Releases ein fehlerbehebender Commit bezüglich des Features festgestellt werden konnte. Dies geschieht über die Analyse der Commit-Nachrichten. Sofern eine Commit-Nachricht die Begriffe \glqq bug\grqq{} (Fehler), \glqq error\grqq{} (schwerwiegender Fehler), \glqq fail\grqq{} (fehlschlagen) oder \glqq fix\grqq{} (beheben) enthält, werten die Autoren des Papers den Commit als fehlerbehebend. Alternative Methoden zur Durchführung dieser Analyse bestehen aus der Einbindung von Daten aus Bug-Tracking-Systemen, die häufig an Software-Repositories angebunden sind, sowie aus der Anwendung des sogenannten SZZ-Algorithmus, welcher in dieser Arbeit verwendet wurde und in \hyperref[szz-def]{Abschnitt 3.2} erläutert wird \cite{Sliwerski2005,Zimmermann2007}. Wie im Rahmen des überwachten Machine Learning üblich, wird das Datenset in Trainings- und Testdaten in einem Verhältnis von $75:25$ geteilt. 

\subsubsection*{Metriken und Klassifikation}

Die Trainingsdaten werden dann den Klassifikatoren zum Training zur Verfügung gestellt. Als Attribute dienen fünf Prozessmetriken mit spezifischer Betrachtung von Software-Features. Einen Überblick über die Beschreibungen dieser gibt \autoref{tab:metrics-rodrigo}. Diese Metriken werden auch für diese Arbeit im Rahmen der Erstellung des featurebasierten Datensets übernommen. Als Klassifikationsalgorithmen wurden Na\"{\i}ve Bayes, Random Forest und J48-Entscheidungsbäume gewählt.

\begin{table}[ht]
\centering
\caption{Übersicht der von \cite{Queiroz2016} verwendeten Prozessmetriken}
\label{tab:metrics-rodrigo}
\begin{tabular}{|c|l|} 
\hline
\textbf{Metrik}  & \textbf{Beschreibung}  \\ 
\hline
COMM & \begin{tabular}[c]{@{}l@{}}Anzahl der Commits, die in einem Release dem betreffenden \\ Feature gewidmet sind. \end{tabular} \\ 
\hline
ADEV & \begin{tabular}[c]{@{}l@{}}Anzahl der Entwickler, die das betreffende Feature\\in einem Release bearbeitet haben. \end{tabular} \\ 
\hline
DDEV & \begin{tabular}[c]{@{}l@{}}kummulierte Anzahl der Entwickler, die das betreffende Feature\\in einem Release bearbeitet haben. \end{tabular} \\ 
\hline
EXP & \begin{tabular}[c]{@{}l@{}}geometrisches Mittel der \glqq Erfahrung*\grqq{} aller Entwickler, die am \\ betreffenden Feature in einem Release gearbeitet haben. \end{tabular} \\ 
\hline
OEXP & \begin{tabular}[c]{@{}l@{}}\glqq Erfahrung*\grqq{} des Entwicklers, der am meisten zum betreffenden \\ Feature in einem Release beigetragen hat. \end{tabular} \\ 
\hline
\multicolumn{2}{|c|}{\begin{tabular}[c]{@{}c@{}}*Erfahrung ist definiert als Summe der geänderten, gelöschten\\oder hinzugefügten Zeilen im zugehörigen Release. \end{tabular}} \\
\hline
\end{tabular}
\end{table}

\begin{figure}[ht]
    \centering
    \includegraphics[width=0.8\textwidth]{images/ML2}
    \caption{Teil 2: Featurebasierter Prozess des überwachten Machine Learnings nach \cite{Queiroz2016}}\label{fig:ml2}
\end{figure}

\subsubsection*{Test der Klassifikatoren}

Wie in \autoref{fig:ml2} dargestellt ist, wird für jeden Klassifikationsalgorithmus ein Klassifikator erstellt, welcher anschließend getestet und evaluiert wird. Dazu werden die jeweiligen Klassifikatoren auf das Testdatenset angewendet, ohne jedoch die Werte der Zielklassen mit anzugeben. Diese werden im Anschluss an den Klassifikationsvorgang mit den vorhergesagten Werten auf Übereinstimmung verglichen. Anhand dieses Vergleiches können die Genauigkeit sowie weitere Metriken zur Bewertung der Leistung der Klassifikatoren gemessen werden. Eine Übersicht von Evaluationsmetriken kann in \hyperref[eval-metrics]{Abschnitt 5.2.1} gefunden werden.

Die so erstellten Klassifikatoren können dann zur Vorhersage von neuen und unbekannten Daten genutzt werden, um defekte Features eines zukünftigen Releases zu identifizieren. Dazu müssen die fünf zuvor genannten Prozessmetriken dieser Daten berechnet werden. 

\cleardoublepage


% !TEX root = /doc.tex

\section{Methodology}

\subsection{Creation of Dataset}

The data set forms the basis for the training of the Machine Learning Classifiers and is created specifically for this work using commit data from 13 feature-based software projects. The software projects are selected based on previous use in scientific literature \cite{Hunsen2015,Liebig2010,Queiroz2015,Queiroz2016}. However, it was also important that the variability in the source code of the software projects was implemented by means of preprocessor directives and that their git repositories had a clear "revision history" regarding release numbers. Another sufficient criterion was the use of a common language. All software projects originate from the English-speaking world. The software projects used for this work are listed in \autoref{tab:tools} together with their purpose and data sources.

\begin{table}[]
\centering
\caption{Subject systems}
\label{tab:systems}
\begin{tabular}{@{}lll@{}}
\toprule
project           & purpose          & repository \\ \midrule
\textbf{Blender}  & 3D nodeling tool & \cite{blender:online}          \\
\textbf{Busybox}  & UNIX toolkit     & \cite{busybox:online}          \\
\textbf{Emacs}    & text editor      & \cite{emacs:online}          \\
\textbf{GIMP}     & graphics editor  & \cite{gimp:online}          \\
\textbf{Gnumeric} & spreadsheet      & \cite{gnumeric:online}          \\
\textbf{gnuplot}  & plotting tool    & \cite{gnuplot:online}          \\
\textbf{Irssi}    & IRC client       & \cite{irssi:online}          \\
\textbf{libxml2}  & XML parser       & \cite{libxml2:online}          \\
\textbf{lighttpd} & web server       & \cite{lighttpd:online}          \\
\textbf{MPSolve}  & polynom solver   & \cite{mpsolve:online}         \\
\textbf{Parrot}   & virtual machine  & \cite{parrot:online}         \\
\textbf{Vim}      & text editor      & \cite{vim:online}         \\ \bottomrule
\end{tabular}
\end{table}

To retrieve the commit data of the software projects the Python library PyDriller\footnote{\href{https://github.com/ishepard/pydriller}{https://github.com/ishepard/pydriller}} was used \cite{Spadini2018}. This allows easy data extraction from Git repositories to obtain commits, commit messages, developers, diffs, and more (called "metadata" in the following). The URLs to the Git repositories of the software projects were used as input for the specially created Python scripts for receiving the commit metadata. Furthermore, the metadata was divided into commits per release. This was made possible by specifying release tags in the PyDriller code, based on the tag structure of Git repositories. For each modified file within a commit and a release, the following metadata was retrieved using PyDriller:

\begin{itemize}
\item commit hash (unique identifier of a commit)
\item commit author
\item commit message
\item filename
\item lines of code
\item cyclomatic complexity
\item number of added lines
\item number of removed lines
\item diff (changeset)
\end{itemize}

The data obtained in this way was stored in a MySQL database after retrieval. For each software project, a separate table was created in which, in addition to the metadata above, the name of the software project and the release numbers associated with the commits were stored. Each modified file of a commit receives one row of the database tables. The further construction of the data set is divided into several phases of data processing and optimization.

The first phase consists of extracting the features involved in a modified file. This was done by using a Python script to identify the preprocessor statements \texttt{\#IFDEF} and \texttt{\#IFNDEF} in the diffs of the modified files, and then saving the string following the directives as a feature until the end of the line of code. The identification was done using regular expressions. The features identified per file are stored in an additional column in the respective MySQL tables. In case a feature is identified after the \texttt{\#IFNDEF} directive, the feature is stored with a preceding "not". It will be saved as a separate feature, along with its non-negated form. Combinations of features are stored in their identified form. If no feature could be identified, "\texttt{none}" is saved accordingly.

This way of identification has some obstacles. In some C programming paradigms, it is common to include header files in the source code using preprocessor directives, so that they appear to be features. However, these "header features", as they will be referred to later, should be ignored as they do not create variability throughout the source code. In general, these header features are identifiable by their naming in the form of an appended \texttt{\_h\_} to the feature name, such as \texttt{featurename\_h\_}. This appended part allows the header features to be identified and filtered out using regular expressions. It is also possible that "wrong" features can be identified. Examples of this can come from \texttt{\#IFDEFs} used in comments. Such false features were removed in a manual review of the identified features and replaced with "\texttt{none}".

The next phase of processing consists of identifying corrective commits. A common method used for this, and one that was used in this paper, is to analyze commit messages for the presence of certain keywords \cite{Zimmermann2007}. The keywords used are "bug", "bugs", " bugfix", "error", "fail", "fix", "fixed" and "fixes". The analysis was performed using a Python script that checks whether any of the keywords are in the commit message alone or within a combination of words. The identification was limited to the first line of each commit message, since these contain the relevant information of the message. The results were stored in another column ("corrective") of the MySQL tables (true = corrective, false = uncorrective).

The search for corrective commits is followed by an analysis for commits that introduced bugs. A PyDriller implementation of the SZZ algorithm according to Sliwerski, Zimmermann and Zeller was used \cite{Sliwerski2005,Spadini2018}. This algorithm allows to find commits that introduce bugs in locally stored software repositories \cite{Borg2019}. It requires that the corrective commits have already been identified, as they serve as the algorithm's input \cite{Borg2019}.

An overview of the number of corrective and bug-introducing commits and the number of features identified per software project is given in \autoref{tab:tools-values2}.

\begin{table}[t]
\centering
\caption{Number of corrective and bug-introducing commits and number of identified features}
\label{tab:tools-values2}
\begin{tabular}{@{}lrrr@{}}
\toprule
project           & \#corrective & \#bug-introducing & \#features \\ \midrule
\textbf{Blender}  & $7.760$      & $3.776$           & $1.400$    \\
\textbf{Busybox}  & $1.236$      & $802$             & $628$      \\
\textbf{Emacs}    & $4.269$      & $2.532$           & $718$      \\
\textbf{GIMP}     & $1.380$      & $854$             & $204$      \\
\textbf{Gnumeric} & $1.498$      & $1.191$           & $637$      \\
\textbf{gnuplot}  & $854$        & $1.215$           & $558$      \\
\textbf{Irssi}    & $52$         & $22$              & $9$        \\
\textbf{libxml2}  & $324$        & $88$              & $200$      \\
\textbf{lighttpd} & $1.078$      & $929$             & $230$      \\
\textbf{MPSolve}  & $151$        & $211$             & $54$       \\
\textbf{Parrot}   & $3.109$      & $3.072$           & $397$      \\
\textbf{Vim}      & $371$        & $696$             & $1.158$    \\ \bottomrule
\end{tabular}
\end{table}

A real example from the data of the Vim software project, showing the diffs of a corrective (A) and a bug-introducing (B) commit to a feature \texttt{FEAT\_TEXT\_PROP}, is shown in \autoref{fig:bug-example}. The section of the diff shows that the method call \texttt{vim\_memset} has been replaced with different arguments. According to the associated commit message, the original method call caused a "memory access error". This commit was identified as corrective because the commit message contains the keyword "error". The corresponding entry in Vim's main table thus gets the value \texttt{true} in the "corrective" column. Using the SZZ algorithm, specifying the commit hash of the corrective commit, the error-initiating commit (B) of the file concerned. In its portion of the diff, you can see that this commit has put the feature \texttt{FEAT\_TEXT\_PROP} in the file with the incorrect method call. As a result, it is assigned the value \texttt{true} in the "bug\_introducing" column in the main table.

\begin{figure*}[ht]
    \centering
    \includegraphics[width=\textwidth]{Bug_example_real}
    \Description{Real example of a defect.}
    \caption{Real example of a defect with corrective (A) and bug-introducing (B) commit\label{fig:bug-example}}
\end{figure*}


\subsection{Selection of Metrics}

As already mentioned, attributes are used to train the machine learning classifiers. In this scenario, so-called metrics are used as attributes. Metrics are numerical values that quantify properties of a software project. In this case, the metrics are divided into the usual categories code metrics and process metrics and each is calculated using the existing raw data of the main tables \cite{Rahman2013}. Code metrics are used to measure properties of source code, such as "size" or complexity \cite{Rahman2013}. Process metrics are used to measure properties that can be discussed using metadata from software repositories \cite{Rahman2013}. Examples include the number of changes made to a particular file or the number of active developers on a project. For this work, eleven feature-based metrics were calculated, divided into seven process and four code metrics. Five of the process metrics were taken from scientific papers \cite{Rahman2013,Queiroz2016}. The other six metrics were calculated based on the commit metadata obtained from PyDriller. A list of the eleven metrics and their descriptions can be found in \autoref{tab:metrics}. The individual metrics were calculated either directly using SQL queries or combined using SQL queries and calculations by a Python script.

In view of the evaluation of the feature-based dataset that follows in the next section, two additional datasets were created to make the impact of the feature-based metrics on the classifiers' predictions comparable. Both further data sets follow a classical file-based approach as it is common in machine learning based error detection and are based on the methodology developed by Moser et al. \cite{Moser2008}. The 17 process metrics of this scientific work were also adopted and are listed in \autoref{tab:metrics-moser}. A visualization to distinguish the three data sets is shown in \autoref{fig:dataset}.

%\begin{table*}[ht]
%\centering
%\caption{Overview of used feature-based metrics}
%\label{tab:metrics}
%\resizebox{\linewidth}{!}{%
%\begin{tabular}{|>{\hspace{0pt}}p{0.027\linewidth}|>{\hspace{0pt}}p{0.318\linewidth}|>{\hspace{0pt}}p{0.584\linewidth}|>{\centering\arraybackslash\hspace{0pt}}p{0.064\linewidth}|} 
%\cline{2-4}
%\multicolumn{1}{>{\hspace{0pt}}p{0.027\linewidth}|}{}  & \textbf{Metric}                            & \textbf{Description}                                                                                                                                                                                                                                               & \textbf{Source}   \\ 
%\hline
%\multirow{7}{0.027\linewidth}{\hspace{0pt}\rotatebox[origin=c]{90}{Process metrics}\textbf{}} & Number of commits                          & number of commits associated with the feature in a release.                                                                                                                                                                                                        & \cite{Queiroz2016,Rahman2013}              \\ 
%\cline{2-4}
%                                                       & Number of active developers                & number of developers who have edited (changed, deleted or added) \par{}the feature within a release.                                                                                                                                                               & \cite{Queiroz2016,Rahman2013}              \\ 
%\cline{2-4}
%                                                       & Number of distinct developers              & cumultative number of developers who have edited (changed, deleted or added) \par{}the feature within a release.                                                                                                                                                   & \cite{Queiroz2016,Rahman2013}              \\ 
%\cline{2-4}
%                                                       & Experience of all develepoers              & geometric mean of the experience of all developers who have edited \par{}(changed, deleted or added) the feature within a release. Experience is\par{}defined as the sum of the changed, deleted or added \par{}lines in the commits associated with the feature.  & \cite{Queiroz2016,Rahman2013}              \\ 
%\cline{2-4}
%                                                       & Experience of the most involved developers & experience of the developer who has edited (changed, deleted or added) \par{}the feature most often within a release. Experience is defined as the \par{}sum of changed, deleted, or added lines in the commits associated with the \par{}feature.                 & \cite{Queiroz2016,Rahman2013}              \\ 
%\cline{2-4}
%                                                       & Degree of modifications                    & number of edits (change, removal, extension) of the feature within a release.                                                                                                                                                                                      & *                 \\ 
%\cline{2-4}
%                                                       & Scope of modifications                     & number of edited features within a release (feature overlapping value). \par{}Idea: The more features have been edited in a release, \par{}the more error-prone they seem to be.                                                                                   & *                 \\ 
%\hline
%\multirow{4}{0.027\linewidth}{\hspace{0pt}\rotatebox[origin=c]{90}{Code metrics}\textbf{}} & Lines of code                              & average number of lines of code of the files associated with the feature\par{}within a release.                                                                                                                                                                    & *                 \\ 
%\cline{2-4}
%                                                       & Cyclomatic Complexity                      & average cyclomatic complexity of the files associated with the feature\par{}within a release.                                                                                                                                                                      & *                 \\ 
%\cline{2-4}
%                                                       & Number of added lines                      & average number of lines of code added to the files associated with \par{}the feature within a release.                                                                                                                                                             & *                 \\ 
%\cline{2-4}
%                                                       & Number of removed lines                    & average number of lines of code deleted from the files associated \par{}with the feature within a release.                                                                                                                                                         & *                 \\ 
%\hline
%\multicolumn{4}{|>{\centering\arraybackslash\hspace{0pt}}p{0.993\linewidth}|}{\textit{* These values were calculated based on the metadata obtained with PyDriller.}\par{}\textit{Feature-level metrics were calculated based on the metadata of the underlying files.} }                                                                                                                    \\
%\hline
%\end{tabular}
%}
%\end{table*}
%
%\begin{table*}[ht]
%\centering
%\caption{Feature metrics. To apply the metrics to a file, we aggregate the metric values of all features associated to the given file, using the listed aggregation operator.}
%\label{tab:metrics}
%\resizebox{\linewidth}{!}{%
%\begin{tabular}{|l|l|l|l|c|} 
%\cline{2-5}
%\multicolumn{1}{l|}{}        & \textbf{Metric}                            & \textbf{Description}                                                                                                                                                                                                                                                                           & \textbf{Aggregation~operator} & \textbf{Source}   \\ 
%\hline
%\multirow{7}{*}{\rotatebox[origin=c]{90}{Process metrics}\textbf{}} & Number of commits (FCOMM)                         & number of commits associated with the feature in a release.                                                                                                                                                                                                                                    & Mean                          & \cite{Queiroz2016,Rahman2013}              \\ 
%\cline{2-5}
%                             & Number of active developers (FADEV)              & \begin{tabular}[c]{@{}l@{}}number of developers who have edited (changed, deleted or added) \\the feature within a release.\end{tabular}                                                                                                                                                       & Mean                          & \cite{Queiroz2016,Rahman2013}              \\ 
%\cline{2-5}
%                             & Number of distinct developers (FDDEV)              & \begin{tabular}[c]{@{}l@{}}cumultative number of developers who have edited (changed, deleted or added) \\the feature within a release.\end{tabular}                                                                                                                                           & Mean                          & \cite{Queiroz2016,Rahman2013}              \\ 
%\cline{2-5}
%                             & Experience of all develepoers (FEXP)             & \begin{tabular}[c]{@{}l@{}}geometric mean of the experience of all developers who have edited \\(changed, deleted or added) the feature within a release. Experience is\\defined as the sum of the changed, deleted or added \\lines in the commits associated with the feature. \end{tabular} & Mean                          & \cite{Queiroz2016,Rahman2013}              \\ 
%\cline{2-5}
%                             & Experience of the most involved developers (FOEXP) & \begin{tabular}[c]{@{}l@{}}experience of the developer who has edited (changed, deleted or added) \\the feature most often within a release. Experience is defined as the \\sum of changed, deleted, or added lines in the commits associated with the \\feature. \end{tabular}                & Mean                          & \cite{Queiroz2016,Rahman2013}              \\ 
%\cline{2-5}
%                             & Degree of modifications (FMODD)                    & number of edits (change, removal, extension) of the feature within a release.                                                                                                                                                                                                                  & Mean                          & *                 \\ 
%\cline{2-5}
%                             & Scope of modifications (FMODS)                    & \begin{tabular}[c]{@{}l@{}}number of edited features within a release (feature overlapping value). \\Idea: The more features have been edited in a release, \\the more error-prone they seem to be. \end{tabular}                                                                              & Mean                          & *                 \\ 
%\hline
%\multirow{4}{*}{\rotatebox[origin=c]{90}{Code metrics}\textbf{}} & Lines of code (FNLOC)                              & \begin{tabular}[c]{@{}l@{}}average number of lines of code of the files associated with the feature\\within a release. \end{tabular}                                                                                                                                                           & Mean                          & *                 \\ 
%\cline{2-5}
%                             & Cyclomatic Complexity (FCyCO)                     & \begin{tabular}[c]{@{}l@{}}average cyclomatic complexity of the files associated with the feature\\within a release. \end{tabular}                                                                                                                                                             & Mean                          & *                 \\ 
%\cline{2-5}
%                             & Number of added lines (FADDL)                      & \begin{tabular}[c]{@{}l@{}}average number of lines of code added to the files associated with \\the feature within a release. \end{tabular}                                                                                                                                                    & Mean                          & *                 \\ 
%\cline{2-5}
%                             & Number of removed lines (FREML)                   & \begin{tabular}[c]{@{}l@{}}average number of lines of code deleted from the files associated \\with the feature within a release.\end{tabular}                                                                                                                                                 & Mean                          & *                 \\ 
%\hline
%\multicolumn{5}{|c|}{\begin{tabular}[c]{@{}c@{}}\textit{* These values were calculated based on the metadata obtained with PyDriller.}\\\textit{Feature-level metrics were calculated based on the metadata of the underlying files.} \end{tabular}}                                                                                                                                                                           \\
%\hline
%\end{tabular}
%}
%\end{table*}

\begin{table}
\centering
\caption{Process metrics according to \cite{Moser2008}}
\label{tab:metrics-moser}
\resizebox{\linewidth}{!}{%
\begin{tabular}{|l|l|} 
\hline
\textbf{Metric}   & \textbf{Description}                                                      \\ 
\hline
REVISIONS         & Number of revisions of a file.                                            \\ 
\hline
REFACTORINGS      & Number of times a file has been refactored.                               \\ 
\hline
BUGFIXES          & Number of times a file was involved in bug-fixing.                        \\ 
\hline
AUTHORS           & Number of distinct authors that checked a file into the repository.       \\ 
\hline
LOC\_ADDED        & Sum over all revisions of the lines of code added to a file.              \\ 
\hline
MAX\_LOC\_ADDED   & Maximum number of lines of code added for all revisions.                  \\ 
\hline
AVE\_LOC\_ADDED   & Average lines of code added per revision.                                 \\ 
\hline
LOC\_DELETED      & Sum over all revisions of the lines of code deleted from a file.          \\ 
\hline
MAX\_LOC\_DELETED & Maximum number of lines of code deleted for all revisions.                \\ 
\hline
AVE\_LOC\_DELETED & Average lines of code deleted per revision.                               \\ 
\hline
CODECHURN         & Sum of (added lines of code – deleted lines of code) over all revisions.  \\ 
\hline
MAX\_CODECHURN    & Maximum CODECHURN for all revisions.                                      \\ 
\hline
AVE\_CODECHURN    & Average CODECHURN per revision.                                           \\ 
\hline
MAX\_CHANGESET    & Maximum number of files committed together to the repository.             \\ 
\hline
AVE\_CHANGESET    & Average number of files committed together to the repository.             \\ 
\hline
AGE               & Age of a file in weeks (counting backwards from a specific release).      \\ 
\hline
WEIGHTED\_AGE     & $Weighted Age = \frac{\sum_{i=1}^N Age(i)*LOC\_ADDED(i)}{\sum_{i=1}^N LOC\_ADDED(i)}$                                                                    \\
\hline
\end{tabular}
}
\end{table}

\begin{figure*}[ht]
    \centering
    \includegraphics[width=\textwidth]{Dataset}
    \Description{Visualization to distinguish the three data sets.}
    \caption{Visualization to distinguish the three data sets\label{fig:dataset}}
\end{figure*}

The figure shows the ways of creating the feature-based dataset (path A - G) and the file-based data sets according  to \cite{Moser2008}.These are divided into the "simple" file-based dataset (paths A, B, H, I) and the extended file-based dataset (combination of both paths).  The creation of the feature-based dataset has already been explained in detail up to this section. More information about finalizing the datasets will follow as this section progresses.

The creation of the file-based datasets is also based on the raw data of the software project commits obtained with PyDriller and the modified files (A + B) extracted from them. Thus no further processing of the raw data is necessary. They then serve as a basis for the calculation of the 17 metrics (H) according to \cite{Moser2008}. To calculate the time-based metrics \texttt{AAGE} and \texttt{WAGE}, the raw data or the commits listed in the raw data had to be supplemented with their publication date. This was also done with PyDriller. The date of the first commit of a release was chosen as the starting point for calculating the past weeks.

The feature-based and the "simple" file-based dataset (G + I) consist of the calculated feature- (F) and file-related (H) metrics and the labels of the target class. The values of the metrics are calculated for the data of each software project. The resulting tables contain as columns the values of the eleven or 17 metrics and the label (target class) "defective" or "clean" and as rows the features or files aggregated by release.  This means that if a feature or file has been edited multiple times in a release (i.e. it is edited in multiple commits), the average value of the respective metrics within the release is calculated and stored. The procedure for determining the label is performed for each change to a feature or file using the following pattern:

\begin{table}[H]
\centering
\resizebox{\linewidth}{!}{%
\begin{tabular}{>{\hspace{0pt}}p{0.428\linewidth}>{\hspace{0pt}}p{0.046\linewidth}>{\hspace{0pt}}p{0.283\linewidth}>{\hspace{0pt}}p{0.046\linewidth}>{\hspace{0pt}}p{0.187\linewidth}}
bug-introducing       & + & corrective       & = & defective      \\
bug-introducing       & + & not corrective & = & defective      \\
not bug-introducing & + & corrective       & = & clean  \\
not bug-introducing & + & not corrective & = & clean 
\end{tabular}
}
\end{table}

The information on the status of a feature is based on the files on which it is based. If a feature or a file has been edited several times within a release, the label is determined according to the following rules:

\begin{itemize}
\setlength{\itemsep}{-2pt}
\item in the case of features, it is checked whether the feature was marked as "defective" at least once within the release. If this is the case, it is assumed that the feature is broken in the release in question. If this is not the case, the feature is assumed to be clean.
\item in the case of files, the last commit of the files in that release is checked. If the file is marked "clean" there, it is assumed to be error-free. If it is marked as "defective", it is assumed that it is defective in this release.
\end{itemize}

The individual tables created in this way are then concatenated into a common table, resulting in a comprehensive listing of metrics including the associated labels. This list specifies the characteristics a feature or file must have in order to be classified as "defective" or "clean" and serves as a training basis for classifiers for future predictions.

A special case is the second file-based data set created for the comparison within the evaluation (J). It is based on the "simple" file-based dataset (I) with the metrics from \cite{Moser2008}, but additionally includes the eleven metrics of the feature-based dataset (G) from \autoref{tab:metrics}. For this purpose, the values of the feature-based metrics were mapped or transferred at file level. This means that for each file of a release referenced in the file-based dataset, it was analyzed which features were mentioned in the respective file. From these features, the corresponding metrics of the feature-based dataset were determined and the average value was calculated and entered into the entered dataset. If no features were mentioned in a file, the value $0$ is stored for the feature-based metrics. As mentioned at the beginning of this section, this way the feature-based dataset can be compared with a classical file-based dataset, because the conditions for the comparison have been created. A direct comparison between the different datasets is not practical for the reasons mentioned above.

Some supplementary key figures for the datasets are listed in \autoref{tab:dataset-numbers}. The row "unique" indicates how many unique rows are contained in the datasets. The datasets created in the manner described above can now be used to train the classifiers. This process will be explained in the next section.

\begin{table}[t]
\centering
\caption{Key figures of the data sets}
\label{tab:dataset-numbers}
\begin{tabular}{@{}lrrr@{}}
\toprule
characteristic    & \begin{tabular}[c]{@{}r@{}}feature-based\\ data set\end{tabular} & \begin{tabular}[c]{@{}r@{}}"simple" file-\\ based data set\end{tabular} & \begin{tabular}[c]{@{}r@{}}extended file-\\ based data set\end{tabular} \\ \midrule
\#attributes      & $11$ + label                                                     & $17$ + label                                                            & $28$ + label                                                            \\
\#data records    & $14.059$                                                         & $76.986$                                                                & $76.986$                                                                \\
thereof defective & $2.735$                                                          & $1.899$                                                                 & $1.899$                                                                 \\
thereof clean     & $11.324$                                                         & $75.087$                                                                & $75.087$                                                                \\
thereof unique    & $8.447$                                                          & $52.564$                                                                & $52.783$                                                                \\ \bottomrule
\end{tabular}
\end{table}

\subsection{Selection of Classifiers}

Although programming with the Python programming language was often used to create the data sets, the choice of a machine learning tool was not the library scikit-learn \cite{scikit}, but an independent solution. The WEKA-Workbench\footnote{\href{https://www.cs.waikato.ac.nz/ml/weka/}{https://www.cs.waikato.ac.nz/ml/weka/}} is used as a machine learning tool. This tool proved to be suitable for the underlying task by numerous citations in scientific papers (among others in \cite{Hammouri2018,Queiroz2016,Ratzinger2008}). The WEKA-Workbench (WEKA as acronym for \textbf{W}aikato \textbf{E}nvironment for \textbf{K}nowledge \textbf{A}nalysis) was developed at the University of Waikato in New Zealand and offers a large collection of machine learning algorithms and preprocessing tools for use within a graphical user interface \cite{Weka2016}. There are also interfaces for the Java programming language \cite{Weka2016}. An overview of the selected classification algorithms can be found in \autoref{tab:classifiers}. This also includes the abbreviations of the classifiers that will be used in the following.

\begin{table}[t]
\centering
\caption{Selection of classification algorithms}
\label{tab:classifiers}
\begin{tabular}{@{}ll@{}}
\toprule
algorithm                                        & abbreviation \\ \midrule
J48 Decision Trees                               & J48          \\
k-Nearest-Neighbors                              & KNN          \\
Logistic Regression                              & LR           \\
Na\"{\i}ve Bayes Bayes & NB           \\
Artificial Neural Networks                       & NN           \\
Random Forest                                    & RF           \\
Stochastic Gradient Descent                      & SGD          \\
Support Vector Machines                          & SVM          \\ \bottomrule
\end{tabular}
\end{table}

All classification algorithms presented above are already integrated in the WEKA tool. It receives as input the final data sets in a proprietary file format. The 13 calculated metrics form the attributes, whereas the target class is represented by the labels "defective" and "clean".

The training process of the classifiers took place within the graphical user interface of WEKA. Before the training, the split ratios for the division of the data sets into training data and test data were determined. It was determined individually for each of the software projects used and is based on the number of releases available in each case. However, care was always taken to approximate the commonly used split ratios of 80:20 and 75:25 as well as $70:30$. The training data ranges from $67\%$ to $80\%$ of the data records of the data sets. The earlier releases were assigned to the training data. The training data contains the data records of the later releases. An overview of the division into training and test data per software project is shown in \autoref{tab:splits}. The resulting split ratios for the entire data sets are: $69:31$ (feature-based data set) and $71:29$ ("simple" and extended file-based data set).

\begin{table}[t]
\centering
\caption{}
\label{tab:my-table}
\begin{tabular}{lrrr}
\hline
project           & \begin{tabular}[c]{@{}r@{}}releases fort\\ raining set\end{tabular} & \begin{tabular}[c]{@{}r@{}}releases for\\ test set\end{tabular} & \begin{tabular}[c]{@{}r@{}}split\\ ratio\end{tabular} \\ \hline
\textbf{Blender}  & 2.70 - 2.77                                                         & 2.78 - 2.80                                                     & $73:27$                                               \\
\textbf{Busybox}  & 1\_16\_0 - 1\_25\_0                                                 & 1\_26\_0 - 1\_30\_0                                             & $71:29$                                               \\
\textbf{Emacs}    & 25.0 - 26.0                                                         & 26.1 - 26.2                                                     & $71:29$                                               \\
\textbf{GIMP}     & 2\_8\_2 - 2\_10\_4                                                  & 2\_10\_6 - 2\_10\_12                                            & $71:29$                                               \\
\textbf{Gnumeric} & 1\_10\_0 - 1\_12\_10                                                & 1\_12\_20 - 1\_12\_30                                           & $75:25$                                               \\
\textbf{gnuplot}  & 4.0.0 - 4.6.0                                                       & 5.0.0                                                           & $80:20$                                               \\
\textbf{Irssi}    & 1.0.0 - 1.0.4                                                       & 1.0.5 - 1.0.6                                                   & $71:29$                                               \\
\textbf{libxml2}  & 2.9.0 - 2.9.7                                                       & 2.9.8 - 2.9.9                                                   & $80:20$                                               \\
\textbf{lighttpd} & 1.3.10 - 1.4.20                                                     & 1.4.30 - 1.4.40                                                 & $67:33$                                               \\
\textbf{MPSolve}  & 3.0.1 - 3.1.5                                                       & 3.1.6 - 3.1.7                                                   & $75:25$                                               \\
\textbf{Parrot}   & 1\_0\_0 - 5\_0\_0                                                   & 6\_0\_0 - 7\_0\_0                                               & $71:29$                                               \\
\textbf{Vim}      & 7.0 - 7.4                                                           & 8.0 - 8.1                                                       & $71:29$                                               \\ \hline
\end{tabular}
\end{table}

The training of the classifiers in WEKA was carried out for each classification algorithm with the respective standard settings. Only for the algorithms NN and RF further configurations were made. For the RF-algorithm a number of instances of $200$ was defined. This means that 200 decision trees perform parallel processing at the same time. There are no clear recommendations on how many instances should be selected. The selected value of $200$ was therefore determined independently, taking into account the scope of the data sets and the high number of attributes. For the NN algorithm an independently determined hidden layer structure of \texttt{(13,13,13} was chosen. This means that the artificial neural network has three hidden layer layers of 13 hidden layer neurons each. This allows them to process the large number of attributes more efficiently. Furthermore, no validation data was generated, since it is not intended to perform attribute selection or to adjust further classifier settings.

An analysis of the training process also showed that the file-based data sets are highly unbalanced with regard to the target class. With a value of about $97\%$, there are far more entries assigned to the label "clean". Balancedness, i.e. a balanced ratio ($50:50$ is not mandatory in the binary case) within the target classes, is however a prerequisite for the correct training of most classifiers. Ignoring this problem can lead to misleading accuracy, since most records are correctly assigned to the over-represented class and is a fundamental problem of accuracy metrics. To solve this problem, the SMOTE algorithm was applied to the file-based datasets \cite{Chawla2002}. The algorithm, whose acronym stands for \textbf{S}ynthetic \textbf{M}inority \textbf{O}ver-sampling \textbf{Te}chnique, performs an oversampling of the underrepresented class \cite{Chawla2002}. Using next-neighbor calculations based on the Euclidean distance between the attribute values of each dataset's datasets, new synthetic datasets are added (oversampling) so that the number of datasets of the relevant class increases \cite{Chawla2002}. In this case, the percentage of synthetic records generated is 2000, so for each existing record of the underrepresented class, 20 additional synthetic records are generated. Thus, the percentage of records with the label "defective" was increased to about $41\%$. At the same time the number of records grew by about $60\%$. The algorithm was only applied to the training data \cite{Chawla2002} . It is not intended to be applied to the test data so that the "ground truth" is not distorted or falsified. As a result, the metrics of the file-based datasets also changed. These are shown in \autoref{tab:dataset-numbers-new}.

\begin{table*}[t]
\centering
\caption{Updates key figures of the data sets}
\label{tab:dataset-numbers-new}
\begin{tabular}{@{}lrrrr@{}}
\toprule
\multirow{2}{*}{}   & \multicolumn{2}{c}{"simple" data set} & \multicolumn{2}{c}{extended data set} \\ \cmidrule(l){2-5} 
                    & before            & after             & before            & after             \\ \midrule
\#attributes        & $17$ + label      & $17$ + label      & $28$ + label      & $28$ + label      \\
\#data records      & $76.986$          & $111.706$         & $76.986$          & $112.706$         \\
thereof defective   & $1.899$           & $37.619$          & $1.899$           & $37.619$          \\
thereof clean       & $75.087$          & $75.087$          & $75.087$          & $75.087$          \\
thereof unique      & $52.564$          & $86.155$          & $52.783$          & $86.381$          \\
overall split ratio & $71:29$           & $81:19$           & $71:29$           & $81:19$           \\ \bottomrule
\end{tabular}
\end{table*}

The results obtained using the test data, which reflect the performance of the individual classifiers, are presented in the following chapter as part of the evaluation.


\setlength{\tymin}{3.5cm}
\begin{table*}[!t]
	\fontsize{7}{7}\selectfont
	\centering
	\caption{Let $R$ be a release consisting of $q$ commits: $\mathnormal{R = \{c_{1}, c_{2}, . . . c_{q}}\}$, $F$ be the set of all $p$ files changed by commits in $R$: $\mathnormal{F = \{f_{1}, f_{2}, . . . f_{p}}\}$. For each file $f\in F$, let $T$ be the set of all $n$ features (in $f$) affected by changes in $R$: $\mathnormal{T = \{feat_{1}, feat_{2}, . . . feat_{n}}\}$ , and each feature $feat\in T$ has a set $A$ of $m$ files which implement it: $A\subseteq F$: $\mathnormal{A = \{featfile_{1}, featfile_{2}, . . . featfile_{m}}\}$ We define our feature based metric as follows for each file \mformula{f}:}
	%\vspace{-.2cm}
	\label{tbl:featuredefects}
	\begin{threeparttable}
		\fontfamily{ptm}\selectfont
		
		\begin{tabulary}{\textwidth}{p{1cm}p{5cm}p{5cm}p{5cm}}
			
			\toprule
			\theader{metric}&\theader{description}	&	\theader{equation}	&	\theader{supplementary eq.}	\\
			\midrule
			$FCOMM$\tnote{1} & Average number of commits associated to the changed features of a file within a release. & $\mathit{FCOMM_R\,(f)} = \mathnormal{\dfrac{1}{n}\sum_{i=1}^{n} comm(feat_i,A_i)}$ & $\mathnormal{comm(feat_i,A_i) = \sum_{j=1}^{m} comm(featfile_j) }$ \\
			$FADEV$\tnote{2} & Average number of developers who changed the features of a files within a release. & $\mathit{FADEV_R\,(f)} = \mathnormal{\dfrac{1}{n}\sum_{i=1}^{n} adev(feat_i,A_i)}$ & $\mathnormal{adev(feat_i,A_i) = \sum_{j=1}^{m} adev(featfile_j) }$ \\
			$FDDEV$\tnote{3} & Average cumulated number of developers who changed the features of a files within a release. & $\mathit{FDDEV_R\,(f)} = \mathnormal{\dfrac{1}{n}\sum_{i=1}^{n} ddev(feat_i,A_i)}$ & $\mathnormal{ddev(feat_i,A_i) = \sum_{j=1}^{m} ddev(featfile_j) }$ \\
			$FEXP$\tnote{4} & Average experience\mtnote{5} of all developers who changed the features of a file within a release. & $\mathit{FEXP_R\,(f)} = \mathnormal{\dfrac{1}{n}\sum_{i=1}^{n} exp(feat_i,A_i)}$ & $\mathnormal{exp(feat_i,A_i) = \sum_{j=1}^{m} exp(featfile_j) }$ \\
			$FOEXP$\tnote{6} & Average experience\mtnote{5} of the developer who changed the features of a file most often within a release. & $\mathit{FOEXP_R\,(f)} = \mathnormal{\dfrac{1}{n}\sum_{i=1}^{n} oexp(feat_i,A_i)}$ & $\mathnormal{oexp(feat_i,A_i) = \sum_{j=1}^{m} oexp(featfile_j) }$ \\
			$FMODD$\tnote{7} & Average number of changes to the features of a file within a release. & $\mathit{FMODD_R\,(f)} = \mathnormal{\dfrac{1}{n}\sum_{i=1}^{n} modd(feat_i,A_i)}$ & $\mathnormal{modd(feat_i,A_i) = \sum_{j=1}^{m} modd(featfile_j) }$ \\
			$FMODS$\tnote{8} & Average number of changed features of a file within a release. & $\mathit{FMODS_R\,(f)} = \mathnormal{\dfrac{1}{n}\sum_{i=1}^{n} mods(feat_i,A_i)}$ & $\mathnormal{mods(feat_i,A_i) = \sum_{j=1}^{m} mods(featfile_j) }$ \\
			$FNLOC$\tnote{9} & Average number of lines of code of the underlying files of the changed features of a file within a release. & $\mathit{FNLOC_R\,(f)} = \mathnormal{\dfrac{1}{n}\sum_{i=1}^{n} nloc(feat_i,A_i)}$ & $\mathnormal{nloc(feat_i,A_i) = \sum_{j=1}^{m} nloc(featfile_j) }$ \\
			$FCYCO$\tnote{10} & Average cyclomatic complexity of the underlying files of the changed features of a file within a release. & $\mathit{FCYCO_R\,(f)} = \mathnormal{\dfrac{1}{n}\sum_{i=1}^{n} cyco(feat_i,A_i)}$ & $\mathnormal{cyco(feat_i,A_i) = \sum_{j=1}^{m} nloc(featfile_j) }$ \\								
			$FADDL$\tnote{11} & Average number of lines of code added to the underlying files of the changed features of a file within a release. & $\mathit{FADDL_R\,(f)} = \mathnormal{\dfrac{1}{n}\sum_{i=1}^{n} addl(feat_i,A_i)}$ & $\mathnormal{addl(feat_i,A_i) = \sum_{j=1}^{m} addl(featfile_j) }$ \\
			$FREML$\tnote{12} & Average number of lines of code deleted from the underlying files of the changed features of a file within a release. & $\mathit{FREML_R\,(f)} = \mathnormal{\dfrac{1}{n}\sum_{i=1}^{n} reml(feat_i,A_i)}$ & $\mathnormal{reml(feat_i,A_i) = \sum_{j=1}^{m} reml(featfile_j) }$ \\						
			\bottomrule				
		\end{tabulary}
		\begin{tablenotes}[para]
			~\item[1] $comm(featfile_j)$ counts commits in which the file $featfile_j$ was changed.
			~\item[2] $adev(featfile_j)$ counts the developers who changed the file $featfile_j$.  
			~\item[3] $ddev(featfile_j)$ cumulates the counts of developers who changed the file over commits $featfile_j$.
			~\item[4] $exp(featfile_j)$ returns the geometric mean of the experience\mtnote{5} of all developers who changed the file $featfile_j$.
			~\item[5] Experience is defined as the sum of the changed, deleted or added lines in the commits associated with the files.
			~\item[6] $oexp(featfile_j)$ returns the experience\mtnote{5} of the developer who changed the file $featfile_j$ most often within a release.
			~\item[7] $modd(featfile_j)$ returns the number of changes to the file $featfile_j$.
			~\item[8] $mods(featfile_j)$  returns the number of changed files in the release of the file $featfile_j$.
			~\item[9] $nloc(featfile_j)$ returns the number of lines of code of the file $featfile_j$.
			~\item[10] $cyco(featfile_j)$ returns the cyclomatic complexity of the file $featfile_j$.
			~\item[11] $addl(featfile_j)$ returns the number of added lines of code to the file $featfile_j$.
			~\item[12] $reml(featfile_j)$ returns the number of removed lines of code from the file $featfile_j$.
		\end{tablenotes}
	\end{threeparttable}
	\vspace{-0.3cm}
	%	\tbottom
\end{table*}

% !TEX root = ../doc.tex

\section{Evaluation}

Lorem ipsum.

% !TEX root = ../doc.tex

\section{Discussion}

Lorem ipsum.

% !TEX root = ../doc.tex

\section{Conclusion}

\st{Lorem ipsum.}


%% The next two lines define the bibliography style to be used, and
%% the bibliography file.
\bibliographystyle{ACM-Reference-Format}
\bibliography{doc}

%%
%% If your work has an appendix, this is the place to put it.
%\appendix


\end{document}
\endinput
