\documentclass[master,proposal,extern,palatino]{rgseThesis}

% für "lorem ipsum" Blindtext - sollte bei echten Arbeiten natürlich raus ;-)
\usepackage{lipsum}

% Angaben für die Titelseite

\author{John Doe}
% \studiengang{Computervisualistik} % Default ist Informatik, bei Proposals ignoriert

\title{Frying without Fat}

% \supervisor{w}{Dr. Sabine Mustermann} % Default ist Jan Jürjens
% \supervisorInfo{Muster-Institut} % Default ist Institut für Softwaretechnik

\secondSupervisor{w}{Dr. Karin Nickel}
\secondSupervisorInfo{Rabbit Burrow Inc.}

\externLogo{6cm}{logos/ist-logo-en} % optional Logo des externen Partners
\externName{Institut für Softwaretechnik} % optional Untertitel des Logos


% Literatur-Datenbank
\addbibresource{literature.bib}

% Tiefe des Inhaltsverzeichnisses; 1: bis section (empfohlen), 2: bis subsection
\setcounter{tocdepth}{1} 

\begin{document}

    % Umschalten der Sprache für englische Rubrikbezeichnungen (möglich: english, ngerman)
    \selectlanguage{english}

    % Titelseite
    \maketitle

    % Inhaltsverzeichnis
    \tableofcontents

    % Hier kommt jetzt der eigentliche Text des Proposals

    \section{Introduction and Motivation}

    In `The Bible' \cite{Juerjens2005SSD}, Jan Jürjens defines an extension to the UML modeling language that enables modeling of security features in UML diagrams.

    \paragraph{Suggestion:}
    To make version control more easy, try to put each sentence on a single line.
    Use your text editors wrapping option to wrap long sentences (i.e. lines) at the window borders.

    \section{Research Questions}

    \lipsum[1-2]

    \section{Work Packages and Schedule}

    \lipsum[3-4]

    \section{Scientific Background}

    \lipsum[5]

    % Literaturverzeichnis
    \printbibliography[heading=bibintoc]

\end{document}