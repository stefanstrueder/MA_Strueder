% !TEX root = ../thesis.tex

\chapter{Ergebnisse und Diskussion der Evaluation der Klassifikatoren}
\label{evaluation}

Dieses Kapitel dient der Evaluation der im vorangegangenen Kapitel erläuterten Klassifikatoren, die auf den drei in \hyperref[dataset-creation]{Kapitel 3} vorgestellten Datensets basieren und auf der Grundlage der Daten von 12 featurebasierten Softwareprojekten (siehe Anmerkung in \hyperref[xfig]{Kapitel 6}) aufgebaut wurden. Bei diesen handelt es sich wiederum um das featurebasierte Datenset sowie das \glqq einfache\grqq{} und das erweiterte dateibasierte Datenset, welches zusätzlich die Metriken des featurebasierten Datensets, auf Dateiebene gemapped, enthält. Die Datensets bestehen jeweils aus den Attributen sowie dem Label als Spalten und in den Zeilen die zugehörigen Werte für jedes Feature beziehungsweise jede Datei aggregiert nach Release. Die Evaluation geschieht durch verschiedene Evaluationsmetriken, welche bereits in \hyperref[eval-metrics]{Abschnitt 2.4} vorgestellt wurden und auf Werten von sogenannten Konfusionsmatrizen und ROC-Kurven basieren. Erweitert wird dieser Teil der Evaluation mit dem Vergleich zwischen dem \glqq einfachen\grqq{} und erweiterten dateibasierten Datenset zur Messung der Einflüsse der featurebasierten Metriken auf die Performanz der Vorhersagen. Es werden somit die Arbeitsschritte erläutert, welche durchgeführt wurden, um das dritte Forschungsziel \glqq Evaluation und Gegenüberstellung der Klassifikatoren sowie Vergleich zu einer klassischen Vorhersagetechnik, die keine Features nutzt\grqq{} abzuschließen. Darüber hinaus werden die folgenden zugehörigen Forschungsfragen beantwortet:
\vspace{-\topsep}
\begin{itemize}
\setlength{\itemsep}{-2pt}
 \item RQ3a: Welches Vorgehen bietet sich zum internen Vergleich der Klassifikatoren an?
 \item RQ3b: Welche Ergebnisse können bei der Vorhersage von fehlerhaften oder fehlerfreien Features erzielt werden? (\hyperref[feat-results]{Abschnitt 5.1})
 \item RQ3d: Wie beeinflusst die Verwendung von featurebasierten Metriken die dateibasierte Fehlervorhersage? (\hyperref[classic-eval]{Abschnitt 5.2})
\end{itemize} 
\smallskip
\hrule
\smallskip
\label{results}
Dieses Kapitel umfasst die Ergebnisse und vergleichenden Diskussionen zu den Tests der Klassifikatoren der drei Datensets, aufgeteilt in einen Abschnitt für das featurebasierte Datenset und einen Abschnitt für die dateibasierten Datensets. Darüber hinaus sind die jeweiligen Abschnitte wie folgt aufgeteilt: \glqq Konfusionsmatrizen\grqq{}, \glqq Accuracy, TP-Rate / Recall, FP-Rate, Precision, F-Score\grqq{}, \glqq ROC-Kurven und ROC-Bereiche\grqq{} sowie \glqq Zusammenfassung\grqq{}. Zur Anwendung kommen sämtliche Evaluationsmetriken, welche bereits in \hyperref[eval-metrics]{Abschnitt 2.4} vorgestellt wurden. Das zum Training der Klassifikatoren verwendete Werkzeug WEKA besitzt die Option, Konfusionsmatrizen zu den durchgeführten Tests der Klassifikatoren auszugeben. Darüber hinaus wurden alle Evaluationsmetriken automatisiert von WEKA berechnet. Ferner besitzt das Werkzeug zudem die Fähigkeit, ROC-Kurven mit den entsprechenden ROC-Werten auszugeben.

\fbox{\parbox{\linewidth}{RQ3a: WELCHES VORGEHEN BIETET SICH ZUM INTERNEN VERGLEICH DER KLASSIFIKATOREN AN?\medskip\\
Das Vorgehen zum internen Vergleich der Klassifikatoren umfasst die Betrachtung von Evaluationsmetriken, welche auf Basis der Ergebnisse des Tests der Klassifikatoren errechnet werden und die Performanz der Vorhersagen auf verschiedene Weisen messen. Für den Vergleich werden klassische Evaluationsmetriken verwendet, welche auf Basis von Konfusionsmatrizen berechnet werden. Die betrachteten Metriken lauten: Accuracy, TP-Rate / Recall, Precision und F-Score. Ebenfalls hinzugezogen werden die jeweiligen ROC-Kurven der Klassifikatoren inklusive der ROC-Bereiche. Diese Metriken stellen einen Standard für die Messung der Performanz der Vorhersagen von Klassifikatoren dar.}}

\section{Ergebnisse und Vergleich der featurebasierten Klassifikatoren}
\label{feat-results}

Die Präsentation der Ergebnisse beginnt mit denen der Klassifikatoren des featurebasierten Datensets. Auf der Basis der vorgestellten Ergebnisse wird die folgende Forschungsfrage im Fließtext sowie zusammengefasst am Ende des Abschnitts beantwortet:
\vspace{-\topsep}
\begin{itemize}
\setlength{\itemsep}{-2pt}
 \item RQ3b: Welche Ergebnisse können bei der Vorhersage von fehlerhaften oder fehlerfreien Features erzielt werden?
\end{itemize}

Auf Basis der Betrachtung der Ergebnisse der Evaluationsmetriken kann die Performanz hinsichtlich der Vorhersagen von fehlerhaften oder fehlerfreien Features gemessen und interpretiert werden.  

\paragraph{Konfusionsmatrizen}
Die Konfusionsmatrizen dienen als Basis der Ergebnisse der Evaluation der Klassifikatoren anhand der zuvor vorgestellten Metriken. Die Matrizen des featurebasierten Datensets sind in \autoref{tab:mat-feat} aufgeführt. Dabei bilden die Spalten die von den Klassifikatoren vorhergesagten Label ab. Die Zeilen bilden wiederum die \glqq Ground Truth\grqq{}, also die im Rahmen der Erstellung der Datensets ermittelte Realität, ab. Außerdem werden die ermittelten Werte der jeweiligen Klassen in den Spalten \glqq total\grqq{} zusammengezählt. Da für jeden Klassifikator dasselbe Testset gewählt wurde, sind die Werte der \glqq total\grqq-Spalte identisch. Die für die Klassifikatoren verwendeten Abkürzungen können \autoref{tab:classifiers} entnommen werden.

\begin{table}[h!t]
\centering
\caption{Konfusionsmatrizen des featurebasierten Datensets}
\label{tab:mat-feat}
\resizebox{\linewidth}{!}{%
\begin{tabular}{|>{\hspace{0pt}}p{0.119\linewidth}>{\hspace{0pt}}p{0.362\linewidth}|>{\RaggedLeft\hspace{0pt}}p{0.156\linewidth}>{\RaggedLeft\hspace{0pt}}p{0.214\linewidth}>{\RaggedLeft\hspace{0pt}}p{0.139\linewidth}|} 
\cline{2-5}
\multicolumn{1}{>{\Centering\hspace{0pt}}p{0.119\linewidth}|}{} & \textbf{Ermittelt -\textgreater{}} & \textbf{defekt}  & \textbf{fehlerfrei}  & \textbf{total}  \\ 
\hline
\multirow{3}{0.119\linewidth}{\hspace{0pt}J48}                  & Realität defekt                    & $446$              & $320$                  & $766$             \\
                                                                & Realität fehlerfrei                & $483$              & $3.142$                & $3.625$           \\
                                                                & total                              & $929$              & $3.462$                & $4.391$           \\ 
\hline
\multirow{3}{0.119\linewidth}{\hspace{0pt}KNN}                  & Realität defekt                    & $371$              & $395$                  & $766$             \\
                                                                & Realität fehlerfrei                & $569$              & $3.056$                & $3.625$           \\
                                                                & total                              & $940$              & $3.451$                & $4.391$           \\ 
\hline
\multirow{3}{0.119\linewidth}{\hspace{0pt}LR}                   & Realität defekt                    & $309$              & $457$                  & $766$             \\
                                                                & Realität fehlerfrei                & $1.020$            & $2.605$                & $3.625$           \\
                                                                & total                              & $1.329$            & $3.061$                & $4.391$           \\ 
\hline
\multirow{3}{0.119\linewidth}{\hspace{0pt}NB}                   & Realität defekt                    & $322$              & $444$                  & $766$             \\
                                                                & Realität fehlerfrei                & $265$              & $3.360$                & $3.625$           \\
                                                                & total                              & $587$              & $3.804$                & $4.391$           \\ 
\hline
\multirow{3}{0.119\linewidth}{\hspace{0pt}NN}                   & Realität defekt                    & $327$              & $439$                  & $766$             \\
                                                                & Realität fehlerfrei                & $1.019$            & $2.60$6                & $3.625$           \\
                                                                & total                              & $1.346$            & $3.045$                & $4.391$           \\ 
\hline
\multirow{3}{0.119\linewidth}{\hspace{0pt}RF}                   & Realität defekt                    & $504$              & $262$                  & $766$             \\
                                                                & Realität fehlerfrei                & $450$              & $3.175$                & $3.625$           \\
                                                                & total                              & $954$              & $3.437$                & $4.391$           \\ 
\hline
\multirow{3}{0.119\linewidth}{\hspace{0pt}SGD}                  & Realität defekt                    & $251$              & $515$                  & $766$             \\
                                                                & Realität fehlerfrei                & $938$              & $2.687$                & $3.625$           \\
                                                                & total                              & $1.189$            & $3.202$                & $4.391$           \\ 
\hline
\multirow{3}{0.119\linewidth}{\hspace{0pt}SVM}                  & Realität defekt                    & $176$              & $590$                  & $766$             \\
                                                                & Realität fehlerfrei                & $899$              & $2.726$                & $3.625$           \\
                                                                & total                              & $1.075$            & $3.316$                & $4.391$           \\
\hline
\end{tabular}
}
\end{table}

\paragraph{Accuracy, TP-Rate / Recall, FP-Rate, Precision, F-Score}
Die erste Metrik die verglichen wird ist die Accuracy. Dargestellt werden die Ergebnisse zum besseren Vergleich als Balkendiagramme. Das Diagramm des feturebasierten Datensets ist in \autoref{fig:final-feat} dargestellt. Die konkreten Zahlenwerte können im \hyperref[appendix2]{Anhang} eingesehen werden.

\begin{figure}[h!t]
    \centering
    \includegraphics[width=\textwidth]{images/final_feat}
    \caption{Vergleich der Accuracies des featurebasierten Datensets\label{fig:final-feat}}
\end{figure}

Die Ergebnisse zeigen, dass die Accuracies der Klassifikatoren zwischen $66\%$ und $84\%$ liegen. Die höchsten Werte erreichten die Klassifikatoren NB und RF mit einer Treffergenauigkeit von jeweils $84\%$. Mit einem Wert von $82\%$ erreichte auch der J48-Klassifikator eine überdurchschnittlich hohe Performanz. Darauf folgt der KNN-Klassifikator mit $78\%$. Mit Werten von $66\%$ beziehungsweise $67\%$ schnitten die Klassifikatoren LR, NN, SGD und SVM am schlechtesten ab. Sie ermittelten somit in über $30\%$ der Vorhersagen ein falsches Ergebnis. Auffällig an diesen Ergebnissen ist die hohe Performanz beider Entscheidungsbaum-basierter Klassifikatoren J48 und RF. Sie scheinen somit die Eingabemenge am besten verarbeiten zu können, um Rückschlüsse auf die neuen Daten in Form der Testdaten ziehen zu können.

Die Ergebnisse der weiteren Evaluationsmetriken TP-Rate / Recall, FP-Rate, Precision und F-Score können in \autoref{tab:met-results-feat} eingesehen werden. Aufgeteilt werden die Ergebnisse nach den Werten der Zielklasse, \glqq defekt\grqq{} und \glqq fehlerfrei\grqq. Zudem wird der Mittelwert der Ergebnisse beider Werte der Zielklasse angegeben. Er zeigt die Performanz aggregiert für beide Werte an und liegt im Idealfall bei $1,00$ beziehungsweise $0,00$ für die FP-Rate. Die vollständigen Tabellen der Ergebnisse der Evaluation, inklusive einer weiteren Metrik, können im \hyperref[appendix3]{Anhang} gefunden werden.

\begin{table}[h!t]
\centering
\caption{Ergebnisse der Evaluationsmetriken des featurebasierten Datensets}
\label{tab:met-results-feat}
\resizebox{\linewidth}{!}{%
\begin{tabular}{|>{\hspace{0pt}}p{0.12\linewidth}>{\hspace{0pt}}p{0.349\linewidth}|>{\RaggedLeft\hspace{0pt}}p{0.156\linewidth}>{\RaggedLeft\hspace{0pt}}p{0.214\linewidth}>{\RaggedLeft\hspace{0pt}}p{0.141\linewidth}|} 
\cline{3-5}
\multicolumn{1}{>{\hspace{0pt}}p{0.12\linewidth}}{} &                  & \textbf{defekt}  & \textbf{fehlerfrei}  & \textbf{Mittel}   \\ 
\hline
\multirow{4}{0.12\linewidth}{\hspace{0pt}J48}       & TP-Rate / Recall & $0,58$             & $0,87$                 & $0,73$              \\
                                                    & FP-Rate          & $0,13$             & $0,42$                 & $0,28$              \\
                                                    & Precision        & $0,48$             & $0,91$                 & $0,70$              \\
                                                    & F-Score          & $0,53$             & $0,89$                 & $0,71$              \\ 
\hline
\multirow{4}{0.12\linewidth}{\hspace{0pt}KNN}       & TP-Rate / Recall & $0,48$             & $0,86$                 & $0,67$              \\
                                                    & FP-Rate          & $0,16$             & $0,52$                 & $0,68$              \\
                                                    & Precision        & $0,40$             & $0,89$                 & $0,65$              \\
                                                    & F-Score          & $0,44$             & $0,86$                 & $0,66$              \\ 
\hline
\multirow{4}{0.12\linewidth}{\hspace{0pt}LR}        & TP-Rate / Recall & $0,40$             & $0,72$                 & $0,56$              \\
                                                    & FP-Rate          & $0,28$             & $0,60$                 & $0,44$              \\
                                                    & Precision        & $0,23$             & $0,85$                 & $0,54$              \\
                                                    & F-Score          & $0,30$             & $0,78$                 & $0,54$              \\ 
\hline
\multirow{4}{0.12\linewidth}{\hspace{0pt}NB}        & TP-Rate / Recall & $0,42$             & $0,93$                 & $0,68$              \\
                                                    & FP-Rate          & $0,07$             & $0,58$                 & $0,33$              \\
                                                    & Precision        & $0,55$             & $0,88$                 & $0,72$              \\
                                                    & F-Score          & 0,48             & $0,91$                 & $0,70$              \\ 
\hline
\multirow{4}{0.12\linewidth}{\hspace{0pt}NN}        & TP-Rate / Recall & $0,43$             & $0,72$                 & $0,58$              \\
                                                    & FP-Rate          & $0,28$             & $0,57$                 & $0,43$              \\
                                                    & Precision        & $0,24$             & $0,86$                 & $0,55$              \\
                                                    & F-Score          & $0,31$             & $0,78$                 & $0,55$              \\ 
\hline
\multirow{4}{0.12\linewidth}{\hspace{0pt}RF}        & TP-Rate / Recall & $0,66$             & $0,8$8                 & $0,77$              \\
                                                    & FP-Rate          & $0,12$             & $0,34$                 & $0,23$              \\
                                                    & Precision        & $0,53$             & $0,92$                 & $0,73$              \\
                                                    & F-Score          & $0,59$             & $0,90$                 & $0,75$              \\ 
\hline
\multirow{4}{0.12\linewidth}{\hspace{0pt}SGD}       & TP-Rate / Recall & $0,34$             & $0,74$                 & $0,54$              \\
                                                    & FP-Rate          & $0,56$             & $0,67$                 & $0,62$              \\
                                                    & Precision        & $0,21$             & $0,84$                 & $0,53$              \\
                                                    & F-Score          & $0,26$             & $0,79$                 & $0,53$              \\ 
\hline
\multirow{4}{0.12\linewidth}{\hspace{0pt}SVM}       & TP-Rate / Recall & $0,23$             & $0,75$                 & $0,48$              \\
                                                    & FP-Rate          & $0,25$             & $0,77$                 & $0,51$              \\
                                                    & Precision        & $0,16$             & $0,82$                 & $0,49$              \\
                                                    & F-Score          & $0,19$             & $0,7$9                 & $0,49$              \\
\hline
\end{tabular}
}
\end{table}

Eine erste Betrachtung der Ergebnisse der Evaluationsmetriken zeigt, dass die Resultate (mit Ausnahme der FP-Rate) bezogen auf das Label \glqq defekt\grqq{} schlechter ausfallen, als die des Labels \glqq fehlerfrei\grqq. Die im Vergleich besten Ergebnisse wurden vom RF-Klassifikator für das Label \glqq defekt\grqq{} erreicht. $66\%$ der tatsächlich defekten Datenpunkte wurden auch als solche vorhergesagt und lediglich $12\%$ der Datenpunkte wurden hingegen fälschlicherweise als defekt vorhergesagt. Die Werte der Precision und des F-Scores liegen bei $53\%$, respektive $59\%$, auf einem durchschnittlichen Niveau. Vergleichbare Ergebnisse erzielte der J48-Klassifikator. Die insgesamt schlechtesten Werte weisen SGD und SVM auf. Sie erkannten tatsächlich defekte Datenpunkte mit einer Genauigkeit von $34\%$ beziehungsweise $23\%$. Die weiteren Klassifikatoren operierten auf einem durchschnittlichen Level.

Die jeweiligen Mittelwerte können betrachtet werden, um die Gesamtperformanz der Klassifikatoren für beide Label zu vergleichen. Hier schneidet erneut der RF-Klassifikator am besten ab. Die Werte für die TP-Rate, die Precision und den F-Score liegen im Umfeld von über $73\%$ und sind damit vergleichsweise überdurchschnittlich hoch. Die FP-Rate in Höhe von $23\%$, die hauptsächlich durch die Ergebnisse hinsichtlich des Labels \glqq fehlerfrei\grqq{} begründet ist, liegt auf dem niedrigsten Niveau aller Klassifikatoren. Erneut sind die Ergebnisse des ebenfalls Entscheidungsbaum-basierten Klassifikators J48 auf einem ähnlichen Niveau. Die Klassifikatoren KNN und NB weisen durch ihre hohen Werte der FP-Rate insgesamt durchschnittliche Performanz auf. In der Gesamtheit betrachtet zeigen die Werte der Klassifikatoren LR, NN, SGD und SVM, dass diese im Vergleich die schlechteste Performanz besitzen. Insbesondere die FP-Raten von über $43\%$ sind auf einem hohen, nicht wünschenswerten Niveau. 

\paragraph{ROC-Kurven und -Bereiche}
Die Interpretation der ROC-Kurven und ROC-Bereiche erfolgt anhand des in \hyperref[roc-def]{Abschnitt 2.4} vorgestellten Schemas. Die ROC-Kurven samt der Werte der ROC-Bereiche (repräsentiert durch \glqq AUC\grqq{}) sind in \autoref{fig:roc-feat}  dargestellt. Zu sehen sind jeweils die von WEKA ausgegebenen und unveränderten Plots. Die dargestellten Farbverläufe der Kurven verdeutlichen keine für diesen Zweck relevanten Informationen und können somit ignoriert werden.

\begin{figure}[h!t]
  \centering
  \subfloat[][J48\\AUC = $0,77$]{\includegraphics[width=0.25\linewidth]{images/j48_feat}} 
  \subfloat[][LR\\AUC = $0,62$]{\includegraphics[width=0.25\linewidth]{images/lr_feat}}
  \subfloat[][NN\\AUC = $0,61$]{\includegraphics[width=0.25\linewidth]{images/nn_feat}}
  \subfloat[][SGD\\AUC = $0,53$]{\includegraphics[width=0.25\linewidth]{images/sgd_feat}}
  \qquad
  \subfloat[][KNN\\AUC = $0,73$]{\includegraphics[width=0.25\linewidth]{images/knn_feat}}
  \subfloat[][NB\\AUC = $0,80$]{\includegraphics[width=0.25\linewidth]{images/nb_feat}}
  \subfloat[][RF\\AUC = $0,84$]{\includegraphics[width=0.25\linewidth]{images/rf_feat}} 
  \subfloat[][SVM\\AUC = $0,49$]{\includegraphics[width=0.25\linewidth]{images/svm_feat}}
  \caption{ROC-Kurven des featurebasierten Datensets \label{fig:roc-feat}}
\end{figure}

Es ist zu erkennen, dass keine ROC-Kurve den in \hyperref[roc-def]{Abschnitt 2.4} vorgestellten Idealfall annähert. Die beste Vorhersageperformanz zeigt erneut der RF-Klassifikator mit einem ROC-Bereich von $0,84$. Der NB-Klassifikator zeigt eine minimal geringere Performanz mit einem ROC-Bereich von $0,80$. Eine durchschnittliche Performanz von $0,77$ beziehungsweise $0,72$, bezogen auf den ROC-Bereich, weisen die Klassifikatoren J48 und KNN auf. Die Performanz der LR- und NN-Klassifikatoren liegen im unteren Durchschnitt. Den unerwünschten Fall einer winkelhalbierenden ROC-Kurve zeigen die SGD- und SVM-Klassifikatoren. Dies spiegeln auch die ROC-Bereiche wider. Die Vorhersagen nähern sich also einem zufälligen Raten an, statt wie gewünscht präzise zu sein.

\paragraph{Zusammenfassung}
In jeder der betrachteten Kategorien von Metriken erwiesen sich die Ent-scheidungsbaum-basierten Klassifikatoren J48 und RF als am performantesten im Vergleich zu den weiteren Klassifikatoren. Der RF-Klassifikator erreichte dabei meist zudem eine etwas höhere Performanz. Die den Klassifikatoren zugrundeliegenden Algorithmen scheinen das gegebene featurebasierte Datenset mit seinen elf Attributen, hinsichtlich der Entwicklung der Ableitungsfunktion zur Vorhersage neuer Daten, am besten verarbeiten zu können.
Die Klassifikatoren SGD und SVM zeigten für jede Kategorie von Evaluationsmetriken stets die im Vergleich schlechtesten Performanzen. Es wurde zudem festgestellt, dass die Vorhersagen dieser Klassifikatoren sich einem zufälligen Raten annähern. Das featurebasierte Datenset scheint somit für diese Klassifikationsalgorithmen nicht geeignet zu sein. Gründe dafür können eine zu geringe Anzahl an Instanzen oder eine zu hohe Anzahl an Attributen sein.

\fbox{\parbox{\linewidth}{RQ3b: WELCHE ERGEBNISSE KÖNNEN BEI DER VORHERSAGE VON FEHLERHAFTEN ODER FEHLERFREIEN FEATURES ERZIELT WERDEN?\medskip\\
Es konnte gezeigt werden, dass die Entscheidungsbaum-basierten Klassifikatoren J48 und RF mit Accuracies von $84\%$ die höchste Performanz der acht Klassifikatoren aufweisen. Die weiteren Evaluationsmetriken bestätigten diese Ergebnisse. Die Klassifikatoren SGD und SVM zeigten die im Vergleich schlechtesten Performanzergebnisse.}}

\section{Ergebnisse und Vergleich der dateibasierten Klassifikatoren}
\label{classic-eval}

Dieser Abschnitt dient zur Evaluation und zum Vergleich der Klassifikatoren der dateibasierten Datensets, deren Erstellung in \hyperref[new-datasets]{Abschnitt 3.3} erläutert wurde. Der Grund dieses Vergleiches ist die Messung der Einflüsse der featurebasierten Metriken auf die Vorhersagen einer klassischen dateibasierten Methode, die der wissenschaftlichen Literatur entnommen wurde \cite{Moser2008}. Die Aufteilung dieses Abschnitts ist analog zum vorherigen Abschnitt. Auf der Basis der vorgestellten Ergebnisse wird die folgende Forschungsfrage im Fließtext sowie zusammengefasst am Ende des Abschnitts beantwortet:
\vspace{-\topsep}
\begin{itemize}
\setlength{\itemsep}{-2pt}
 \item RQ3d: Wie beeinflusst die Verwendung von featurebasierten Metriken die dateibasierte Fehlervorhersage?
\end{itemize} 

Durch die Betrachtung der Ergebnisse der Evaluationsmetriken beider Datensets, kann der Einfluss der featurebasierten Metriken im erweiterten dateibasierten Datensets auf die Performanz der Vorhersagen analysiert werden. 

\paragraph{Konfusionsmatrizen}
Die Konfusionsmatrizen der dateibasierten Datensets sind in \autoref{tab:mat-eval} aufgeführt. Die übergeordneten Spalten sind dabei in das \glqq einfache\grqq{} und das erweiterte Datenset angeordnet. Beide Datensets umfassen die selbe Anzahl an Datensätzen, somit sind die \glqq total\grqq -Spalten identisch.
Eine kurze Analyse der \glqq defekt\grqq - \glqq Realität defekt\grqq - Felder zeigt, dass nur sehr wenige tatsächlich defekten Datenpunkte auch wirklich als defekt vorhergesagt wurden. Dies spiegelt sich auch in den weiteren Ergebnissen der Evaluationsmetriken wider.

\begin{table}[h!t]
\centering
\caption{Konfusionsmatrizen der dateibasierten Datensets}
\label{tab:mat-eval}
\resizebox{\linewidth}{!}{%
\begin{tabular}{|>{\hspace{0pt}}p{0.075\linewidth}>{\hspace{0pt}}p{0.231\linewidth}|>{\RaggedLeft\hspace{0pt}}p{0.1\linewidth}>{\RaggedLeft\hspace{0pt}}p{0.135\linewidth}>{\RaggedLeft\hspace{0pt}}p{0.1\linewidth}|>{\RaggedLeft\hspace{0pt}}p{0.102\linewidth}>{\RaggedLeft\hspace{0pt}}p{0.137\linewidth}>{\RaggedLeft\hspace{0pt}}p{0.104\linewidth}|} 
\cline{3-8}
\multicolumn{1}{>{\hspace{0pt}}p{0.075\linewidth}}{}            &                                    & \multicolumn{3}{>{\Centering\hspace{0pt}}p{0.335\linewidth}|}{\textbf{\glqq einfaches\grqq{} dateibasiertes Datenset}} & \multicolumn{3}{>{\Centering\hspace{0pt}}p{0.343\linewidth}|}{\textbf{erweitertes dateibasiertes Datenset}}  \\ 
\cline{2-8}
\multicolumn{1}{>{\Centering\hspace{0pt}}p{0.075\linewidth}|}{} & \textbf{Ermittelt -\textgreater{}} & \textbf{defekt} & \textbf{fehlerfrei} & \textbf{total}                                & \textbf{defekt} & \textbf{fehlerfrei} & \textbf{total}                                        \\ 
\hline
\multirow{3}{0.075\linewidth}{\hspace{0pt}J48}                   & Realität defekt                    & $11$              & $102$                 & $113$                                           & $15$              & $98$                  & $113$                                                   \\
                                                                & Realität fehlerfrei                & $911$             & $20.923$              & $21.834$                                        & $1.120$           & $20.714$              & $21.834$                                                \\
                                                                & total                              & $922$             & $21.025$              & $21.947$                                        & $1.135$           & $2.0812$              & $21.947$                                                \\ 
\hline
\multirow{3}{0.075\linewidth}{\hspace{0pt}KNN}                  & Realität defekt                    & $29$              & $84$                  & $113$                                           & $20$              & $93$                  & $113$                                                   \\
                                                                & Realität fehlerfrei                & $5.209$           & $16.625$              & $21.834$                                        & $4.861$           & $16.973$              & $21.834$                                                \\
                                                                & total                              & $5.238$           & $16.709$              & $21.947$                                        & $4.881$           & $17.066$              & $21.947$                                                \\ 
\hline
\multirow{3}{0.075\linewidth}{\hspace{0pt}LR}                   & Realität defekt                    & $36$              & $77$                  & $113$                                           & $32$              & $81$                  & $113$                                                   \\
                                                                & Realität fehlerfrei                & $2.520$           & $19.314$              & $21.834$                                        & $2.578$           & $19.256$              & $21.834$                                                \\
                                                                & total                              & $2.556$           & $19.391$              & $21.947$                                       & $2.610$           & $19.337$              & $21.947$                                                \\ 
\hline
\multirow{3}{0.075\linewidth}{\hspace{0pt}NB}                   & Realität defekt                    & $77$              & $36$                  & $113$                                           & $83$              & $30$                  & $113$                                                   \\
                                                                & Realität fehlerfrei                & $18.700$          & $3.134$               & $21.834$                                        & $18.750$          & $3.084$               & $21.834$                                                \\
                                                                & total                              & $18.777$          & $3.170$               & $21.947$                                        & $18.833$          & $3.114$               & $21.947$                                                \\ 
\hline
\multirow{3}{0.075\linewidth}{\hspace{0pt}NN}                   & Realität defekt                    & $6$               & $107$                 & $113$                                           & $1$               & $112$                 & $113$                                                   \\
                                                                & Realität fehlerfrei                & $4.341$           & $17.493$              & $21.834$                                        & $141$             & $21.693$              & $21.834$                                                \\
                                                                & total                              & $4.347$           & $17.600$              & $21.947$                                        & $142$             & $21.805$              & $21.947$                                                \\ 
\hline
\multirow{3}{0.075\linewidth}{\hspace{0pt}RF}                   & Realität defekt                    & $7$               & $106$                 & $113$                                           & $4$               & $109$                 & $113$                                                   \\
                                                                & Realität fehlerfrei                & $382$             & $21.452$              & $21.834$                                        & $366$             & $21.468$              & $21.834$                                                \\
                                                                & total                              & $389$             & $21.558$              & $21.947$                                        & $370$             & $21.577$              & $21.947$                                                \\ 
\hline
\multirow{3}{0.075\linewidth}{\hspace{0pt}SGD}                  & Realität defekt                    & $22$              & $91$                  & $113$                                           & $20$              & $93$                  & $113$                                                   \\
                                                                & Realität fehlerfrei                & $1.260$           & $20.574$              & $21.834$                                        & $1.226$           & $20.608$              & $21.834$                                                \\
                                                                & total                              & $1.282$           & $20.665$              & $21.947$                                        & $1.246$           & $20.701$              & $21.947$                                                \\ 
\hline
\multirow{3}{0.075\linewidth}{\hspace{0pt}SVM}                  & Realität defekt                    & $22$              & $91$                  & $113$                                           & $21$              & $92$                  & $113$                                                   \\
                                                                & Realität fehlerfrei                & $1.041$           & $20.793$              & $21.834$                                        & $991$             & $20.843$              & $21.834$                                                \\
                                                                & total                              & $1.063$           & $20.884$              & $21.947$                                        & $1.012$           & $20.935$              & $21.947$                                                \\
\hline
\end{tabular}
}
\end{table}

\paragraph{Accuracy, TP-Rate / Recall, FP-Rate, Precision, F-Score}
Die Accuracies der Klassifikatoren der dateibasierten Datensets sind in \autoref{fig:final-eval} dargestellt. Die konkreten Zahlenwerte können dem \hyperref[appendix2]{Anhang} entnommen werden.

\begin{figure}[h!t]
    \centering
    \includegraphics[width=\textwidth]{images/final_eval}
    \caption{Vergleich der Accuracies der dateibasierten Datensets\label{fig:final-eval}}
\end{figure}

Der Vergleich zwischen den Datensets zeigt, dass die Einbindung der featurebasierten Metriken für sechs von acht Klassifikatoren keinen signifikanten Einfluss auf die Performanz der Vorhersagen hat. Die erzielten Werte der Accuracies sind dort jeweils auf dem selben Niveau. Mit etwa $15\%$ Treffergenauigkeit erzielt der NB-Klassifikator die im Vergleich schlechteste Accuracy. Dieser scheint die hohe Anzahl an Datensätzen nicht korrekt verarbeiten zu können. Die Klassifikatoren KNN und LR erreichten mit etwa $76\%$ beziehungsweise $87\%$ überdurchschnittlich hohe Ergebnisse. Mit Werten von über $90\%$ besitzen die Klassifikatoren J48, RF, SGD und SVM äußerst hohe Werte, deren Aussagekraft zunächst infrage gestellt werden sollten. Eine Betrachtung der weiteren Evaluationsmetriken sollte vorgenommen werden.
Eine Auffälligkeit zeigt der NN-Klassifikator. Es ist ein deutlicher Sprung der Werte von etwa $18\%$ zwischen dem \glqq einfachen\grqq{} und dem erweiterten Datenset zu erkennen. Diese Auffälligkeit sollte ebenfalls anhand der weiteren Evaluationsmetriken analysiert und interpretiert werden. Sie kann einerseits verdeutlichen, dass der Klassifikator des erweiterten Datensets durch die Hinzunahme der zusätzlichen Attribute genauere Vorhersagen treffen kann oder durch diesen Umstand die Vorhersagen negativ beeinflusst werden, sodass die Ergebnisse verzerrt werden.

Die Ergebnisse der weiteren Evaluationsmetriken der dateibasierten Datensets sind in \autoref{tab:met-results} aufgeführt. Die übergeordneten Spalten werden erneut durch die beiden Datensets repräsentiert. Die vollständigen Tabellen der Ergebnisse der Evaluation, inklusive einer weiteren Metrik, können im \hyperref[appendix3]{Anhang} gefunden werden.

\begin{table}[h!t]
\centering
\caption{Ergebnisse der Evaluationsmetriken der dateibasierten Datensets (TPR = TP-Rate)}
\label{tab:met-results}
\resizebox{\linewidth}{!}{%
\begin{tabular}{|ll|rrr|rrr|} 
\cline{3-8}
\multicolumn{1}{l}{} &              & \multicolumn{3}{c|}{\textbf{\glqq einfaches\grqq{} dateibasiertes Datenset} } & \multicolumn{3}{c|}{\textbf{erweitertes dateibasiertes Datenset} }  \\ 
\cline{3-8}
\multicolumn{1}{l}{} &              & \textbf{defekt}  & \textbf{fehlerfrei}  & \textbf{Mittel}          & \textbf{defekt}  & \textbf{fehlerfrei}  & \textbf{Mittel}           \\ 
\hline
\multirow{4}{*}{J48} & TPR / Recall & $0,10$             & $0,96$                 & $0,53$                     & $0,13$             & $0,95$                 & $0,54$                      \\
                     & FP-Rate      & $0,04$             & $0,90$                 & $0,47$                     & $0,05$             & $0,87$                 & $0,46$                      \\
                     & Precision    & $0,01$             & $1,00$                 & $0,51$                     & $0,01$             & $1,00$                 & $0,51$                      \\
                     & F-Score      & $0,02$             & $0,98$                 & $0,50$                     & $0,02$             & $0,97$                 & $0,50$                      \\ 
\hline
\multirow{4}{*}{KNN} & TPR / Recall & $0,26$             & $0,76$                 & $0,51$                     & $0,18$             & $0,78$                 & $0,48$                      \\
                     & FP-Rate      & $0,24$             & $0,74$                 & $0,49$                     & $0,22$             & $0,82$                 & $0,47$                      \\
                     & Precision    & $0,01$             & $1,00$                 & $0,51$                     & $0,00$             & $1,00$                 & $0,50$                      \\
                     & F-Score      & $0,01$             & $0,86$                 & $0,44$                     & $0,01$             & $0,87$                 & $0,44$                      \\ 
\hline
\multirow{4}{*}{LR}  & TPR / Recall & $0,32$             & $0,89$                 & $0,61$                     & $0,28$             & $0,88$                 & $0,58$                      \\
                     & FP-Rate      & $0,12$             & $0,68$                 & $0,40$                     & $0,12$             & $0,72$                 & $0,42$                      \\
                     & Precision    & $0,01$             & $1,00$                 & $0,51$                     & $0,01$             & $1,00$                 & $0,51$                      \\
                     & F-Score      & $0,03$             & $0,94$                 & $0,49$                     & $0,02$             & $0,94$                 & $0,48$                      \\ 
\hline
\multirow{4}{*}{NB}  & TPR / Recall & $0,68$             & $0,14$                 & $0,41$                     & $0,74$             & $0,14$                 & $0,44$                      \\
                     & FP-Rate      & $0,86$             & $0,32$                 & $0,59$                     & $0,86$             & $0,27$                 & $0,57$                      \\
                     & Precision    & $0,00$             & $0,99$                 & $0,50$                     & $0,00$             & $0,99$                 & $0,50$                      \\
                     & F-Score      & $0,01$             & $0,25$                 & $0,13$                     & $0,01$             & $0,25$                 & $0,13$                      \\ 
\hline
\multirow{4}{*}{NN}  & TPR / Recall & $0,05$             & $0,80$                 & $0,43$                     & $0,01$             & $0,99$                 & $0,50$                      \\
                     & FP-Rate      & $0,20$             & $0,95$                 & $0,58$                     & $0,01$             & $0,99$                 & $0,50$                      \\
                     & Precision    & $0,00$             & $0,99$                 & $0,50$                     & $0,01$             & $1,00$                 & $0,51$                      \\
                     & F-Score      & $0,00$             & $0,89$                 & $0,45$                     & $0,01$             & $0,99$                 & $0,50$                      \\ 
\hline
\multirow{4}{*}{RF}  & TPR / Recall & $0,06$             & $0,98$                 & $0,52$                     & $0,04$             & $0,98$                 & $0,51$                      \\
                     & FP-Rate      & $0,02$             & $0,94$                 & $0,48$                     & $0,02$             & $0,97$                 & $0,50$                      \\
                     & Precision    & $0,02$             & $1,00$                 & $0,51$                     & $0,01$             & $1,00$                 & $0,51$                      \\
                     & F-Score      & $0,03$             & $0,90$                 & $0,47$                     & $0,02$             & $0,99$                 & $0,51$                      \\ 
\hline
\multirow{4}{*}{SGD} & TPR / Recall & $0,20$             & $0,94$                 & $0,57$                     & $0,18$             & $0,94$                 & $0,56$                      \\
                     & FP-Rate      & $0,06$             & $0,81$                 & $0,44$                     & $0,06$             & $0,82$                 & $0,44$                      \\
                     & Precision    & $0,02$             & $1,00$                 & $0,51$                     & $0,02$             & $1,00$                 & $0,51$                      \\
                     & F-Score      & $0,03$             & $0,97$                 & $0,50$                     & $0,03$             & $0,97$                 & $0,50$                      \\ 
\hline
\multirow{4}{*}{SVM} & TPR / Recall & $0,20$             & $0,95$                 & $0,58$                     & $0,19$             & $1,00$                 & $0,60$                      \\
                     & FP-Rate      & $0,05$             & $0,81$                 & $0,43$                     & $0,05$             & $0,81$                 & $0,43$                      \\
                     & Precision    & $0,02$             & $1,00$                 & $0,51$                     & $0,02$             & $1,00$                 & $0,51$                      \\
                     & F-Score      & $0,04$             & $0,97$                 & $0,51$                     & $0,04$             & $0,98$                 & $0,51$                      \\
\hline
\end{tabular}
}
\end{table}

Die Ergebnisse der weiteren Evaluationsmetriken beider Datensets zeigen erneut den Trend, dass nur wenige defekte Datenpunkte auch tatsächlich als defekt vorhergesagt wurden. Mit TP-Raten von $68\%$ beziehungsweise $74\%$ erreichten die NB-Klassifikatoren die höchsten Trefferquoten. Die weiteren Klassifikatoren liegen mit Werten von höchstens $32\%$ deutlich darunter. Die weiteren Metriken der \glqq defekt\grqq -Spalten repräsentieren ebenfalls äußerst schwache Ergebnisse. Auffällig sind die sehr hohen FP-Raten der NB-Klassifikatoren, die die guten Ergebnisse der TP-Raten deutlich abschwächen.

Die Werte der \glqq fehlerfrei\grqq -Spalten zeigen sehr gute Werte. Lediglich die jeweiligen FP-Raten mit über $68\%$ sind deutlich zu hoch und zeigen, dass die Klassifikatoren viele Datenpunkte inkorrekterweise als \glqq fehlerfrei\grqq{} vorhersagen. Eine Ausnahme davon bilden erneut die NB-Klassifikatoren, die zwar gute Werte bezüglich der FP-Raten und der Precision erzielten, allerdings Mängel bei den TP-Raten und folglich auch dem F-Score aufweisen. Zudem lassen sich die verschiedenen Accuracies der NN-Klassifikatoren mit den Ergebnissen der weiteren Evaluationsmetriken begründen. Es ist zu erkennen, dass sich zwischen den Datensets die Werte der TP-Raten und FP-Raten für beide Label unterscheiden. Infolgedessen weichen auch die F-Scores ab. Diese Abweichungen der Werte führen zu den unterschiedlichen Werten der Accuracies.

Die Mittelwerte zeigen, dass die Gesamtergebnisse durch die Werte der \glqq defekt\grqq{} -Spalten stark nach unten beeinflusst werden. Sie zeigen, dass alle Klassifikatoren auf einem durchschnittlichen bis unterdurchschnittlichen Niveau Vorhersagen treffen.

\paragraph{ROC-Kurven und -Bereiche}

Die ROC-Kurven samt ihrer zugehörigen ROC-Bereiche (als \glqq AUC\grqq{} vermerkt) der dateibasierten Datensets sind in \autoref{roc-file} zu sehen. Die dargestellten Farbverläufe der Kurven vermitteln erneut keine für diesen Zweck relevanten Informationen und können somit ignoriert werden.

Ein erster Blick auf die ROC-Kurven der Klassifikatoren zeigt ein relativ einheitliches Bild von Kurven, die sich sehr stark an der Winkelhalbierenden orientieren und zwischen den Datensets einheitlich sind. Mit der Ausnahme der RF Klassifikatoren beider Datensets zeigen die Kurven mit ihren zugehörigen ROC-Bereichen eine unterdurchschnittliche bis unerwünschte Performanz (zufälliges Raten der Label) an. Mit einem ROC-Bereich von $0,71$ besitzen die RF-Klassifikatoren eine überdurchschnittliche Performanz im übergeordneten Vergleich.

Der Vergleich zwischen den NN-Klassifikatoren zeigt, dass der Klassifikator des einfachen Datensets das unerwünschte Verhalten zeigt und gleichzeitig inverse Vorhersagen trifft. Dies bedeutet, dass er in manchen Fällen statt dem Label \glqq fehlerfrei\grqq{} das Label \glqq defekt\grqq{} vorhersagt und umgekehrt. Die Kurve sowie der ROC-Bereich des Klassifikators des erweiterten Datensets zeigen hingegen nur das unerwünschte Verhalten in Form des zufälligen Ratens der Vorhersagen. Dies zeigt, dass die in \autoref{fig:final-eval} dargestellten hohen Accuracies der NN-Klassifikatoren wahrscheinlich durch Zufall entstanden sind. In einer weiteren Anwendung der Testdaten hätten die Ergebnisse anders ausfallen können.

\begin{figure}[h!t]
  \centering
  \subfloat[][J48 einf.\\AUC = $0,61$]{\includegraphics[width=0.25\linewidth]{images/j48_eval}} 
  \subfloat[][J48 erw.\\AUC = $0,61$]{\includegraphics[width=0.25\linewidth]{images/j48_eval_feat}}
  \subfloat[][NN einf.\\AUC = $0,37$]{\includegraphics[width=0.25\linewidth]{images/nn_eval}}
  \subfloat[][NN erw.\\AUC = $0,54$]{\includegraphics[width=0.25\linewidth]{images/nn_eval_feat}}
  \qquad
  \subfloat[][KNN einf.\\AUC = $0,55$]{\includegraphics[width=0.25\linewidth]{images/knn_eval}}
  \subfloat[][KNN erw.\\AUC = $0,52$]{\includegraphics[width=0.25\linewidth]{images/knn_eval_feat}}
  \subfloat[][RF einf.\\AUC = $0,71$]{\includegraphics[width=0.25\linewidth]{images/rf_eval}} 
  \subfloat[][RF erw.\\AUC = $0,71$]{\includegraphics[width=0.25\linewidth]{images/rf_eval_feat}} 
  \qquad
  \subfloat[][LR einf.\\AUC = $0,65$]{\includegraphics[width=0.25\linewidth]{images/lr_eval}}
  \subfloat[][LR erw.\\AUC = $0,62$]{\includegraphics[width=0.25\linewidth]{images/lr_eval_feat}}
  \subfloat[][SGD einf.\\AUC = $0,57$]{\includegraphics[width=0.25\linewidth]{images/sgd_eval}}
  \subfloat[][SGD erw.\\AUC = $0,56$]{\includegraphics[width=0.25\linewidth]{images/sgd_eval_feat}}
  \qquad
  \subfloat[][NB einf.\\AUC = $0,48$]{\includegraphics[width=0.25\linewidth]{images/nb_eval}}
  \subfloat[][NB erw.\\AUC = $0,47$]{\includegraphics[width=0.25\linewidth]{images/nb_eval_feat}}
  \subfloat[][SVM einf.\\AUC = $0,57$]{\includegraphics[width=0.25\linewidth]{images/svm_eval}}
  \subfloat[][SVM erw.\\AUC = $0,57$]{\includegraphics[width=0.25\linewidth]{images/svm_eval_feat}}
  \caption{ROC-Kurven der datenbasierten Datensets \label{roc-file}}
\end{figure}

\paragraph{Zusammenfassung}
Die Ergebnisse sämtlicher Metriken zeigten ein undifferenziertes und undeutliches Bild der Performanz der Klassifikatoren für beide Datensets. Der Grund für die gezeigten Ergebnisse ist die Unbalanciertheit der jeweiligen Testdatensets mit einem Anteil von \glqq defekt\grqq -Instanzen von nur $1\%$. Damit für dieses Label gute Ergebnisse erzielt werden können, müssen viele korrekte Vorhersagen der Klassifikatoren getroffen werden. Dies war jedoch für das vorhandene Testdatenset nicht der Fall, sodass viele Vorhersagen fälschlicherweise dem dominierenden Label \glqq fehlerfrei\grqq{} zugeordnet wurden, die dann allerdings infolgedessen hohe FP-Raten aufweisen. Diesen Umstand spiegeln auch die Mittelwerte in \autoref{tab:met-results} wider. Aufgrund der deutlich schwächeren Werte der \glqq defekt\grqq -Spalten, werden die Gesamtwerte, die durch den Mittelwert repräsentiert werden, nach unten beeinflusst. Wie jedoch schon in \hyperref[smote]{Abschnitt 4.2} erläutert wurde, ist es nicht vorgesehen, die Unbalanciertheit der Testdatensets mithilfe des SMOTE-Algorithmus auszugleichen.

Zusammenfassend lässt sich jedoch anhand der Ergebnisse der Tests der Klassifikatoren beider Datensets feststellen, dass der Einbezug der featurebasierten Metriken keinen nennenswert positiven Einfluss auf die Ergebnisse der dateibasierten Fehlervorhersage besitzt. Hinzugefügt werden muss jedoch, dass sie die Ergebnisse auch nicht negativ beeinflussen. Ein Grund dafür ist, dass die Anzahl an Attributen für beide dateibasierten Datensets sehr hoch ist. Das \glqq einfache\grqq{} Datenset umfasst 17 Attribute (+ Label) und das erweiterte Datenset umfasst 28 Attribute (+ Label). Eine Reduzierung der Attribute in Form einer Attributsauswahl könnte dazu beitragen, den Einfluss der featurebasierten Metriken näher zu analysieren. Die Attributsauswahl beschreibt einen automatisierten Prozess, welcher dazu dient, die optimale Anzahl und Auswahl von Attributen für das Training eines Klassifikators zu bestimmen. Die Durchführung einer solchen Attributsauswahl würde sich für zukünftige weiterführende Arbeiten anbieten.

\fbox{\parbox{\linewidth}{RQ3d: WIE BEEINFLUSST DIE VERWENDUNG VON FEATUREBASIERTEN METRIKEN DIE DATEIBASIERTE FEHLERVORHERSAGE?\medskip\\
Die Auswirkungen des Einbezugs der featurebasierten Metriken sind kaum messbar. Sowohl die Klassifikatoren des \glqq einfachen\grqq{} dateibasierten Datensets als auch die des erweiterten dateibasierten Datensets, erzielen die selben Ergebnisse mit geringen Abweichungen. Sie sind somit jeweils gleich performant bezüglich der Vorhersagen.}}

\cleardoublepage

%Um ein differenzierteres Bild der Performanz der Klassifikatoren zu erhalten, wurde eine weitere Testmethode als Alternative zur Aufteilung der Datensets in Training- und Testdaten hinzugezogen. Dabei handelt es sich um die sogenannte \glqq \texttt{n}-fold-cross-validation\grqq. Bei dieser Methode werden die Trainingsdaten in \texttt{n} gleich große Datenmengen (\glqq folds\grqq) aufgeteilt, welche dann jeweils mit einer Split-Ratio von $90:10$ zum Training der einzelnen Klassifikatoren dienen \cite{IanWitten}. Die Ergebnisse der Evaluationsmetriken bilden sich dann aus dem Mittelwert der Einzelergebnisse der \texttt{n} Vorgänge \cite{IanWitten}. Diese Methode findet auch in der wissenschaftlichen Literatur Anwendung (zum Beispiel \cite{Alam2013,Chawla2002,Alsaeedi2019}). Für diesen Fall wird eine 10-fold-cross-validation durchgeführt. Die Ergebnisse der Evaluationsmetriken sind in \autoref{tab:kx} aufgeführt. Ein Überblick über die Ergebnisse zeigt, dass diese wesentlich differenzierter und nachvollziehbarer ausgefallen sind sowie keine signifikanten Unterschiede zwischen den Datensets aufweisen. Es zeigt sich, dass die beiden Entscheidungsbaum-basierten Klassifikatoren J48 und RF die im Vergleich höchste Performanz mit einer Accuracy von $96\%$ beziehungsweise $98\%$ besitzen. Die FP-Rate der Klassifikatoren ist außerdem äußerst gering. Gleiches zeigt sich für den KNN-Klassifikator. Die weiteren Ergebnisse der Metriken bestätigen die hohe Performanz. Die Klassifikatoren SGD und SVM zeigen anhand ihrer ROC-Bereiche (siehe ROC-Area), dass sie unerwünschte Vorhersagen in Form des \glqq Ratens\grqq{} durchführen. Auffällig ist, dass die Werte zwischen den NN-Klassifikatoren beider Datensets nicht mehr abweichen. Beide Klassifikatoren arbeiten wie die weiteren, bisher unerwähnten, Klassifikatoren auf einem durchschnittlichen Niveau.
%
%\begin{table}[ht]
%\centering
%\caption{Ergebnisse der Evaluationsmetriken auf Basis der 10-fold-cross-validation (TPR = TP-Rate, Rec. = Recall)}
%\label{tab:kx}
%\resizebox{\linewidth}{!}{%
%\begin{tabular}{|>{\hspace{0pt}}p{0.06\linewidth}>{\hspace{0pt}}p{0.119\linewidth}|>{\RaggedLeft\hspace{0pt}}p{0.123\linewidth}>{\RaggedLeft\hspace{0pt}}p{0.17\linewidth}>{\RaggedLeft\hspace{0pt}}p{0.112\linewidth}|>{\RaggedLeft\hspace{0pt}}p{0.121\linewidth}>{\RaggedLeft\hspace{0pt}}p{0.168\linewidth}>{\RaggedLeft\hspace{0pt}}p{0.112\linewidth}|} 
%\cline{3-8}
%\multicolumn{1}{>{\hspace{0pt}}p{0.06\linewidth}}{} &           & \multicolumn{3}{>{\Centering\hspace{0pt}}p{0.405\linewidth}|}{\textbf{\glqq einfaches\grqq{} dateibasiertes Datenset} } & \multicolumn{3}{>{\Centering\hspace{0pt}}p{0.401\linewidth}|}{\textbf{erweitertes dateibasiertes Datenset} }  \\ 
%\cline{3-8}
%\multicolumn{1}{>{\hspace{0pt}}p{0.06\linewidth}}{} &           & \textbf{defekt}  & \textbf{fehlerfrei}                                 & \textbf{Mittel}                     & \textbf{defekt}  & \textbf{fehlerfrei}                                 & \textbf{Mittel}                      \\ 
%\hline
%\multirow{6}{0.06\linewidth}{\hspace{0pt}J48}       & Accuracy  & \multicolumn{2}{>{\RaggedLeft\hspace{0pt}}p{0.293\linewidth}}{gesamt:} & $0,96$                                & \multicolumn{2}{>{\RaggedLeft\hspace{0pt}}p{0.289\linewidth}}{gesamt:} & $0,96$                                 \\
%                                                    & TPR/Rec.  & $0,94$             & $0,97$                                                & $0,96$                                & $0,94$             & $0,97$                                                & $0,96$                                 \\
%                                                    & FP-Rate   & $0,03$             & $0,06$                                                & $0,05$                                & $0,03$             & $0,06$                                                & $0,05$                                 \\
%                                                    & Precision & $0,96$             & $0,96$                                                & $0,96$                                & $0,96$             & $0,96$                                                & $0,86$                                 \\
%                                                    & F-Score   & $0,95$             & $0,97$                                                & $0,96$                                & $0,95$             & $0,96$                                                & $0,96$                                 \\
%                                                    & ROC-Area  & $0,97$             & $0,97$                                                & $0,97$                                & $0,97$             & $0,97$                                                & $0,97$                                 \\ 
%\hline
%\multirow{6}{0.06\linewidth}{\hspace{0pt}KNN}       & Accuracy  & \multicolumn{2}{>{\RaggedLeft\hspace{0pt}}p{0.293\linewidth}}{gesamt:} & $0,91$                                & \multicolumn{2}{>{\RaggedLeft\hspace{0pt}}p{0.289\linewidth}}{gesamt:} & $0,91$                                 \\
%                                                    & TPR/Rec.  & $0,87$             & $0,94$                                                & $0,91$                                & $0,87$             & $0,94$                                                & $0,91$                                 \\
%                                                    & FP-Rate   & $0,06$             & $0,13$                                                & $0,10$                                & $0,06$             & $0,13$                                                & $0,10$                                 \\
%                                                    & Precision & $0,92$             & $0,91$                                                & $0,92$                                & $0,91$             & $0,91$                                                & $0,91$                                 \\
%                                                    & F-Score   & $0,89$             & $0,93$                                                & $0,91$                                & $0,89$             & $0,92$                                                & $0,91$                                 \\
%                                                    & ROC-Area  & $0,93$             & $0,93$                                                & $0,93$                                & $0,93$             & $0,93$                                                & $0,93$                                 \\ 
%\hline
%\multirow{6}{0.06\linewidth}{\hspace{0pt}LR}        & Accuracy  & \multicolumn{2}{>{\RaggedLeft\hspace{0pt}}p{0.293\linewidth}}{gesamt:} & $0,65$                                & \multicolumn{2}{>{\RaggedLeft\hspace{0pt}}p{0.289\linewidth}}{gesamt:} & $0,65$                                 \\
%                                                    & TPR/Rec.  & $0,34$             & $0,88$                                                & $0,61$                                & $0,35$             & $0,87$                                                & $0,61$                                 \\
%                                                    & FP-Rate   & $0,12$             & $0,66$                                                & $0,39$                                & $0,13$             & $0,65$                                                & $0,39$                                 \\
%                                                    & Precision & $0,66$             & $0,65$                                                & $0,66$                                & $0,66$             & $0,66$                                                & $0,66$                                 \\
%                                                    & F-Score   & $0,45$             & $0,75$                                                & $0,60$                                & $0,46$             & $0,75$                                                & $0,61$                                 \\
%                                                    & ROC-Area  & $0,74$             & $0,74$                                                & $0,74$                                & $0,74$             & $0,74$                                                & $0,74$                                 \\ 
%\hline
%\multirow{6}{0.06\linewidth}{\hspace{0pt}NB}        & Accuracy  & \multicolumn{2}{>{\RaggedLeft\hspace{0pt}}p{0.293\linewidth}}{gesamt:} & $0,52$                                & \multicolumn{2}{>{\RaggedLeft\hspace{0pt}}p{0.289\linewidth}}{gesamt:} & $0,52$                                 \\
%                                                    & TPR/Rec.  & $0,93$             & $0,23$                                                & $0,58$                                & $0,94$             & $0,22$                                                & $0,58$                                 \\
%                                                    & FP-Rate   & $0,77$             & $0,07$                                                & $0,42$                                & $0,78$             & $0,06$                                                & $0,42$                                 \\
%                                                    & Precision & $0,46$             & $0,83$                                                & $0,65$                                & $0,46$             & $0,84$                                                & $0,65$                                 \\
%                                                    & F-Score   & $0,62$             & $0,36$                                                & $0,49$                                & $0,62$             & $0,35$                                                & $0,49$                                 \\
%                                                    & ROC-Area  & $0,65$             & $0,65$                                                & $0,65$                                & $0,65$             & $0,65$                                                & $0,65$                                 \\ 
%\hline
%\multirow{6}{0.06\linewidth}{\hspace{0pt}NN}        & Accuracy  & \multicolumn{2}{>{\RaggedLeft\hspace{0pt}}p{0.293\linewidth}}{gesamt:} & $0,71$                                & \multicolumn{2}{>{\RaggedLeft\hspace{0pt}}p{0.289\linewidth}}{gesamt:} & $0,70$                                 \\
%                                                    & TPR/Rec.  & $0,55$             & $0,82$                                                & $0,69$                                & $0,54$             & $0,82$                                                & $0,68$                                 \\
%                                                    & FP-Rate   & $0,18$             & $0,45$                                                & $0,32$                                & $0,18$             & $0,46$                                                & $0,32$                                 \\
%                                                    & Precision & $0,69$             & $0,72$                                                & $0,71$                                & $0,68$             & $0,72$                                                & $0,70$                                 \\
%                                                    & F-Score   & $0,71$             & $0,77$                                                & $0,74$                                & $0,60$             & $0,76$                                                & $0,68$                                 \\
%                                                    & ROC-Area  & $0,78$             & $0,78$                                                & $0,78$                                & $0,79$             & $0,79$                                                & $0,79$                                 \\ 
%\hline
%\multirow{6}{0.06\linewidth}{\hspace{0pt}RF}        & Accuracy  & \multicolumn{2}{>{\RaggedLeft\hspace{0pt}}p{0.293\linewidth}}{gesamt:} & $0,98$                                & \multicolumn{2}{>{\RaggedLeft\hspace{0pt}}p{0.289\linewidth}}{gesamt:} & $0,98$                                 \\
%                                                    & TPR/Rec.  & $0,97$             & $0,99$                                                & $0,98$                                & $0,97$             & $0,99$                                                & $0,98$                                 \\
%                                                    & FP-Rate   & $0,01$             & $0,04$                                                & $0,03$                                & $0,01$             & $0,03$                                                & $0,02$                                  \\
%                                                    & Precision & $0,98$             & $0,98$                                                & $0,98$                                & $0,98$             & $0,98$                                                & $0,98$                                 \\
%                                                    & F-Score   & $0,97$             & $0,98$                                                & $0,98$                                & $0,97$             & $0,98$                                                & $0,98$                                 \\
%                                                    & ROC-Area  & $0,99$             & $0,99$                                                & $0,99$                                & $1,00$             & $1,00$                                                & $1,00$                                 \\ 
%\hline
%\multirow{6}{0.06\linewidth}{\hspace{0pt}SGD}       & Accuracy  & \multicolumn{2}{>{\RaggedLeft\hspace{0pt}}p{0.293\linewidth}}{gesamt:} & $0,63$                                & \multicolumn{2}{>{\RaggedLeft\hspace{0pt}}p{0.289\linewidth}}{gesamt:} & $0,63$                                 \\
%                                                    & TPR/Rec.  & $0,19$             & $0,94$                                                & $0,57$                                & $0,18$             & $0,94$                                                & $0,56$                                 \\
%                                                    & FP-Rate   & $0,06$             & $0,81$                                                & $0,44$                                & $0,06$             & $0,82$                                                & $0,44$                                 \\
%                                                    & Precision & $0,69$             & $0,62$                                                & $0,66$                                & $0,69$             & $0,62$                                                & $0,66$                                 \\
%                                                    & F-Score   & $0,30$             & $0,75$                                                & $0,53$                                & $0,29$             & $0,75$                                                & $0,52$                                 \\
%                                                    & ROC-Area  & $0,57$             & $0,57$                                                & $0,57$                                & $0,56$             & $0,56$                                                & $0,56$                                 \\ 
%\hline
%\multirow{6}{0.06\linewidth}{\hspace{0pt}SVM}       & Accuracy  & \multicolumn{2}{>{\RaggedLeft\hspace{0pt}}p{0.293\linewidth}}{gesamt:} & $0,62$                                & \multicolumn{2}{>{\RaggedLeft\hspace{0pt}}p{0.289\linewidth}}{gesamt:} & $0,62$                                 \\
%                                                    & TPR/Rec.  & $0,16$             & $0,95$                                                & $0,56$                                & $0,15$             & $0,95$                                                & $0,55$                                 \\
%                                                    & FP-Rate   & $0,05$             & $0,84$                                                & $0,45$                                & $0,05$             & $0,85$                                                & $0,46$                                 \\
%                                                    & Precision & $0,69$             & $0,62$                                                & $0,66$                                & $0,69$             & $0,61$                                                & $0,66$                                 \\
%                                                    & F-Score   & $0,26$             & $0,75$                                                & $0,51$                                & $0,25$             & $0,75$                                                & $0,50$                                 \\
%                                                    & ROC-Area  & $0,55$             & $0,55$                                                & $0,55$                                & $0,55$             & $0,55$                                                & $0,55$                                 \\
%\hline
%\end{tabular}
%}
%\end{table}

%\begin{table}[ht]
%\centering
%\caption{Ergebnisse der Attributsauswahl der Klassifikatoren LR, NB und SGD }
%\label{tab:attribute-selection}
%\resizebox{\linewidth}{!}{%
%\begin{tabular}{|l|l|l|} 
%\hline
%\textbf{Klassifikator} & \textbf{ausgewählte Attribute}                                                                                                    & \textbf{abgelehnte Attribute}                                                                                                                                                                         \\ 
%\hline
%LR                     & \begin{tabular}[c]{@{}l@{}} \texttt{\textbf{MODS}}, \texttt{\textbf{FADDL}}*, \texttt{\textbf{FREML}}*, \texttt{BUGF},\\\texttt{AUTH}, \texttt{ADDL}, \texttt{REML}, \texttt{CCHL},\\\texttt{CCHM}, \texttt{CCHA}, \texttt{AVGC}, \texttt{WAGE}\\(12 von 28)\end{tabular} & \begin{tabular}[c]{@{}l@{}}\texttt{COMM}, \texttt{ADEV}, \texttt{DDEV}, \texttt{EXP},\\\texttt{OEXP}, \texttt{MODD}, \texttt{NLOC}, \texttt{CYCO},\\\texttt{REVI}, \texttt{REFA}, \texttt{ADDM}, \texttt{ADDA},\\\texttt{REMM}, \texttt{REMA}, \texttt{MAXC}, \texttt{AAGE}\\(16 von 28)\end{tabular}                                                 \\ 
%\hline
%NB                     & \begin{tabular}[c]{@{}l@{}}\texttt{\textbf{MODS}}, \texttt{BUGF}, \texttt{AUTH}, \texttt{CCHM},\\\texttt{WAGE} (5 von 28)\end{tabular}                                                  & \begin{tabular}[c]{@{}l@{}}\texttt{COMM}, \texttt{ADEV}, \texttt{DDEV}, \texttt{EXP},\\\texttt{OEXP}, \texttt{MODD}, \texttt{NLOC}, \texttt{CYCO},\\\texttt{FADDL}*, \texttt{FREML}*, \texttt{REVI}, \texttt{REFA},\\\texttt{ADDL}, \texttt{ADDM}, \texttt{ADDA}, \texttt{REML},\\\texttt{REMM}, \texttt{REMA}, \texttt{CCHL}, \texttt{CCHA},\\\texttt{MAXC}, \texttt{AVGC}, \texttt{AAGE} (23 von 28)\end{tabular}  \\ 
%\hline
%SGD                    & \begin{tabular}[c]{@{}l@{}}\texttt{\textbf{MODS}}, \texttt{\textbf{FREML}}*, \texttt{REVI}, \texttt{REFA}\\\texttt{BUGF}, \texttt{AUTH}, \texttt{CCHM}, \texttt{CCHA},\\\texttt{AVGC}, \texttt{AAGE}, \texttt{WAGE} (11 von 28)\end{tabular}           & \begin{tabular}[c]{@{}l@{}}\texttt{COMM}, \texttt{ADEV}, \texttt{DDEV}, \texttt{EXP},\\\texttt{OEXP}, \texttt{MODD}, \texttt{NLOC}, \texttt{CYCO},\\\texttt{FADDL}*, \texttt{ADDL}, \texttt{ADDM}, \texttt{ADDA},\\\texttt{REML}, \texttt{REMM}, \texttt{REMA}, \texttt{CCHL},\\\texttt{MAXC} (17 von 28)\end{tabular}                                         \\ 
%\hline
%\multicolumn{3}{|c|}{* ADDL- und REML-Metriken des featurebasierten Datensets}                                                                                                                                                                                                                                                                                      \\
%\hline
%\end{tabular}
%}
%\end{table}