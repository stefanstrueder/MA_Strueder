% !TEX root = ../thesis.tex

% Kurzfassung in Deutsch und Englisch
\begin{otherlanguage}{ngerman}
    \section*{Kurzfassung}

Softwarefehler sind ein großes Ärgernis in der Softwareentwicklung und können nicht nur zu Rufschädigungen sondern auch zu erheblichen finanziellen Schäden für Unternehmen führen. Aus diesem Grund wurden im vergangenen Jahrzehnt zahlreiche Techniken zur Erkennung und Vorhersage von Fehlern entwickelt, welche zum großen Teil auf Methoden des Machine Learnings basieren. Die übliche Herangehensweise dieser Techniken erfolgt auf der Vorhersage von Fehlern auf Dateiebene. Seit einigen Jahren steigt jedoch die Popularität von featurebasierter Softwareentwicklung: ein Paradigma welches auf Funktionsinkremente eines Softwaresystems (Features) setzt und somit für eine breite Variabilität des Softwareproduktes sorgt. Eine gängige Implementationstechnik für Features basiert auf Annotationen mit Präprozessoranweisungen, wie \texttt{\#IFDEF} und \texttt{\#IFNDEF}, deren Code sich über mehrere Dateien der Quellcodedateien der Software verteilt (\glqq code scattering)\grqq. Ein Fehler in solchem Featureode kann aufgrund dessen weitreichende Folgen für die Funktionalität der gesamten Software haben. Weist ein Teil des Featurecodes Fehler auf, so wird die gesamte Funktion des Features fehlerhaft und führt unter Umständen zum Ausfall der gesamten Funktionalität der Software. An dieses Problem knüpft diese Arbeit an. Es wird eine Vorhersagetechnik für fehlerhafte Features entwickelt, welche auf Methoden des Machine Learnings basiert. Die Auswertung von acht Klassifikatoren, welche jeweils auf einem induviduellen Klassifikationsalgorithmus basieren, zeigt, dass mithilfe des für diese Arbeit erstellten featurebasierten Datensets, eine Genauigkeit von bis zu 92\% für die Vorhersage von fehlerhaften oder fehlerfreien Features erreicht werden konnte. Es wird zudem gezeigt, wie der Aspekt der Featureorientierung im Rahmen der Erstellung des Datensets eingebunden wurde und welche Resultate im Vergleich zur herkömmlichen dateibasierten Methodik erzielt werden konnten.

\end{otherlanguage}

\begin{otherlanguage}{english}
    \section*{Abstract}
    
Software errors are a major nuisance in software development and can lead not only to damage of reputation but also to considerable financial losses for companies. For this reason, numerous techniques for detecting and predicting errors have been developed over the past decade, which are largely based on machine learning methods. The usual approach of these techniques is to predict errors at file level. For some years now, however, the popularity of feature-based software development has been increasing - a paradigm that relies on function increments of a software system (features) and thus ensures a wide variability of the software product. A common implementation technique for features is based on annotations with preprocessor instructions, such as \texttt{\#IFDEF} and \texttt{\#IFNDEF}, whose code is spread over several files of the software's source code files (\glqq code scattering\grqq). A bug in such a feature code can have far-reaching consequences for the functionality of the entire software. If a part of the feature code contains errors, the entire function of the feature becomes faulty and may lead to the failure of the entire functionality of the Software. This problem is the subject of this thesis. A prediction technique for faulty features is developed, which is based on methods of machine learning. The evaluation of eight classifiers, each based on an individual classification algorithm, shows that the feature-based data set created for this thesis allows an accuracy of up to 92\% for the prediction of faulty or error-free features. It is also shown how the feature orientation aspect was incorporated into the creation of the dataset and what results were achieved compared to the traditional file-based methodology.

\end{otherlanguage}
\cleardoublepage
