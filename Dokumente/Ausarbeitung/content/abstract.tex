% !TEX root = ../thesis.tex

% Kurzfassung in Deutsch und Englisch
\begin{otherlanguage}{ngerman}
    \section*{Kurzfassung}

Softwarefehler sind ein großes Ärgernis in der Softwareentwicklung und können nicht nur zu Rufschädigungen, sondern auch zu erheblichen finanziellen Schäden für Unternehmen führen. Aus diesem Grund wurden im vergangenen Jahrzehnt zahlreiche Techniken zur Vorhersage von Fehlern entwickelt, welche zum großen Teil auf Methoden des Machine Learnings basieren. Diese Techniken zielen üblicherweise auf die Vorhersage von Fehlern auf Dateiebene ab. Seit einigen Jahren steigt jedoch die Popularität von featurebasierter Softwareentwicklung: ein Paradigma, welches auf Funktionsinkremente eines Softwaresystems (Features) setzt und somit für eine breite Variabilität des Softwareproduktes sorgt. Eine gängige Implementationstechnik für Features basiert auf Annotationen mit Präprozessoranweisungen, wie \texttt{\#IFDEF} und \texttt{\#IFNDEF}, deren Code sich über mehrere Dateien des Quellcodes der Software verteilt (\glqq code scattering\grqq). Ein Fehler in solchem Featurecode kann aufgrund dessen weitreichende Folgen für die Funktionalität der gesamten Software haben. Weist ein Teil des Featurecodes Fehler auf, so wird die gesamte Funktion des Features fehlerhaft und führt unter Umständen zum Ausfall der gesamten Funktionalität der Software (Features sind \glqq cross-cutting\grqq, d.h. dateiübergreifend). An dieses Problem knüpft diese Masterarbeit an. Es wird eine Vorhersagetechnik für fehlerhafte und fehlerfreie Features entwickelt, welche auf Methoden des Machine Learnings basiert. Die Auswertung von acht Klassifikatoren, welche jeweils auf einem individuellen Klassifikationsalgorithmus basieren, zeigt, dass mithilfe des für diese Arbeit erstellten featurebasierten Datensets, eine Genauigkeit von bis zu $84\%$ für die Vorhersage von fehlerhaften oder fehlerfreien Features erreicht werden konnte. Es wird zudem gezeigt, wie der Aspekt der Featureeinbezugs im Rahmen der Erstellung des Datensets eingebunden wurde und welche Resultate im Vergleich zur herkömmlichen dateibasierten Methodik erzielt werden konnten. Dieser Vergleich zeigte jedoch, dass der zusätzliche Einbezug des Featureaspekts in die dateibasierte Fehlervorhersage keinen signifikanten Einfluss auf die Vorhersageergebnisse hat.

\end{otherlanguage}

\begin{otherlanguage}{english}
    \section*{Abstract}
    
Software errors are a major nuisance in software development and can lead not only to damage to reputation but also to considerable financial losses for companies. For this reason, numerous techniques for predicting defects have been developed over the past decade, which are largely based on machine learning methods. These techniques are usually aimed at predicting defects at the file level. However, in recent years the popularity of feature-based software development has been increasing: a paradigm that relies on functional increments of a software system (features) and thus ensures a wide variability of the software product. A common implementation technique for features is based on annotations with preprocessor instructions, such as \texttt{\#IFDEF} and \texttt{\#IFNDEF}, whose code is spread over several files of the software source code (\glqq code scattering\grqq). A bug in such feature code can have far-reaching consequences for the functionality of the entire software. If a part of the feature code contains defects, the entire function of the feature becomes faulty and may lead to the failure of the entire functionality of the software (features are \glqq cross-cutting\grqq). This problem is the subject of this master thesis. A prediction technique for defective and defect-free features is developed, which is based on machine learning methods. The evaluation of eight classifiers, each based on an individual classification algorithm, shows that the feature-based dataset created for this thesis allows an accuracy of up to $84\%$ for the prediction of defective or defect-free features. It is also shown how the feature oriented aspect was integrated into the creation of the dataset and what results were achieved compared to the traditional file-based methodology. However, this comparison showed that the additional inclusion of the feature aspect in the file-based defect prediction does not have a significant impact on the prediction results.

\end{otherlanguage}
\cleardoublepage
